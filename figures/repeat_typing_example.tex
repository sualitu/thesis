\begin{figure}[h]

  
\subcaptionbox{
The type check takes place in this checking environment. From the top the
environment is: The recursive reference is $repeat\ a$, which is a causal
(\causal) function, and due to the earlier described preprocessing, the clock starts as
open (\open). Finally the typing environment $\Gamma$, is extended as shown.
}{
$\begin{matrix*}[l]CE = & _g\text{repeat\ a};
  \\ & \causal ;\\ &

 \open;\\ &
\Gamma,\text{a} : \text{Type},\ \text{n} : \text{a},\,_g\text{repeat} : (\text{a}\ :\ \text{Type}) \to \forall \kappa.(\text{a} \to \onk{\text{Stream}}\ \text{a}),\ 
\\ & \ \ \ _{g}::\ : (\text{a}\ :\ \text{Type})
\to \text{a} \to \later^\kappa_1 \onk{\text{Stream}}\ \text{a} \to \onk{\text{Stream}}\ \text{a}
\end{matrix*}$
}
\centering
\subcaptionbox{
Starting with the inferred term in the conclusion, we first apply the $App$
rule, an Idris typing rule, as we have a regular application between $(_g::)\ a\
n$ and $(\onk{Next}\,((_grepeat\ a)[\kappa])) \tensor^\kappa_1
(\onk{Next}\,n)$. Repeating the application rule we end up having to show that
$n$ is of type $a$ which we know from the environment.
}{
\AxiomC{(1)}
\AxiomC{}
\RightLabel{$Var_1$}
\UnaryInfC{$CE \vdash n : a$}
\RightLabel{$App$}
\BinaryInfC{$CE \vdash (_g::)\ a\ n : \later^\kappa_1 \onk{Stream}\ a \to
  \onk{Stream}\ a$}
\AxiomC{(2)}
\RightLabel{$App$}
\BinaryInfC{$CE \vdash \,(_g::)\,a\,n\,((\onk{Next}\,((_grepeat\ a)[\kappa])) \tensor^\kappa_1 (\onk{Next}\,
n)) : Stream^\kappa \,a$}
\DisplayProof
}
\vspace{1em}
\subcaptionbox{
This application is taken apart in the same way, leaving us with two things we
know from the environment.
}{
\AxiomC{}
\RightLabel{$Var_{1_\open}$}
\UnaryInfC{$CE \vdash (_g::) : (a : Type) \to a \to \later^\kappa_1 \onk{Stream}\ a \to
  \onk{Stream}\ a$}
\AxiomC{}
\RightLabel{$Var_1$}
\UnaryInfC{$CE \vdash a : Type$}
\LeftLabel{(1)}
\RightLabel{$App$}
\BinaryInfC{$CE \vdash (_g::)\ a : a \to \later^\kappa_1 \onk{Stream}\ a \to
  \onk{Stream}\ a$}
\DisplayProof
}
\vspace{1em}
\subcaptionbox{
Here, we start by applying the $\tensorkappan$-rule, leaving us with two new
obligations, both $\laterkappan$. To the first we can apply the rule for the
recursive references rule in the causal case. Lastly, the $\onk{Next}\ a$-case
is trivially shown by the $I_{\onk{Next}}$-rule and from the environment.
}
{
\AxiomC{}
\RightLabel{$Var_1$}
\UnaryInfC{$CE \vdash \ _grepeat\ a : \forall \kappa . (a \to \onk{Stream}\ a)$}
\RightLabel{$Rec_\causal$}
\UnaryInfC{$CE \vdash (\onk{Next} (_grepeat\ a))[\kappa] : \later^\kappa_1 (a \to \onk{Stream}\ a)$}
\AxiomC{}
\RightLabel{$Var_1$}
\UnaryInfC{$CE \vdash n : a$}
\RightLabel{$I_{\onk{Next}}$}
\UnaryInfC{$CE \vdash \onk{Next}\ n : \later^\kappa_1 a$}
\LeftLabel{(2)}
\RightLabel{$I_{\tensorkappan}$}
\BinaryInfC{$CE \vdash (\onk{Next} (_grepeat\ a))[\kappa] \tensor^\kappa_1 (\onk{Next}\,
n) : \later^\kappa_1 \onk{Stream}\ a$}
\DisplayProof
}
  \caption{Proof of repeat.}
  \label{fig:repeat_typing_example}
\end{figure}
%%% Local Variables:
%%% mode: latex
%%% TeX-master: "../copatterns-thesis"
%%% End:
