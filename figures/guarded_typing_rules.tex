% New things in env
\begin{figure}
\textbf{The Typing Rules}
\begin{center}
\subcaptionbox{
Looking at the clock environment in the premise $\Delta, \kappa$ it is clear
that this translates to the open clock ($\open$). In the conclusion we translate
$\Delta$ to the closed clock, because $\Delta$ without $\kappa$ is empty as $\kappa$ is
the only clock in a singleton clock system. The side condition $\kappa \notin fc(\Gamma)$ translates to
$nofc(\Gamma)$, because if the only clock, $\kappa$, is not allowed to
be free in $\Gamma$ then no clock is allowed free in $\Gamma$.
}{

\AxiomC{$\Delta, \kappa ; \Gamma \vdash A : Type$}
\AxiomC{$\kappa \notin fc(\Gamma)$}
\BinaryInfC{$\Delta ; \Gamma \vdash \forall \kappa . A : Type$}
\DisplayProof
\quad
    \AxiomC{$\CEopen \vdash A:Type$}
    \AxiomC{$nofc(\Gamma)$}
    \RightLabel{$I_\forall$}
    \BinaryInfC{$\CEclosed \vdash \forall \kappa . A : Type$}
\DisplayProof
}
\vspace{1em}
\subcaptionbox{
The interesting thing to note in this rule is $\kappa \in \Delta$. If $\kappa$ is in
$\Delta$ and $\kappa$ is the only clock, then $\Delta$ must translate to the
open clock for this side condition to hold. This eliminates the side condition, as $\kappa \in \open$ is always true.
}{
\AxiomC{$\Delta ; \Gamma \vdash A : Type$}
\AxiomC{$\kappa \in \Delta$}
\BinaryInfC{$\Delta ; \Gamma \vdash \laterkappa A : Type$}
\DisplayProof
\quad
    \AxiomC{$\CEopen \vdash A:Type$}
    \RightLabel{$I_{\laterkappan}$}
    \UnaryInfC{$\CEopen \vdash \laterkappan A : Type$}
\DisplayProof  
\vspace{1em}
}
\vspace{1em}
\subcaptionbox{
The only change in this rule is how we consider the environment. We translate
$\Delta$ to the open clock, as $\onk{Next}$ implies a need for
a clock.
}{
\AxiomC{$\Delta ; \Gamma \vdash t : A$}
\UnaryInfC{$\Delta ; \Gamma \vdash \onk{Next}\ t : \laterkappa A$}
\DisplayProof
\quad
    \AxiomC{$\CEopen \vdash t : A$}
    \RightLabel{$I_{\onk{Next}}$}
    \UnaryInfC{$\CEopen \vdash \onk{Next}\ t : \laterkappan A$}
\DisplayProof
\vspace{1em}
}
\vspace{1em}
\subcaptionbox{
For the same reason as in the $I_\forall$-rule, $\Delta, \kappa$ becomes the open clock, and
just $\Delta$ becomes the closed clock. As before, $\kappa \notin fc(\Gamma)$
becomes $nofc(\Gamma)$.
}{
\AxiomC{$\Delta, \kappa ; \Gamma \vdash t : A$}
\AxiomC{$\kappa \notin fc(\Gamma)$}
\BinaryInfC{$\Delta ; \Gamma \vdash \Lambda \kappa. t : \forall \kappa. A$}
\DisplayProof
\quad
\AxiomC{$\CEopen \vdash t : A$}
\AxiomC{$nofc(\Gamma)$}
    \RightLabel{$I_{\Lambda \kappa}$}
\BinaryInfC{$\CEclosed \vdash \Lambda \kappa . t : \forall \kappa . A$}
\DisplayProof
}

\AxiomC{$\kappa \notin fc(\Gamma)$}
\AxiomC{$\Delta, \kappa ; \Gamma \vdash A : Type$}
\AxiomC{$\Delta, \kappa' ; \Gamma, \Gamma' \vdash t : \forall \kappa. A$}
\TrinaryInfC{$\Delta, \kappa' ; \Gamma , \Gamma ' \vdash t[\kappa'] : A[{  \kappa'}/{\kappa  }]$}
\DisplayProof
\vspace{1em}

\subcaptionbox{
Since, originally, this rule mentions two different clocks $\kappa$ and
$\kappa'$, we have to look at the premises and the conclusion  one by one to
make sense of them in a singleton clock environment. If $\forall \kappa . A$ is
to be a type, then $A$ must be a type under the open clock. If $t$ is to be of type
$\forall \kappa . A$ it must open a clock, according to $I_{\Lambda \kappa}$,
and thusly be checkable under the closed clock. And finally the
conclusion, if we apply a clock on a term, then a clock exist, thus a clock must
be open. Furthermore, the clock substitution on $A$ makes no sense with only one
clock. As usual,  $\kappa \notin fc(\Gamma)$ becomes $nofc(\Gamma)$.
}{
\AxiomC{$nofc(\Gamma)$}
\AxiomC{$\CEopen \vdash A : Type$}
\AxiomC{$\CEclosed ,\Gamma' \vdash t : \forall \kappa . A$}
    \RightLabel{$I_{[\kappa]}$}
\TrinaryInfC{$\CEopen ,\Gamma ' \vdash t[\kappa] : A$}
\DisplayProof
}
\end{center}
\vspace{2em}
\end{figure}

\begin{figure}
\begin{center}
\ContinuedFloat
\AxiomC{$\Delta ; \Gamma \vdash t : \laterkappa (A \to B)$}
\AxiomC{$\Delta ; \Gamma \vdash u : \laterkappa A$}
\BinaryInfC{$\Delta ; \Gamma \vdash t \tensor^\kappa u : \laterkappa B$}
\DisplayProof

\vspace{1em}
\subcaptionbox{
While there is nothing that states anything about the clock environment in the
original rule, both the later types ($\later$) and later composition ($\tensor$)
requires a clock to make sense. Because of this we require the clocks to be
open.
}{

    \AxiomC{$\CEopen \vdash t : \laterkappan (A \to B) $}
    \AxiomC{$\CEopen \vdash u : \laterkappan A$}
    \RightLabel{$I_{\tensorkappan}$}    
    \BinaryInfC{$\CEopen \vdash t \tensorkappan u : \laterkappan B$}
\DisplayProof  
}
\vspace{1em}
\subcaptionbox{
This rule does not have a direct counter part in the original model, but is
rather a consequence of our fixed point elimination from
Section~\ref{sec:fixkappa-rule}. A causal recursive reference must be on the
form $\onk{Next}\ e[\kappa]$, according to the rule in
Figure~\ref{fig:fix_elim_rules}. This ensures the correct lateness compared to
the one of $\iota$, given that $e = \iota$ and $\iota : \forall \kappa . A$.
}{
\AxiomC{$\CEopencausal \vdash \iota : \forall \kappa . A$}
\RightLabel{$e = \iota \quad Rec_{\causal}$}
\UnaryInfC{$\CEopencausal \vdash \onk{Next}\ e[\kappa] : \laterkappan A$}
\DisplayProof
}
\vspace{1em}
\subcaptionbox{To recover from eliminating the indexed fixed point, this rule ensures that
  all arguments to the non-causal recursive references has a $\later^\kappa_1$
  type.}{
\AxiomC{$\begin{matrix}\CEopennoncausal \vdash (\onk{Next}\ e) \tensor^\kappa_1
    (\onk{Next}\ x_0) \tensor^\kappa_1 \\
    \cdots \tensor^\kappa_1 (\onk{Next}\ x_{m-1}) : \later^\kappa_1(A \to B)
\\ \CEopennoncausal \vdash (\onk{Next}\ x_n) : \later^\kappa_1 A\end{matrix}$}
\RightLabel{$\begin{matrix} n > 0 \\ e = \iota\end{matrix} \quad Rec_{\noncausal_n}$}
\UnaryInfC{$\CEopennoncausal \vdash (\onk{Next}\ e) \tensor^\kappa_1 (\onk{Next}\ x_0) 
  \cdots \tensor^\kappa_1 (\onk{Next}\ x_n) : \later^\kappa_1B$}
\DisplayProof
}
\vspace{1em}
\subcaptionbox{Here, the non-causal references with one argument is
  handled. This rule is necessary as a base case for the $Rec_{\noncausal_n}$ rule.}{
\AxiomC{$\begin{matrix}\CEopennoncausal \vdash (\onk{Next}\ e) : \later^\kappa_1(A \to B)\end{matrix}$}
\AxiomC{$\CEopennoncausal \vdash (\onk{Next}\ x_0) : \later^\kappa_1 A$}
\RightLabel{$e = \iota \quad Rec_{\noncausal_0}$}
\BinaryInfC{$\CEopennoncausal \vdash (\onk{Next}\ e) \tensor^\kappa_1 (\onk{Next}\ x_0) : \later^\kappa_1B$}
\DisplayProof
}
\vspace{1em}
\subcaptionbox{
This rule ensures that the recursive reference has the correct lateness compared
to $\iota$.
}{
\AxiomC{$\CEopennoncausal \vdash \iota : A$}
\RightLabel{$e = \iota \quad Rec_{\noncausal}$}
\UnaryInfC{$\CEopennoncausal \vdash \onk{Next}\ e : \laterkappan A$}
\DisplayProof
}
\vspace{1em}

\subcaptionbox{
This is an extension of the Idris $Var$ rules. They state that types with a free
clock can only occur if clock is open.
}{
\AxiomC{$(\lambda t : \onk{A}) \in \Gamma$}
\RightLabel{$Var_{1_\open}$}
\UnaryInfC{$\CEopen \vdash t : \onk{A}$}
\DisplayProof

\AxiomC{$(\forall t : \onk{A}) \in \Gamma$}
\RightLabel{$Var_{2_\open}$}
\UnaryInfC{$\CEopen \vdash t : \onk{A}$}
\DisplayProof
}

\end{center}
  \caption{Guarded Recursion rules with singleton clock.}
  \label{fig:gr_rules_sin_clock}

\end{figure}
%%% Local Variables:
%%% mode: latex
%%% TeX-master: "../copatterns-thesis"
%%% End:
