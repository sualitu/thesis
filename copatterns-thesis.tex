\documentclass{ituthesis}

\settitle{Copatterns in Idris}
\setauthor{Sune Alk\ae rsig \& Thomas Hallier Didriksen}
\setsupervisor{Peter Sestoft}
\setextrasupervisor{David Christiansen}
\setdate{March 2015}


\lstset{
  morekeywords={
    total, 
    causal,
    module,
    corecord,
    where,
    data,
    record,
    codata,
    open,
    coinductive,
    field,
    default
  },
  keywordstyle=\bfseries\rmfamily
}
\usepackage{mdframed}
\usepackage{float}
\usepackage[utf8]{inputenc}
\usepackage{url}
\usepackage{alltt}
\usepackage{amssymb, amsfonts, amsmath, amsthm}
\usepackage{mathtools}
\usepackage{listings}
\usepackage{natbib}
\usepackage{pdflscape}
\usepackage{todonotes}
\bibliographystyle{alpha}
\usepackage{semantic}
\usepackage{fullpage}
\usepackage[normalem]{ulem}
\usepackage{caption}
\usepackage{framed}
\usepackage{subcaption}
\usepackage{bussproofs}
 % Macros to get display-style math in proof trees.
  \newcommand{\AXD}[1]{\AxiomC{\ensuremath{\displaystyle #1}}}
  \newcommand{\UID}[1]{\UnaryInfC{\ensuremath{\displaystyle #1}}}
  \newcommand{\BID}[1]{\BinaryInfC{\ensuremath{\displaystyle #1}}}
  \newcommand{\TID}[1]{\TrinaryInfC{\ensuremath{\displaystyle #1}}}
  \newcommand{\QID}[1]{\QuaternaryInfC{\ensuremath{\displaystyle #1}}}
  \newcommand{\RLabel}[1]{\RightLabel{\ensuremath{#1}}}

\newcommand{\IdrisM}{Idris$^{-}$}
\newcommand{\later}{\rhd}
% Kappa commands
\newcommand{\onk}[1]{#1^\kappa}
\newcommand{\laterkappa}{\later^{\!\!\kappa}}
\newcommand{\laterkappan}{\later^{\!\!\kappa}_{\!\! n}}
\newcommand{\laterk}[1]{\later^{\!\!\!\kappa}#1^\kappa}
\newcommand{\forallk}[1]{\forall\kappa.#1^\kappa}
\newcommand{\flaterk}[1]{\forall\kappa.\later^{\!\!\!\kappa}#1^\kappa}
% Kappa prime commands
\newcommand{\onkp}[1]{#1^\kappa'}
\newcommand{\laterkappap}{\later^{\!\!\!\kappa'}}
\newcommand{\laterkp}[1]{\later^{\!\!\!\kappa'}#1^{\kappa'}}
\newcommand{\forallkp}[1]{\forall\kappa'.#1^{\kappa'}}
\newcommand{\flaterkp}[1]{\forall\kappa'.\later^{\!\!\!\kappa'}#1^{\kappa'}}
\newcommand{\tensor}{\circledast}
\newcommand{\tensorkappan}{\ensuremath{\tensor^\kappa_n}}
\newcommand{\clockEnv}{\ensuremath{\Delta}}

\newcommand{\quine}[1]{\ensuremath{\ulcorner}#1\ensuremath{\urcorner}}
\newcommand{\eps}[6]{\ensuremath{#1\,\vdash\,#2\,:\,#3\,\xRightarrow{#6}\,#4\,\vdash\,#5\,:\,#6}}
\newcommand{\causal}{\ensuremath{\curvearrowleft}}
\newcommand{\noncausal}{\ensuremath{\leftrightarrows}}
\newcommand{\open}{\ensuremath{\sqcup}}
\newcommand{\closed}{\ensuremath{\sqcap}}
\newcommand{\infer}{\ensuremath{\Longrightarrow}}
\newcommand{\IEopen}{\ensuremath{\iota;\Psi;\phi;\open;\Pi;\Gamma}}
\newcommand{\IEclosed}{\ensuremath{\iota;\Psi;\phi;\closed;\Pi;\Gamma}}
\newcommand{\IEc}{\ensuremath{\iota;\Psi;\phi;\clockEnv;\Pi;\Gamma}}
\newcommand{\IEopencausal}{\ensuremath{\iota;\causal;\phi;\open;\Pi;\Gamma}}
\newcommand{\IEclosedcausal}{\ensuremath{\iota;\causal;\phi;\closed;\Pi;\Gamma}}
\newcommand{\IEopennoncausal}{\ensuremath{\iota;\noncausal;\phi;\open;\Pi;\Gamma}}
\newcommand{\IEclosednoncausal}{\ensuremath{\iota;\noncausal;\phi;\closed;\Pi;\Gamma}}
\newcommand{\phieq}{\ensuremath{\overset{\phi}{=}}}

\newcommand{\CEc}{\ensuremath{\iota; \Psi; \clockEnv; \Gamma}}
\newcommand{\CEclosed}{\ensuremath{\iota; \Psi; \closed; \Gamma}}
\newcommand{\CEclosedcausal}{\ensuremath{\iota; \causal; \closed; \Gamma}}
\newcommand{\CEclosednoncausal}{\ensuremath{\iota; \noncausal; \closed; \Gamma}}
\newcommand{\CEopen}{\ensuremath{\iota; \Psi; \open; \Gamma}}
\newcommand{\CEopencausal}{\ensuremath{\iota; \causal; \open; \Gamma}}
\newcommand{\CEopennoncausal}{\ensuremath{\iota; \noncausal; \open; \Gamma}}

%\lstset{basicstyle=\ttfamily}
\DeclareCaptionFormat{listing}{ #1#2#3}
\captionsetup[lstlisting]{format=listing,singlelinecheck=false, margin=0pt,  font={sf}}
%\DeclareCaptionFormat{listing}{\vspace{-0.5em}\hrulefill \par #1#2#3}
%\captionsetup[lstlisting]{format=listing,singlelinecheck=false, margin=0pt, font={sf},labelsep=space,labelfont=bf}

\newcommand{\idrisBlock}{\scriptsize{\texttt{Idris}}}
\newcommand{\ttBlock}{\scriptsize{\texttt{TT}}}

% Theorem Styles
\newtheorem{theorem}{Theorem}[section]
\newtheorem{lemma}[theorem]{Lemma}
\newtheorem{proposition}[theorem]{Proposition}
\newtheorem{corollary}[theorem]{Corollary}
% Definition Styles
\theoremstyle{definition}
\newtheorem{definition}{Definition}[section]
\newtheorem{example}{Example}[section]
\theoremstyle{remark}
\newtheorem{remark}{Remark}

\begin{document}

\frontmatter

\thetitlepage
\newpage

\chapter*{Abstract}
This is an abstract

\cleardoublepage
\setcounter{tocdepth}{1}
\tableofcontents

\mainmatter
\midsloppy
\sloppybottom

%!TEX root = ../copatterns-thesis.tex
\chapter{Introduction}

%!TEX root = ../copatterns-thesis.tex
\chapter{Background}

%!TEX root = ../../copatterns-thesis.tex
% Thomas
%###########
% Hvad er guarded recursion?
% Hvorfor guarded recursion?
% Klassisk guarded recursion vs. Atkey-McBride (clock variables)
% Guarded recursion og dependent types (topos of trees)
% Eksempler

% Ny viden: Hvad er guarded recursion?, Hvorfor guarded recursion?, Clock variables
%###########
\section{Guarded Recursion}
\label{sec:guarded-recursion}

Guarded recursion provides a typing discipline which ensures that all programs adhering to their type specification must be productive. In a system with a syntactic guardedness condition, the productivity of a given program follows from its syntactic structure. With guarded recursion, the productivity of a program follows from its type. Consequently, it is impossible to write a well-typed guarded recursive program which is not productive.

\subsection{The Guardedness Type Constructor}
The use of guarded recursion was pioneered by Nakano\,\citep{Nakano:2000}, who presented a type system (based on $\lambda \mu$) where no well-typed terms diverge by introducing a guardedness type constructor for a type $A$, $\later A$, into the type system. Although Nakano does not call it so explicitly, we recommend that the guardedness type constructor $\later$ is read as ``later''. The addition of the $\later$ operator leads to a type system where there is a distinction between an inhabitant of a type $A$, which must be available now (i.e. at any point in time), and an inhabitant of $\later A$, which is available later (i.e. not now). Using this intuition, a stream of elements of type $A$ can be defined as described in Figure\,\ref{fig:guarded_recursion_stream}. The idea is that we can access the head of the stream now, but we cannot access the recursive reference in the tail until later.

\begin{figure}
\[
Stream\,A = \mu X. A \times \later X
\]
\[
\frac{\Gamma\vdash x : A \quad \quad \Gamma\vdash s : \later Stream\,A}{\Gamma\vdash StreamCons(x,s) : Stream\,A} StreamCons
\]
\[
\frac{\Gamma\vdash s : Stream\,A}{\Gamma\vdash hd(s) : A} hd
\]
\[
\frac{\Gamma\vdash s : Stream\,A}{\Gamma\vdash tl(s) : \later Stream\,A} tl
\]
\caption{A guarded recursive definition of a stream type, along with its introduction and elimination rules.}
\label{fig:guarded_recursion_stream}
\end{figure} 

%At this point, we can draw a parallel to syntactic guardedness. Since any manipulation of recursive references is disallowed, syntactic guardedness essentially requires that recursive references are also available now, because there is no guarantee that a result can be given at any later point in time. In contrast, any guarded recursive definition must guarantee that even though the recursive reference is manipulated, a result will always be available later.

\subsection{Constructing Guarded Recursive Programs}
\label{sec:constr-guard-recurs}
To obtain the guarantees of productivity provided by guarded recursion, guarded
recursive programs must be constructed in a special way. For this purpose,
Nakano defines a guarded fixed point operator, shown in
Figure\,\ref{fig:guarded_recursion_fixpoint}. As the recursive reference of the
fixed point is only available later, we cannot inadvertently construct a well-typed program where the recursive reference is used in a way that would make the program diverge.
\begin{figure}
\[
\frac { \Gamma \vdash f : \later A \rightarrow A }{ \Gamma \vdash fix(f) : A } fix
\]
\caption{Fixpoint rule for guarded recursive programs.}
\label{fig:guarded_recursion_fixpoint}
\end{figure} 
As an illustration, consider the example of an infinite stream of zeros given in Figure\,\ref{fig:guarded_recursion_zeros}. The function \texttt{zeros} is only well-typed because the recursive reference has type $\later Stream\,Nat$, since a $Stream\,Nat$ could not have been given as an argument to \texttt{StreamCons}. Productivity is thus ensured because the type system forces us to preserve the levels of guardedness required by the type of the program.
\begin{figure}
\begin{alltt}
zeros : Stream Nat
zeros = fix(\(\lambda\)z. StreamCons 0 z)
\end{alltt}
\caption{A guarded recursive definition of an infinite stream of zeros.}
\label{fig:guarded_recursion_zeros}
\end{figure}

\begin{figure}
\[
\frac { \Gamma \vdash f: \later (A\rightarrow B)\quad \quad \Gamma \vdash e : \later A }{ \Gamma \vdash f \tensor e : \later B } { \tensor }_{ I }
\]
\[
\frac { \Gamma \vdash e:A }{ \Gamma \vdash Next(e):{ \later A } } Next
\]
\caption{Later application ($\tensor$) and $next$ introduction rules for guarded recursive programs.}
\label{fig:guarded_recursion_rules}
\end{figure}
To define guarded recursive programs with a more interesting structure than
\texttt{zeros}, we can apply the $\tensor$ operator shown in
Figure\,\ref{fig:guarded_recursion_rules} defined by Atkey and
McBride\,\citep{Atkey:2013}. We choose to read this operator as ``later
application''.  The later application operator facilitates the composition of
guarded recursive programs by allowing us to apply functions which are available
later to values which are available later. As hinted at by its type, later
application is also an applicative functor in the sense introduced by McBride
and Paterson\,\citep{Mcbride:2008}, and the standard laws for these apply here
as well. Also shown in Figure~\ref{fig:guarded_recursion_rules} is the $Next$
rule, which lifts a value available now into a into a later context. Since
$Next$ can be applied multiple times, any value available now can be made
available at any later point in time.

% \begin{figure}
% \begin{lstlisting}[mathescape]
% map : (a $\rightarrow$ b) $\rightarrow$ Stream a $\rightarrow$ Stream b
% map = fix($\lambda$m.$\lambda$f.$\lambda$s. StreamCons$\,$(f (hd s)) 
%                               $\;$(m $\tensor$ (next f) $\tensor$ (tl s)))
% nats : Stream Nat
% nats = fix($\lambda$n. StreamCons 0 ((next map $\tensor$ next S) $\tensor$ n))
% \end{lstlisting}
% \caption{A guarded recursive definition of a function \texttt{map} for streams.}
% \label{fig:guarded_recursion_map}
% \end{figure}

\begin{figure}
\begin{lstlisting}[mathescape]
map : (a $\rightarrow$ b) $\rightarrow$ Stream a $\rightarrow$ Stream b
map = fix($\lambda$m.$\lambda$f.$\lambda$s. StreamCons$\,$(f (hd s)) 
                              $\;$((m $\tensor$ (Next f)) $\tensor$ (tl s)))

nats : Stream Nat
nats = fix($\lambda$n. StreamCons Z (((Next map) $\tensor$ (Next S)) $\tensor$ n))
\end{lstlisting}
%\begin{alltt}
nats : Stream Nat
nats = fix(\(\lambda\)n. StreamCons 0 ((next map \(\circledast\) next S) \(\circledast\) n))
\end{alltt}

\caption{A guarded recursive definition of the natural numbers, using the \texttt{map} function. The function \texttt{S} is the successor function for the natural numbers, which has the standard definition without any added guardedness information.}
\label{fig:guarded_recursion_nats}
\end{figure}

Both the later application rule and the $Next$ rule are used in the example of
\texttt{nats} in Figure~\ref{fig:guarded_recursion_nats}. In order to obtain a
value of type $\later Stream\,Nat$ for the second argument of
\texttt{StreamCons}, we proceed as follows:

\begin{enumerate}
\item We apply $Next$ to \texttt{map} and \texttt{S}, obtaining
  \texttt{Next~map} of type ${\later((a~\rightarrow~b)~\rightarrow~Stream~a~\rightarrow~Stream~b)}$
    and \texttt{Next~S} of type ${\later(Nat~\rightarrow~Nat)}$
\item Applying the later application operator, $\tensor$, we get
  (\texttt{Next~map} $\tensor$ \texttt{Next~S}) of type 
    ${\later(Stream\,Nat \rightarrow Stream\,Nat)}$
  
\item Applying this to the recursive reference, \texttt{n}, leaves us with an element of ${\later Stream\,Nat}$, as desired.
\end{enumerate}

 % first apply $Next$ to \texttt{map} and \texttt{S}, obtaining \texttt{next map
 %   :} $\later((a~\rightarrow~b)~\rightarrow~Stream~a~\rightarrow~Stream~b)$ and
 % \texttt{next~S~:}~$\later(Nat~\rightarrow~Nat)$. Applying the later application
 % operator, $\tensor$, we get (\texttt{Next map} $\tensor$ \texttt{Next S}) : $\later(Stream\,Nat \rightarrow Stream\,Nat)$, which when applied to the recursive reference \texttt{n} leaves us with an element of $\later Stream\,Nat$, as desired.

\subsection{Clock Variables}
While the system of guarded recursion presented so far is useful for manipulating values that we know will be available later, it is still impossible to leave the time constraints behind (go from $\later A$ to $A$), even when it is safe to do so. This puts some limitations on the practical use of the system. To alleviate this situation, Atkey and McBride\,\citep{Atkey:2013} introduce the idea of clock variables. A clock variable $\kappa$ represents a clock with a fixed amount of time remaining. Clocks can be associated with types, such that when $x : \onk{A}$, we know that at least the first $\kappa$ elements of $x$ can be provided. This intuition is then extended with universal quantification over clocks, such that when $x : \forall\kappa. A^\kappa$ we know that $x$ is productive, since for any $\kappa$ the first $\kappa$ elements of $x$ can be provided. 
\begin{figure}
\centering
\textbf{Wellformed Types} 

\vspace{1em}

\AxiomC{$\Delta, \kappa ; \Gamma \vdash A : Type$}
\AxiomC{$\kappa \notin fc(\Gamma)$}
\BinaryInfC{$\Delta ; \Gamma \vdash \forall \kappa . A : Type$}
\DisplayProof
\quad
\AxiomC{$\Delta ; \Gamma \vdash A : Type$}
\AxiomC{$\kappa \in \Delta$}
\BinaryInfC{$\Delta ; \Gamma \vdash \laterkappa A : Type$}
\DisplayProof

\vspace{1em}

\textbf{Typing Rules} 

\vspace{1em}

\AxiomC{$\Delta ; \Gamma \vdash t : A$}
\UnaryInfC{$\Delta ; \Gamma \vdash \onk{next}\ t : \laterkappa A$}
\DisplayProof
\quad
\AxiomC{$\Delta ; \Gamma \vdash t : \laterkappa (A \to B)$}
\AxiomC{$\Delta ; \Gamma \vdash u : \laterkappa A$}
\BinaryInfC{$\Delta ; \Gamma \vdash t \tensor^\kappa u : \laterkappa B$}
\DisplayProof

\vspace{1em}

\AxiomC{$\Delta ; \Gamma , x : \laterkappa A \vdash t : A$}
\UnaryInfC{$\Delta ; \Gamma \vdash \onk{fix}x.t : A$}
\DisplayProof
\quad
\AxiomC{$\Delta, \kappa ; \Gamma \vdash t : A$}
\AxiomC{$\kappa \notin fc(\Gamma)$}
\BinaryInfC{$\Delta ; \Gamma \vdash \Lambda \kappa. t : \forall \kappa. A$}
\DisplayProof

\vspace{1em}

\AxiomC{$\kappa \notin fc(\Gamma)$}
\AxiomC{$\Delta, \kappa ; \Gamma \vdash A : Type$}
\AxiomC{$\Delta, \kappa' ; \Gamma, \Gamma' \vdash t : \forall \kappa. A$}
\TrinaryInfC{$\Delta, \kappa' ; \Gamma , \Gamma ' \vdash t[\kappa'] : A[{  \kappa'}/{\kappa  }]$}
\DisplayProof

%%% Local Variables:
%%% mode: latex
%%% TeX-master: "../copatterns-thesis"
%%% End:

% \[
\frac { \Delta ;\Gamma \vdash f: \laterkappa(A\rightarrow B)\quad \quad \Delta;\Gamma \vdash e : \laterkappa A }{ \Delta;\Gamma \vdash f \tensor e : \laterkappa B } { \tensor }_{ I }
\]
%%% Local Variables: 
%%% mode: latex
%%% TeX-master: t
%%% End: 

% \[
\frac { \Delta ;\Gamma \vdash e:A\quad \quad \kappa \in \Delta  }{ \Delta ;\Gamma \vdash next(e):{ \laterkappa A } } next
\]
%%% Local Variables: 
%%% mode: latex
%%% TeX-master: t
%%% End: 

% \[
\frac { \Delta;\Gamma \vdash f : \laterkappa A \rightarrow A }{ \Delta;\Gamma \vdash fix(f): A } fix
\]
%%% Local Variables: 
%%% mode: latex
%%% TeX-master: t
%%% End: 

% \[
\frac { \Delta ;\Gamma \vdash e:\forall\kappa. A \quad \quad \kappa \notin fv(A) }{ \Delta ;\Gamma \vdash applyFresh(e):A } applyFresh
\]
%%% Local Variables: 
%%% mode: latex
%%% TeX-master: t
%%% End: 

% \[
\frac { \Delta ;\Gamma \vdash e:\forall\kappa.\laterkappa A }{ \Delta;\Gamma \vdash force(e) : \forall\kappa.A } force
\]
%%% Local Variables: 
%%% mode: latex
%%% TeX-master: t
%%% End: 

% \[
\frac { \Delta ,\kappa ;\Gamma \vdash e:A\quad \quad \kappa \notin fv(\Gamma)  }{ \Delta ;\Gamma \vdash \Lambda\kappa.e:\forall\kappa. A } { \forall }_{ I }
\]

%%% Local Variables: 
%%% mode: latex
%%% TeX-master: t
%%% End: 

% \[
\frac { \Delta ;\Gamma \vdash e:\forall\kappa. A \quad \quad \kappa' \in \Delta }{ \Delta;\Gamma\vdash e : A [\kappa \mapsto \kappa'] } { \forall }_{ E }
\]
%%% Local Variables: 
%%% mode: latex
%%% TeX-master: t
%%% End: 

% \[
% \frac { \Delta;\Gamma\vdash \later^{\!\!\!\kappa'}\forall\kappa.A \quad \quad \kappa \neq \kappa' } { \Delta;\Gamma\vdash \forall\kappa.\later^{\!\!\!\kappa'}A} switch
% \]
\caption{Rules for guarded recursive definitions with clocks, as defined by
  M\o gelberg\,\citep{Mogelberg:2014}.}
\label{fig:guarded_recursion_rules_clocks}
\end{figure}
\begin{figure}
\[
Stream\,A = \mu X. A \times \laterk{X}
\]
\[
\frac{\Delta;\Gamma\vdash x : A \quad \quad \Delta;\Gamma\vdash s : \laterk{Stream}\,A}{\Delta;\Gamma\vdash StreamCons(x,s) : \onk{Stream}\,A} StreamCons
\]
\[
\frac{\Delta;\Gamma\vdash s : \onk{Stream}\,A}{\Delta;\Gamma\vdash hd(s) : A} hd
\]
\[
\frac{\Delta;\Gamma\vdash s : \onk{Stream}\,A}{\Delta;\Gamma\vdash tl(s) : \laterk{Stream}\,A} tl
\]
\caption{The guarded recursive definition of streams from
  Figure\,\ref{fig:guarded_recursion_stream} extended with clocks.}
\label{fig:guarded_recursion_stream_clocks}
\end{figure} 
The introduction of clock variables and quantification over these requires some
changes to the rules presented above. In
Figure~\ref{fig:guarded_recursion_rules_clocks}, the extended rules, defined for
a dependently typed system by M\o gelberg\,\citep{Mogelberg:2014}, are
shown. Addtionally, a stream definition extended with clocks is shown in
Figure~\ref{fig:guarded_recursion_stream_clocks}. To exemplify the use of clock
variables, we have extended the example of \texttt{nats} from
Figure~\ref{fig:guarded_recursion_nats} with clock variables and clock
quantification in Figure~\ref{fig:guarded_recursion_nats_with_clocks}. Notably,
we must use clock abstraction ($\Lambda\kappa$) in order to obtain types
quantified over clocks. Furthermore, clock application
(e.g. \texttt{map[$\kappa$]}) ensures that \texttt{map} operates on the same
clock as \texttt{nats}. However, not all examples of interesting programs are as
simple as \texttt{nats}. If we wish to write programs such as
\texttt{evens} in Figure~\ref{fig:guarded_recursion_evens}, where the
input and output are individually quantified over a clock, we must define it using an
indexed fixed point. The use of the indexed fixed point ($\mathit{pfix}$ in
Figure~\ref{fig:pfix}) ensures that the arguments to the recursive reference are
not placed under a later ($\laterkappa$). Note that the indexed fixed point can
be defined in terms of the original fixed point (defined in
Figure~\ref{fig:guarded_recursion_rules_clocks}), so it can be added to a
guarded recursive system without introducing any additional axioms.
\begin{figure}
\begin{lstlisting}[mathescape]
map : $\forall\kappa.$ (a $\rightarrow$ b) $\rightarrow$ Stream a $\rightarrow$ Stream b
map = $\Lambda\kappa.$ fix$^\kappa$($\lambda$m.$\lambda$f.$\lambda$s. StreamCons$\,$(f (hd s)) 
                                   ((m $\tensor$ (Next f)) $\tensor$ (tl s)))

nats : $\forall\kappa.$ Stream Nat
nats = $\Lambda\kappa.$ fix$^\kappa$($\lambda$n. StreamCons Z (((Next (map[$\kappa$]) $\tensor$ (Next S)) $\tensor$ n))
\end{lstlisting}
%\begin{alltt}
nats : Stream Nat
nats = fix(\(\lambda\)n. StreamCons 0 ((next map \(\circledast\) next S) \(\circledast\) n))
\end{alltt}

\caption{A guarded recursive definition of the natural numbers using clock
  variables and clock quantification.}
\label{fig:guarded_recursion_nats_with_clocks}
\end{figure}

\begin{figure}
\begin{lstlisting}[mathescape]
evens$'$ : ($\forall\kappa.$ Stream$^\kappa$ a $\to$ $\laterkappa$(Stream$^\kappa$ a)) $\to$ ($\forall\kappa.$ Stream$^\kappa$ a) $\to$ Stream$^\kappa$ a
evens$'$ = $\lambda$rec.$\lambda$s. StreamCons (head s) (rec (tail (tail s)))

evens : ($\forall\kappa.$ Stream$^\kappa$ a) $\to$ ($\forall\kappa.$ Stream$^\kappa$ a)
evens = $\Lambda\kappa.$ pfix$^{\kappa}$(evens$'$)
\end{lstlisting}
\caption{A guarded recursive definition of a function that removes every second element from a stream.}
\label{fig:guarded_recursion_evens}
\end{figure}

\begin{figure}
\begin{prooftree}
\AXD{\Delta;\Gamma,x:A\to \laterkappa B\vdash t:A\to B}
\UID{\Delta;\Gamma\vdash \mathit{pfix}^{\kappa}x.t:A\to B}
\end{prooftree}
\begin{center}
$\mathit{pfix}^{\kappa}x.t = \mathit{fix}^{\kappa}\,y:\laterkappa(A\to
B).t[\lambda a.y\tensor^{\kappa}next^{\kappa}(a)/x]$
\end{center}
%%% Local Variables:
%%% mode: latex
%%% TeX-master: "../copatterns-thesis"
%%% End:

\caption{An indexed fixed point used to define functions where both the input
  and the output have types quantified over clocks.}
\label{fig:pfix}
\end{figure}


% The most important change is the addition of the $Force$ rule, which says that whenever it is possible to universally quantify over a clock on a type $\laterk{A}$, the time constraints imposed by the $\later$ operator can be removed. Universal quantification over clocks is introduced by adding a clock abstraction form $\Lambda\kappa$, as indicated by the rules $\forall_{I}$ and $\forall_{E}$.
% \begin{figure}
% \begin{lstlisting}[mathescape]
% evens : $\forall\kappa.$ Stream$^\kappa$ $\to$ $\forall\kappa.$ Stream$^\kappa$
% evens = fix^{\kappa}(evens$'$)
% \end{lstlisting}
% \caption{A guarded recursive definition of a function that removes every second element from a stream.}
% \label{fig:guarded_recursion_evens}
% \end{figure}
% The $force$ rule is necessary for the \texttt{evens} function in Figure~\ref{fig:guarded_recursion_evens} to be well-typed, since \texttt{StreamCons} requires as its second argument a $\laterk{Stream} A$. Seeing as we need to have two calls to \texttt{tl} on the input stream \texttt{s} to obtain the desired semantics, we have a situation in which \texttt{(next tl $\tensor$ (tl s[$\kappa'$])) :} $\laterkappap\laterkp{Stream} A$, i.e. the result is available too late. We can recover from this situation by forcing the result to be available earlier. Forcing is only possible because the type of \texttt{evens} requires the input stream to be universally quantified over all clocks, so we could not have defined this function in a guarded recursive type system without clock variables. Since the $force$ rule only works on values that are universally quantified over all clocks, we abstract over $\kappa'$ in the expression, getting \texttt{($\Lambda\kappa'$.(next tl $\tensor$ (tl s[$\kappa'$]))) :} $\forall\kappa'.\laterkappap\laterkp{Stream} A$. The universal quantification can now be handled with $force$, although according to the type of \texttt{evens} this leaves us with the clock quanfication of the wrong side of the $\laterkappap$ operator. Therefore the $switch$ rule must be used to get the desired type. For the first argument to \texttt{StreamCons} in \texttt{evens}, the universal quantification can be eliminated by the $applyFresh$ rule since no $\laterkappap$ constructors are involved.
% \begin{figure}
% \begin{lstlisting}[mathescape]
% take : Nat $\rightarrow$ ($\forall\kappa.$Stream$^\kappa$ a) $\rightarrow$ List a
% take Z     s = []
% take (S n) s = applyFresh($\Lambda\kappa$.(hd(s[$\kappa$])))
%                :: take n (force($\Lambda\kappa$.(tl(s[$\kappa$]))))
% \end{lstlisting}
% \caption{A function that takes the first \texttt{n} elements of the input stream \texttt{s}. The definition of the \texttt{List} type is standard, without any $\later$ operations or clocks.}
% \label{fig:guarded_recursion_take}
% \end{figure}

% A different example of the use of clock variables is shown in Figure~\ref{fig:guarded_recursion_take}. The \texttt{take} function is not defined in terms of the $fix$, since elements of its result type, \texttt{List}, should always be available. To be able to make the recursive call well-typed, however, we need to rid ourselves of the $\laterkappa$ constraint imposed on the stream argument \texttt{s} by the use of \texttt{tl}. The solution is to use the $force$ rule, which requires that \texttt{s} is known to be productive. 

% Note that the examples in Figure~\ref{fig:guarded_recursion_zeros} and Figure~\ref{fig:guarded_recursion_nats} are all still valid in a setting with clock variables. Their types simply have to be annotated with a top-level clock quantification, as shown in Figure~\ref{fig:guarded_recursion_quantified_examples}.

% \begin{figure}
% \begin{lstlisting}[mathescape]
% zeros : $\forall\kappa$.Stream$^\kappa$$\,$Nat
% map   : $\forall\kappa$.(A $\rightarrow$ B) $\rightarrow$ Stream$^\kappa$$\,$A $\rightarrow$ Stream$^\kappa$$\,$A
% nats  : $\forall\kappa$.Stream$^\kappa$$\,$Nat
% \end{lstlisting}
% \caption{The types of the examples from Section~\ref{sec:constr-guard-recurs} extended with clock variables.}
% \label{fig:guarded_recursion_quantified_examples}
% \end{figure}
%\todo{Introduce indexed fixed point}
%\todo{Explanation of topos of trees goes here.}

% \begin{landscape}
% \begin{figure}
% \[
% \cfrac { \cfrac { \cfrac { \cfrac {  }{ \Gamma ,\, n\, \vdash \, 0\, :\, \mathbb{N} } \, \, \cfrac { \cfrac { \cfrac { \cfrac { \cfrac {  }{ \Gamma ,\, n\, \vdash \, map\, :\, (\mathbb{N}\, \rightarrow \, \mathbb{N})\, \rightarrow \, Stream\, \mathbb{N}\, \rightarrow \, Stream\, \mathbb{N} } \, \, \cfrac {  }{ \Gamma ,\, n\, \vdash \, S\, :\, \mathbb{N}\, \rightarrow \, \mathbb{N} }  }{ \Gamma ,\, n\, \vdash \, map\, S\, :\, Stream\, \mathbb{N}\, \rightarrow \, Stream\, \mathbb{N} }  }{ \Gamma ,\, n\, \vdash \, next\, (map\, S)\, :\, \rhd \, (Stream\, \mathbb{N}\, \rightarrow \, Stream\, \mathbb{N}) } \, \, \cfrac {  }{ \Gamma ,\, n\, \vdash \, n\, :\, \rhd \, Stream\, \mathbb{N} }  }{ \Gamma ,\, n\, \vdash \, next\, (map\, S)\, \circledast \, n\, :\, \rhd \, Stream\, \mathbb{N}\, \,  }  }{ \Gamma ,\, n\, \vdash \, map\, S\, n\, :\, \rhd \, Stream\, \mathbb{N} }  }{ \Gamma ,\, n\, :\, \rhd \, Stream\, \mathbb{N}\, \vdash \, StreamCons\, 0\, (map\, S\, n)\, :\, Stream\, \mathbb{N} }  }{ \Gamma \, \vdash \, fix\, (\lambda n.\, StreamCons\, 0\, (map\, S\, n))\, :\, Stream\, \mathbb{N}\,  }  }{ \Gamma \, \vdash \, 0\, ::\, map\, S\, nats\, :\, Stream\, \mathbb{N} } 
% \]
% \caption{An inference tree for a program \texttt{nats} defining the natural numbers.}
% \end{figure}
% \end{landscape}

% \begin{figure}[h]
%   Møgelberg rules here!
%   \caption{The rules for guarded recursion in a dependently typed system.}
%   \label{fig:guarded_recursion_dependent_rules}
% \end{figure}

%\paragraph{}
%Note that guardedness type constructors can be nested, meaning that $\later A$ and $\later \later A$ are both valid and distinct types. However, an element of type $\later A$ can never be given where an element of type $\later \later A$ is expected, and vice versa, since guardedness levels cannot be collapsed.

%%% Local Variables: 
%%% mode: latex
%%% TeX-master: "../../copatterns-thesis"
%%% End: 


%%% Local Variables: 
%%% mode: latex
%%% TeX-master: "../copatterns-thesis"
%%% End: 



\chapter{Motivation}
In this chapter, we describe the motivations behind adding copatterns to
Idris. We find that copatterns can be a means of recovering from an unfortunate
loss of subject reduction. However, copatterns might not be the universal tool
for coinductive reasoning. Additionally, we motivate the addition of an
inference system for guarded recursion.

\section{Recovering Subject Reduction}
\label{sec:recov-subj-reduct}
%#########
% Hvorfor mistes subject reduction ved dependent pattern matching på codata?
% Mister Idris subject reduction?
% Modeksempel (helst implementeret i Idris)
% Hvordan afhjælper copatterns problemet?

% Ny viden: Copatterns kan være et værktøj til at undgå at man mister subject reduction
%#########
Subject reduction is a property of a type system. Also called type preservation,
a type system has the property of subject reduction if evaluation of a
well-typed term does not cause its type to change\,\citep[Section~8.3.3]{Pierce:2002:TPL:509043}.
In some dependently typed programming languages with explicit coinductive types,
e.g. Coq and Idris, values of such types are given as (potentially) infinite
trees of constructors. Accordingly, they can be analysed with (dependent)
pattern matching. However, this leads to a loss of subject reduction. The
problem was first identified by Gim\'{e}nez\,\citep{Gimenez96uncalcul} for the
Calculus of (Co)Inductive constructions, but was later (in 2008) shown by Oury
to persist in both Coq and
Agda\,\citep{OuryCounterexampleCoq,OuryCounterexampleAgda}.

By implementing Oury's counterexample
(Figure~\ref{fig:ourys_counterexample_idris}), we have found that the problem
arises in Idris as well. Oury's counterexample unfolds as follows: Given a
coinductive type \texttt{A} with a single constructor \texttt{A} $\to$ \texttt{A}, which we may call
\texttt{In}, and an inhabitant \texttt{b} of \texttt{A} (of which there exists only one,
namely the fixed point of \texttt{In}), we can show that \texttt{b} is definitionally
equal to its own unfolding by exploiting dependent pattern matching. When
\texttt{b} is given as argument to \texttt{forceEq} in
Figure~\ref{fig:ourys_counterexample_idris}, \texttt{b} is reduced to \texttt{In
  (Delay (In (Delay a)))} by dependent pattern matching, such that \texttt{forceEq b}
reduces to \texttt{Refl}. These reductions ensure that the implementation of
\texttt{p} is well-typed. The problem arises when we attempt to replace the
right-hand side of \texttt{p} with \texttt{Refl}, which is the definition of
\texttt{forceEq b}. In this case, the Idris type checker rejects the program,
complaining that \texttt{b} cannot be unified with \texttt{In b}. Thus, the
evaluation of \texttt{forceEq b} changes the type of \texttt{Refl}, and subject
reduction is lost.

\begin{figure}
\begin{lstlisting}[mathescape]
codata A : Type where
  In : A $\to$ A

b : A
b = In b

force : A $\to$ A
force (In (Delay (In (Delay a)))) = In (Delay (In (Delay a)))

forceEq : (x : A) $\to$ x = force x
forceEq (In (Delay (In (Delay a)))) = Refl

p : b = In (In b)
p = forceEq b
\end{lstlisting}
  \caption{Oury's counterexample implemented in Idris, leading to a loss of
    subject reduction.}
\label{fig:ourys_counterexample_idris}
\end{figure}

Dependent pattern matching on coinductive data is problematic, but we cannot
simply disallow it, since we then lose the ability to write any interesting
programs involving coinductive data. This ability can be regained by introducing
copatterns. Instead of analysing coinductive data with dependent pattern
matching, we can choose to synthesize it using copatterns. In
Figure~\ref{fig:ourys_counterexample_copatterns}, we have translated Oury's
counterexample to an Idris implementation with copatterns, where we have
refrained from using dependent pattern matching. Using copatterns, we can still
define both \texttt{b} and \texttt{force}, but we cannot provide an
implementation for \texttt{p}, since the reduction behaviour from dependent
pattern matching is (virtually) unavailable. Hence, we have both regained
expressiveness and avoided this particular loss of subject reduction.

\begin{figure}[h]
\begin{lstlisting}[mathescape]
corecord A : Type where
  out : A $\to$ A

b : A
&out b = b

force : A $\to$ A
&out force a = out a
&out &out force a = out (out a)

forceEq : (x : A) $\to$ x = force x
forceEq a = $\uwave{\text{Refl}}$

p : b = out (out b)
p = $\uwave{\text{forceEq b}}$
\end{lstlisting}
  \caption{By using copatterns to synthesize coinductive data, we can both
    implement functions on coinductive data and preserve subject
    reduction. The underlined right-hand sides indicate type errors detected by
    the Idris type checker.}
\label{fig:ourys_counterexample_copatterns}
\end{figure}

In Agda, pattern matching on coinductive data is currently
disallowed\,\citep{Danielsson09}, providing the user with copatterns instead. By
implementing copatterns in Idris, we provide a stepping stone in the same
direction. However, this solution may seem to avoid the problem, rather than
solving it. As pointed out by McBride\,\citep{McBride:2009}, the underlying
problem is that we can have intensional equalities between coinductive values
which are merely bisimilar. To remedy the situation, he proposes that one could
have an observational propositional equality, which takes into account both
functional extensionality and bisimilarity for coinductive values. Also,
Altenkirch et al.\,\citep{Altenkirch:2007} proposes the adoption of an
observational type theory, which is essentially an intensional type theory where
one can have propsitional equalities up to observation, as opposed to
construction. But until these ideas are incorporated into a practical system,
avoiding the problem seems superior to ignoring it.


\section{The Use Case for Copatterns}
\label{sec:motivation_copatterns}
%############
% Pattern matching on data with top-level product structure vs. copatterns
% Mixed induction-coinduction, coinductive resumption monad
% Coinduction (definition og operationel intuition) (måske bisimilarity)

% Ny viden: Hvornår bør/kan jeg bruge copatterns?
%############
% Pattern matching has become a ubiquitous tool for analysing data in functional
% programming languages. In combination with inductive data types, pattern
% matching has evolved into such a widely applicable technique that many users may

% be willing to disregard its drawbacks. Consider the following implementation of
% a function which interleaves a filtered version of \texttt{xs} with \texttt{ys}:
% \begin{lstlisting}[mathescape]
% interleaveFilter : (a $\to$ Bool) $\to$ Stream a $\to$ Stream a $\to$ Stream a
% interleaveFilter p (x :: xs) (y :: ys) = 
% if p (head xs) then case xs of 
%                      xx :: xxs 
%    case (filter p xs) of
%      px :: _ => (px :: y :: (interleaveFilter p xs ys)
% \end{lstlisting}
% \begin{lstlisting}[mathescape]
% interleaveFilter : (a $\to$ Bool) $\to$ Stream a $\to$ Stream a $\to$ Stream a
% head       (interleaveFilter p xs ys)  = head (filter p xs)
% head (tail (interleaveFilter p xs ys)) = head ys
% tail (tail (interleaveFilter p xs ys)) = 
%                                interleaveFilter p (tail xs) (tail ys)

% mapNth : (a -> b) -> Nat -> Nat -> Stream a -> Stream b
% mapNth f (S n') m s = case s of
%                        x :: xs => x :: mapNth f n' xs
% mapNth f Z m s = case s of
%                   x :: xs => f x :: mapNth f m m xs

% \end{lstlisting}

%\subsection{Copatterns are Useful for Record Types}
% Recursive types!
\label{sec:prod-vs.-copr}
If we must disallow pattern matching on coinductive data to preserve subject
reduction, how much expressiveness can we recover by using copatterns? At
present, not as far as one might have hoped. Copatterns are well-suited for
working with data that has a top-level product structure. In the general sense
of recursive types, this means that copatterns can be readily used for types
which has a product-of-sums structure, i.e.
${\mu X.\,A\times B\times\cdots\times X}$ or
${\nu X.\,A\times B\times\cdots\times X}$, since a well-defined projection
exists for each $A,\,B,\,\cdots,\,X$. From the definition of coinductive types
(Definition~\ref{sec:coinductive-types}), these are the types that can be
defined from an endofunctor with a top-level product structure, allowing the
structure map for the coalgebra to be defined as a product of projections. In
Idris, this amounts exactly to the types which can be defined as (co)inductive
record types. Consider the deterministic finite automaton defined in
Figure~\ref{fig:state_machine}, which accepts strings of binary numbers that
contain an even number of zeros (a graphical representation is given for
reference in Figure~\ref{fig:state_machine_graphical}). Inspired by
Jacobs\,\citep{JacobsCoalgebra}, we define a state as a type
${\mu S. (A\,\to\,S)\times Bool}$, where $A\,\to\,S$ is the transition function
(defined for each state) and $Bool$ indicates whether the state is a final
(i.e. accepting) state. Consequently, we can define each state (\texttt{SA} and
\texttt{SB} in Figure~\ref{fig:state_machine}) quite elegantly using copatterns.

\begin{figure}[h]
\begin{lstlisting}[mathescape]
data Binary = Zero | One

record State a where
  transitions : a $\to$ State a 
  isFinal     : Bool

mutual  
  SA : State Binary
  &transitions SA = ts
   where
     ts : Binary $\to$ State Binary
     ts Zero = SB
     ts One  = SA
  &isFinal     SA = True 
  
  SB : State Binary
  &transitions SB = ts
   where
     ts : Binary $\to$ State Binary
     ts Zero = SA 
     ts One  = SB
  &isFinal     SB = False

isAccepted : List Binary $\to$ Bool
isAccepted = isAccepted$'$ SA
  where
   isAccepted$'$ : State Binary $\to$ List Binary $\to$ Bool
   isAccepted$'$ s []        = isFinal s
   isAccepted$'$ s (b :: bs) = isAccepted$'$ ((transitions s) b) bs
\end{lstlisting}
  \caption{A finite state machine implemented using copatterns, identifying
    binary strings which contain an even amount of zeros.}
\label{fig:state_machine}
\end{figure}

\begin{figure}[h]
\centering
\includegraphics{figures/dfa}
\caption{A graphical representation of the automation defined in
  Figure~\ref{fig:state_machine}.}
\label{fig:state_machine_graphical}
\end{figure}

Unfortunately, copatterns are not very well-suited for types that have a top-level
sum-of-products structure, i.e. ${\mu X.\,A + B + \cdots + X}$ and
${\nu X.\,A + B + \cdots + X}$, because values of such types can take on
different forms in different contexts. In other words, the observable behaviour
depends on a further analysis, and the projections must therefore make
such an analysis possible. Consider the possibly infinite list encoded in
Figure~\ref{fig:colist}. Here, it is not possible to have a projection that
accesses the first element of a \texttt{CoList} directly, since it may be
empty. Instead, we can access an unfolding of the list, which may give either
answer.

\begin{figure}[h]
\begin{lstlisting}[mathescape]
corecord CoList a where
  unfold : Either (a, CoList a) ()

map : (a $\to$ b) $\to$ CoList a $\to$ CoList b 
&unfold map f xs = case (unfold xs) of
                      Left (x, xs) => Left (f x, map f xs)
                      Right () => Right ()
\end{lstlisting}
  \caption{A possibly infinite list type and a \texttt{map} function operating on it.}
  \label{fig:colist}
\end{figure}

The inherent uncertainty or ambiguity in types with sums-of-product structure
cannot be elegantly handled with copatterns. We can try to eliminate the need
for further analysis by tagging all values with an indicator of its state, as
shown with the modified CoList in Figure~\ref{fig:dependent_colist}. However, as
soon as the value of the tag is given indirectly, e.g. by reference to other values,
the type checker may become unable to reduce the tag to a canonical form, thereby
defeating its purpose.

\begin{figure}[h]
\begin{lstlisting}[mathescape]
corecord CoList a where
  isNil : Bool
  elem  : if isNil then () else (a, CoList a)

nil : CoList a
&isNil nil = True
&elem  nil = ()

cons : a $\to$ CoList a $\to$ CoList a
&isNil cons x xs = False
&elem  cons x xs = (x, xs)

toggle : Nat $\to$ Nat $\to$ CoList Nat
&isNil toggle n m = False
&elem toggle n m = (n, (cons m (toggle m n)))
\end{lstlisting}
\caption{A version of CoList defined with a boolean tag indicating whether
  additional elements of the list are available.}
\label{fig:dependent_colist}
\end{figure}

Since many useful techniques involve coinductive types which do not have a
straightforward product-of-sums structure, e.g. coinductive
resumptions\,\citep{Pirog2014273} and mixed
induction-coinduction\,\citep{Danielsson09mixinginduction}, \texttt{co}patterns
are therefore not necessarily a valuable tool for defining all kinds of
\texttt{co}inductive data. Instead, they provide elegant definitions when your data can be
clearly defined by its external properties, rather than by its internal
structure. Hence, we find copatterns at this stage to be universally \emph{applicable}, but not
universally \emph{useful}.


\section{Less Restrictive Productivity Checking} 
\label{sec:less-restr-prod}
% Hvorfor vil vi inferere guarded recursion?
%###########
% Dette afsnit har i princippet intet at gøre med copatterns!
% Referer til baggrund om syntactic guardedness
% Hvad kan guarded recursion som syntactic guardedness?
% Hvorfor er guarded recursion frygteligt at skrive / ikke brugervenligt?
% Bedre end syntactic guardedness + Bruger vil ikke skrive det => Inferer det


% Ny viden: Hvorfor vil vi gerne inferere guarded recursion?
%###########
Because the current productivity checker in Idris uses the syntactic
guardedness principle to prove productivity, users writing programs on
coinductive types must place all recursive calls directly under a call to a
coinductive constructor. This limitation means that we cannot have simple
definitions such as \texttt{toggle} from Figure~\ref{fig:dependent_colist}
proven total by the compiler. Using the technique of guarded recursion presented
in Section~\ref{sec:guarded-recursion} we are able to prove that toggle is
total, as shown in Figure~\ref{fig:toggle_guarded_recursion}. The proof is quite
involved, as it also requires us to define guarded recursive versions of
\texttt{cons} and \texttt{CoList} on which the original definition of
\texttt{toggle} depends. The coinductive \texttt{CoList} must be amended such
that the recursive reference in the \texttt{elem} projection is not be
immediately available.

\begin{figure}[h]
\begin{lstlisting}[mathescape]
corecord $_g$CoList a where
  isNil : Bool
  elem  : if isNil then () else (a, $\laterkappa$(CoList a))
  constructor CoCons

cons : $\forall\kappa.$ a $\to$ $_g$CoList a $\to$ $_g$CoList a
cons = $\Lambda\kappa$. fix$^\kappa$($\lambda$rec.$\lambda$x.$\lambda$xs. CoCons$\ $False (x, (Next xs)))

toggle$'$ : $\laterkappa$(Nat $\to$ Nat $\to$ Stream) $\to$ Nat $\to$ Nat $\to$ $_g$CoList Nat
toggle$'$ rec n m = CoCons False (n, (cons m ((rec $\tensor$ m) $\tensor$ n))) 

toggle : $\forall\kappa.$ Nat $\to$ Nat $\to$ $_g$CoList Nat
toggle = $\Lambda\kappa.$ fix$^\kappa$($\lambda$rec.$\lambda$n.$\lambda$m. toggle$'$ rec n m)
\end{lstlisting}
  \caption{An implementation of \texttt{toggle} from
    Figure~\ref{fig:dependent_colist} using guarded recursion, given in
    Idris-like syntax.}
\label{fig:toggle_guarded_recursion}
\end{figure}

Although guarded recursion widens the range of programs we can prove total as
compared to syntactic guardedness, requiring the user to write guarded recursive
programs is probably not feasible due to their complexity. Even disregarding
this complexity, building such programs quickly becomes an onerous
task. Consequently, we would like to have a system which can automatically build
guarded recursive versions of at least some user-written programs, lifting the burden of
building productivity proofs by guarded recursion from the user and onto the
compiler. The structure and implementation of such a system will be the subject
of Chapter~\ref{cha:infer-guard-recurs}.

%%% Local Variables:
%%% mode: latex
%%% TeX-master: "../copatterns-thesis"
%%% End:


%!TEX root = ../copatterns-thesis.tex
\chapter{Idris}
Idris is a general-purpose functional programming language with full dependent
types. To aid the presentation of our solution in the following chapters, this
chapter discusses the internal structure of the language implementation. Many of
the details of the implementation which are not covered here have been described
thoroughly by Brady\,\citep{BradyIdrisImpl13}.

% To understand how an implementation of guarded recursion could be realized in Idris, we must first understand the internal structure of the language. In this section we will first outline the overall structure of Idris and then dig into specific parts relevant to guarded recursion. This is not a thorough description of all of Idris's components, but rather an explanation of parts of the language. For more reading on this topic see Edwin Brady's .%todo: REF
\section{Overview}
%Idris -> Idris- -> TT -> Executable
\begin{figure}
\includegraphics[scale=0.9]{figures/Idris-overview}
\caption{The phases of the Idris compiler. Phases are shown as rectangles, and
  each transition (arrow) is annotated with the input or output representation
  of a given phase. Ovals designate endpoints.}
\label{fig:idris-overview}
\end{figure}
\subsection{Language Representations}
\begin{itemize}
\item \textbf{(concrete) Idris}
The high-level language in which Idris programs are written.
\item \textbf{(abstract) Idris}
The abstract representation of the high-level Idris language.
\item \textbf{\IdrisM}
A (strict) subset of abstract Idris without any syntactic sugar. Do-notation and infix operators
are desugared, and implicit arguments are bound explicitly. Note that
\IdrisM and abstract Idris are essentially the same language, where the
syntactic sugar from abstract Idris is reduced to desugared terms.
\item \textbf{TT}
The core type theory, TT, is a dependently typed lambda calculus with inductive
types and pattern matching. TT only allows pattern matching on top-level values,
so all \texttt{case}-expressions are converted to top-level pattern matching
during elaboration. In TT, all terms are fully annotated with their types and
all implicit arguments are explicit.
\item \textbf{Raw}
A (raw) representation of TT terms without any type information on terms. This
representation is used for type reconstruction during type checking.
\item \textbf{IBC}
Idris Byte Code (IBC) is the bytecode representation of an Idris program.
\end{itemize}

\subsection{Phases of Compilation}
\begin{itemize}
\item \textbf{Source}
The source code of the program, given in concrete Idris syntax.
\item \textbf{Parsing}
The parser generates an abstract syntax tree (abstract Idris) from the source code.
\item \textbf{Desugaring}
 In the desugaring phase, abstract Idris is reduced to \IdrisM through
 desugaring.
\item \textbf{Elaboration}
Elaboration reduces \IdrisM terms to terms in the core calculus, TT. The
elaboration phase consists of several notable sub-phases:
\begin{itemize}
\item \textit{Unification}
Unification is the process of substituting holes for appropriate terms. It
enables elaboration to progress gradually by continually unifying holes with
terms, until a complete TT term has been built. Further details will be provided
in Section~\ref{sec:elaboration}.
\item \textit{Case Tree Generation}
For each function definition a case tree\,\citep{Augustsson:1985} is generated,
describing the structure of the top-level pattern matching. These case trees are
used for coverage checking during the totality checking phase.
\item \textit{Totality Checking}
During the totality checking phase, a totality analysis of each function
definition is performed. First, a coverage analysis determines whether the
function in question is covering (using the previously generated case
trees). Next, a termination analysis based on the size-change principle is
performed on supposedly terminating functions, while a productivity analysis is
performed on supposedly productive functions via the syntactic guardedness principle.
\item \textit{Type Checking}
All TT terms resulting from the previous steps of elaboration are type checked
at the end of elaboration to ensure that no ill-typed terms are
constructed. Type checking proceeds by converting TT terms to Raw terms, and
then reconstructing the type of each Raw term according to the typing
environment. If the reconstructed type is convertible with the annotated type of
the original TT term, type checking succeeds; otherwise, it fails.
\end{itemize}
\item \textbf{IBC Generation}
After successful elaboration, an Idris Byte Code representation is
generated by a script which is built up gradually during elaboration. 
\item \textbf{IBC Compilation}
During compilation, the IBC representation is reduced to a binary
representation.
\item \textbf{Executable}
The final executable generated by the compiler.
\end{itemize}
\subsection{Internal Representations}
\subsection{High-Level Abstract Syntax}
%PDecl/PTerm
%	Top level abstract syntax
%	Functions with multiple clauses are multiple Decls
\subsection{Dependently Typed Lambda Calculus}
%		TT
%			Dependently typed lambda calculus
%			Wrapped in case trees
%			Data and Type constructors??
\section{Elaboration}
\label{sec:elaboration}
\section{Totality checking}
%		Happens during and after elaboration
%			What happens when and why?
%		Coverage: Case Trees
%		Termination: Size Change
%			When are the graphs build?
%		Productivity: Syntactic Guardedness
\section{Type checking}
%??

%########
%% TT
% Alt er eksplicit
% Typeregler

%%% Type checking

%% Idris- / Desugaring
% Desugaring er en transformation fra Idris- til Idris-

%% Elaboration
% Hvorfor elaboration?
% Faser "smelter sammen" 
% Teknisk forklaring (tactic prover)

%% Totality checking
% Size-change termination
% Nuværende implementation af produktivitetschecker
% Totality er en forudsætning for type checking
% Erasure? (måske)


%########

%%% Local Variables:
%%% mode: latex
%%% TeX-master: "../copatterns-thesis"
%%% End:


%!TEX root = ../copatterns-thesis.tex
\chapter{Adding Copatterns}
\label{sec:adding_copatterns}
Copatterns can be added to Idris without changing its core, the TT
calculus. Throughout this chapter, we will show how copatterns can be implemented as
syntactic sugar, i.e. a transformation from \IdrisM{} to \IdrisM{}, such that
all copattern clauses have been reduced to pattern clauses by the
time elaboration is reached.

% What, Why
\section{Intuition}
%#########
% Eliminationsregler => Introduktionsregler
% Top-level produktstruktur (een introduktionsregel)
% Pattern matching?
%#########
Following the discussion in Section~\ref{sec:prod-vs.-copr}, support for
copatterns have only been added for record types, both inductive and
coinductive. This decision is based upon the fact that Idris record types have
an obvious top-level structure, making them ideal companions for
copatterns. All examples are therefore given for record types only.

\subsection{The Duality Between Construction and Observation}
Desugaring copattern clauses, which give definitions by observation, to pattern
clauses, which give definitions by construction, relies on the duality between
introduction and elimination rules for (co)inductive types. For our first example,
we will consider the simple record type \texttt{Pair} in
Figure~\ref{fig:record_pair}. 
\begin{figure}
\begin{lstlisting}[mathescape]
record Pair (a : Type) (b : Type) where
  MkPair : (fst : a) $\to$ (snd : b) $\to$ Pair a b
\end{lstlisting}
  \centering
  $\begin{matrix} 
    \frac { \Gamma \,\vdash\, x\,:\,a \quad \Gamma \,\vdash\, y\,:\,b }{
      \Gamma\, \vdash\,MkPair\,x\,y\,:\,Pair\,a\,b} \scriptstyle I_{Pair}
  & \frac { \Gamma \,\vdash\, e\,:\,Pair\, a\,b }{ \Gamma\,
    \vdash\,fst(e)\,:\,a} \scriptstyle E_{fst}
  & \frac { \Gamma\, \vdash\, e\,:\,Pair\, a\,b }{ \Gamma\, \vdash\,
    snd(e)\,:\,b } \scriptstyle E_{snd}  \end{matrix}$
  $\vspace{12pt}$
  $\begin{matrix}
      \frac { \Gamma\, \vdash\, MkPair\,x\,y\,:\,Pair\,a\,b }{ \Gamma
      \,\vdash\, fst(MkPair\,x\,y)\,\equiv\,x  } \eta_{fst}  & \frac { \Gamma\, \vdash\,
      MkPair\,x\,y\,:\,Pair\,a\,b }{ \Gamma\, \vdash \,snd(MkPair\,x\,y)\,\equiv\,y } \eta_{snd} \end{matrix}$
  \caption{A record defining a type \texttt{Pair}, along with its introduction,
    elimination, and $\eta$-rules.}
  \label{fig:record_pair}
\end{figure}
The traditional way of creating a value of type \texttt{Pair} is by construction
using pattern clauses, as shown in Figure~\ref{fig:pair_example_pattern}. Here,
we construct a value of \texttt{Pair} using its introduction rule, giving the values
\texttt{``Idris''} and \texttt{42} as arguments. We can observe the data in the
constructed value by using the elimination rules \texttt{fst} and \texttt{snd},
which have corresponding $\eta$-rules as expected (see
Figure~\ref{fig:record_pair}). As seen from
Figure~\ref{fig:pair_example_pattern_typing_derivation}, such observations
provide unsurprising results, since \texttt{MkPair} simply acts as a container
for the observed data.

\begin{figure}
\begin{lstlisting}
pattern : Pair String Nat
pattern = MkPair "Idris" 42
\end{lstlisting}
  \caption{A value of \texttt{Pair} defined by construction.}
  \label{fig:pair_example_pattern}
\end{figure}

\begin{figure}
\centering
    $
     \frac { \Gamma \,\vdash\, "Idris"\,:\,String \quad \Gamma \,\vdash\, 42\,:\,Nat }{
      \Gamma\, \vdash\,MkPair\,"Idris"\,42\,:\,Pair\,String\,Nat} \scriptstyle
    I_{Pair}$
   $\vspace{10pt}$
  $\begin{matrix} \frac { \Gamma \,\vdash\,MkPair\,"Idris"\,42\,:\,Pair\, String\,Nat }{ \Gamma\,
    \vdash\,fst(MkPair\,"Idris"\,42)\,:\,String} \scriptstyle E_{fst}
  & \frac { \Gamma\, \vdash\,MkPair\,"Idris"\,42\,:\,Pair\,String\,Nat }{ \Gamma\, \vdash\,
    snd(MkPair\,"Idris"\,42)\,:\,Nat } \scriptstyle E_{snd}  \end{matrix}$
   $\vspace{10pt}$
  $\begin{matrix}
      \frac { \Gamma\, \vdash\, MkPair\,"Idris"\,42\,:\,Pair\,String\,Nat }{ \Gamma
      \,\vdash\, fst(MkPair\,"Idris"\,42)\,\equiv\,"Idris"  } \eta_{fst}  & \frac { \Gamma\, \vdash\,
      MkPair\,"Idris"\,42\,:\,Pair\,String\,Nat }{ \Gamma\, \vdash \,snd(MkPair\,"Idris"\,42)\,\equiv\,42 } \eta_{snd} \end{matrix}$
  \caption{Using the rules for \texttt{Pair} on the example in Figure~\ref{fig:pair_example_pattern}.}
  \label{fig:pair_example_pattern_typing_derivation}
\end{figure}

The idea behind copatterns is that whenever a (co)data constructor simply acts
as a container for values, we can treat the (co)data constructor as a black box,
defining only the values we expect to observe when making specific observations
on the constructed (co)data. As an example,
Figure~\ref{fig:pair_example_copatterns} shows how we can use copatterns to
define a program by observation which is equivalent to the definition by
construction in Figure~\ref{fig:pair_example_pattern}. The most striking
difference is that \texttt{pattern} itself is never directly defined. Instead,
it is implicitly defined, since the two copattern clauses provide all the data
necessary to derive a corresponding pattern clause which defines
\texttt{pattern} explicitly, as in Figure~\ref{fig:pair_example_pattern}. From
this observation, we arrive at the main intuition behind our desugaring
procedure: Any definition given by copattern clauses has an equivalent
definition given by pattern clauses, and this definition can be derived without
providing any further information. To make this intuition concrete, consider the
derivation in
Figure~\ref{fig:pair_elimination_on_introduction_rules_rewrite}. Rewriting with
the observations defined in Figure~\ref{fig:pair_example_copatterns}, the
elimination rules for \texttt{Pair} are used to directly derive the right-hand
side of the \texttt{pattern} example from
Figure~\ref{fig:pair_example_pattern}. 

\begin{figure}
\begin{lstlisting}
pattern' : Pair String Nat
fst pattern' = "Idris"
snd pattern' = 42
\end{lstlisting}
  \caption{A value of \texttt{Pair} defined by construction.}
  \label{fig:pair_example_copatterns}
\end{figure}

\begin{figure}
  \centering
$\frac { \begin{matrix} \frac { \Gamma \, \vdash \,pattern'\, :\,
      Pair\, String\, Nat }{ \Gamma \, \vdash \, "Idris"\,:\,String
    } \scriptstyle E_{fst}   & \frac { \Gamma \, \vdash \,pattern'\, :\, Pair\, String\,
      Nat }{ \Gamma \, \vdash \, 42\,:\,Nat } \scriptstyle E_{snd}  \end{matrix} }{
  \Gamma \, \vdash \, MkPair\, "Idris"\, 42\, :\, Pair\, String\, Nat } \scriptstyle I_{ Pair }$
  \caption{Deriving the right-hand side of \texttt{pattern} from \texttt{pattern'}.}
  \label{fig:pair_elimination_on_introduction_rules_rewrite}
\end{figure}

\subsection{Desugaring Copatterns}
While the derivation in
Figure~\ref{fig:pair_elimination_on_introduction_rules_rewrite} provides a fine
display of our intention, such a direct derivation is only possible for simple
examples which have only one resulting pattern clause. In particular, desugaring
copatterns becomes more complex as the following challendes must be taken into account:

\begin{enumerate}
\item Nested copatterns
\item Pattern matching
\item Subsumed copattern clauses
\item Non-covering definitions
\end{enumerate}

To address each of these challenges, we have devised a desugaring procedure for
copatterns which operates in three steps:

\begin{enumerate}
\item \textbf{Expansion}, in which right-hand sides are expanded into partially
  defined constructor invocations.
\item \textbf{Reduction}, in which copatterns are eliminated from the reduction rules.
\item \textbf{Merging}, in which compatible clauses are merged into one clause.
  \begin{enumerate}
  \item Identify compatible clauses
  \item Merge compatible clauses
  \end{enumerate}
\end{enumerate}

Before discussing how our solution tackles each of the challenges, the inner
workings of the procedure will be illustrated with a simple example. Consider
the program \texttt{zeros} in Figure~\ref{fig:zeros}, which produces a value of
the type \texttt{Stream} defined in Figure~\ref{fig:corecord_stream}. In the
first step of desugaring \texttt{zeros}, expansion, the right-hand side of each
copattern clause is expanded into an explicit projection on a partially defined
Stream. Thus, the \texttt{head} clause of Figure~\ref{fig:desugared_zeros_step1}
may be read as: ``The \texttt{head} of \texttt{zeros} is the \texttt{head} of
the partially defined \texttt{Stream} which has \texttt{Z} as its first
constructor argument''. A corresponding reading exists for the \texttt{tail}
clause. For constructor arguments where no output is defined, Idris
metavariables are inserted, e.g. \texttt{?zeroshead}.

\begin{figure}
\begin{lstlisting}[mathescape]
corecord Stream (a : Type) where
  head : a
  tail : Stream a
  constructor (::)
\end{lstlisting}
  \centering
  $\begin{matrix} 
    \frac { \Gamma \,\vdash\, x\,:\,a \quad \Gamma \,\vdash\, xs\,:\,Stream\,a }{
      \Gamma\, \vdash\,(x\,::\,xs)\,:\,Stream \,a} \scriptstyle I_{Stream}
  & \frac { \Gamma \,\vdash\, s\,:\,Stream\, a }{ \Gamma\,
    \vdash\,head(s)\,:\,a} \scriptstyle E_{head}
  & \frac { \Gamma\, \vdash\, s\,:\,Stream\, a }{ \Gamma\, \vdash\,
    tail(s)\,:\,Stream\, a } \scriptstyle E_{tail}  \end{matrix}$
  $\vspace{10pt}$
  $\begin{matrix}
      \frac { \Gamma\, \vdash\, (x\,::\,xs\,:\,Stream\,a }{ \Gamma
      \,\vdash\, head(x\,::\,xs)\,\equiv\,x  } \eta_{head}  & \frac { \Gamma\, \vdash\,
      (x\,::\,xs)\,:\,Stream\,a }{ \Gamma\, \vdash \,tail(x\,::\,xs)\,\equiv\,xs } \eta_{tail} \end{matrix}$
  \caption{A coinductive record type \texttt{Stream} defining an infinite list,
    along with introduction, elimination, and $\eta$-rules.}
  \label{fig:corecord_stream}
\end{figure}

\begin{figure}
\begin{lstlisting}[mathescape]
zeros : Stream Nat
head zeros = Z
tail zeros = zeros
\end{lstlisting}
  \caption{A function \texttt{zeros} defining a stream of 0.}
  \label{fig:zeros}
\end{figure}

In the second step, reduction, each clause from
Figure~\ref{fig:desugared_zeros_step1} is reduced by removing equivalent
projections on both side of a clause, similar to how one would reduce a
mathematical equation. The result of the reduction is shown in
Figure~\ref{fig:desugared_zeros_step2}. We are allowed to perform such
reductions at this point because all the clauses make right-hand side
projections directly on the (as yet unmerged) output of the function, while all
the left-hand side projections happen directly on the input.

\begin{figure}
\begin{lstlisting}[mathescape]
zeros : Stream Nat
head zeros = head (Z :: ?zerostail)
tail zeros = tail (?zeroshead :: zeros)
\end{lstlisting}
  \caption{Desugaring \texttt{zeros}, step 1: Expansion.}
  \label{fig:desugared_zeros_step1}
\end{figure}

\begin{figure}
\begin{lstlisting}[mathescape]
zeros : Stream Nat
-- Before: head zeros = head (Z :: ?zerostail)
-- Eliminate 'head' on each side
zeros = Z :: ?zerostail
-- Before: tail zeros = tail (?zeroshead :: zeros)
-- Eliminate 'tail' on each side
zeros = ?zeroshead :: zeros
\end{lstlisting}
  \caption{Desugaring \texttt{zeros}, step 2: Reduction.}
  \label{fig:desugared_zeros_step2}
\end{figure}

The final step, merging, is actually a two-step process, where compatible clauses are
merged with each other such that their right-hand sides are combined into
possibly complete constructor invocations. Some of the right-hand sides may remain
incomplete after the merging step if the definition we are desugaring is not
covering. In Figure~\ref{fig:desugared_zeros_step3a}, the clauses of
\texttt{zeros} which are compatible with each other are identified. Since
\texttt{zeros} is a simple example, the two clauses are compatible with each
other because their left-hand sides are identical. Therefore, the two clauses are
merged into one, as shown in Figure~\ref{fig:desugared_zeros_step3b}.

\begin{figure}
\begin{lstlisting}[mathescape]
zeros : Stream Nat
-- The two clauses are compatible with each other
(1) zeros = Z :: ?zerostail
(2) zeros = ?zeroshead :: zeros
\end{lstlisting}
  \caption{Desugaring \texttt{zeros}, step 3a: Identifying compatible
    clauses. The clauses have been numbered for reference.}
  \label{fig:desugared_zeros_step3a}
\end{figure}

\begin{figure}
\begin{lstlisting}[mathescape]
zeros : Stream Nat
(1,2) zeros = Z :: zeros 
\end{lstlisting}
  \caption{Desugaring \texttt{zeros}, step 3b: Merging compatible clauses. The
    numbers indicate that both of the clauses from
    Figure~\ref{fig:desugared_zeros_step3a} have been merged into one clause.}
  \label{fig:desugared_zeros_step3b}
\end{figure}

\begin{definition}[\textit{Compatibility}]
\label{def:compatibility}
  Compatibility is a reflexive, transitive, and anti-symmetric binary relation
  $C$ on pattern clauses which describes whether the pattern in one pattern
  clause is more general or equally general to the pattern in another pattern
  clause. Let $V$ be a function which, given a subpattern, returns the set of
  values matched by that subpattern. For two pattern clauses
  $a\,p_{1}\,p_{2}\,\cdots\,p_{n}\,=\,rhs_{a}$ and
  $b\,q_{1}\,q_{2}\,\cdots\,q_{n}\,=\,rhs_{b}$ of an $n$-ary function,
  $(a,b)\in C$ if $V(p_{i})\subseteq V(q_{i})$ for $0 < i\le n$. For a nullary
  function, all pattern clauses are trivially compatible.
\end{definition}
% if the set of values
% for each argument $i$ matched by the pattern of $a$, $P_{i}(a)$, is a subset of of the set of
% values for each argument $i$ matched by the pattern of $b$, $P_{i}(b)$ i.e.
% $P_{i}(a)\subseteq P_{i}(b)$ for all $0 < i\le n$. 

\subsubsection{Addressing the Challenges}
The three-step desugaring procedure presented above has been devised as a
unified solution to four main challenges. As we go through each of the
challenges, the \texttt{dupNth} program in Figure~\ref{fig:dupNth} will serve as
a running example of the desugaring of a non-trivial definition with copattern
clauses.

\paragraph{Challenge 1: Nested Copatterns}
Copatterns, i.e. left-hand side projections, can be nested arbitrarily as long
as the nesting is well-typed. As described in
Chapter~\ref{cha:coind-records-idris}, definitions with copatterns only behave
as intended when all projection names are unique. Therefore, our solution
assumes that projection names not only indicate semantics, but also argument
positions to the corresponding constructor. To illustrate, consider the
transformation of clause (2) from Figure~\ref{fig:dupNth} to the expanded clause
\texttt{(2)} in Figure~\ref{fig:desugared_dupNth_step1}. Because this clause
defines the \texttt{head} of the \texttt{tail} of the output, the expanded
clause has the right-hand side, \texttt{head~s}, inserted at this position in
the partially defined constructor on the right-hand side of (2) in
Figure~\ref{fig:desugared_dupNth_step1}. The same reasoning holds for the remaining clauses.

In general, nested copatterns are handled in the first two steps, expansion and
reduction. Expansion ensures that the original right-hand sides are inserted
into the correct positions in the expanded constructor invocations, while
reduction eliminates all left-hand side projections through equational
reasoning. A different solution to the problem of handling nested copatterns has
been proposed by Setzer et al.\,\citep{Setzer14Unnesting}, in which auxiliary
functions are created for each level of nested copatterns. The difference
between this and our approach will be discussed in Section~\ref{sec:related_work_copatterns}.

\paragraph{Challenge 2: Pattern Matching}
Whenever the copattern clauses for a definition have different patterns on the
left-hand side, these patterns must be preserved in the desugared version of the
definition. After eliminating copatterns in the reduction step
(Figure~\ref{fig:desugared_dupNth_step2}), preservation of pattern matching is
ensured in the first sub-step of merging
(Figure~\ref{fig:desugared_dupNth_step3a}), where compatible clauses are
identified. In the following sub-step, shown in
Figure~\ref{fig:desugared_dupNth_step3b}, only compatible clauses are
merged. Merging only compatible clauses ensures that a clause is only merged
with other clauses which match the same or a more general set of input values. As an example, we will show exactly why clause (2) is compatible
with clause (1) as postulated in
Figure~\ref{fig:desugared_dupNth_step3a}. The proof
proceeds in Figure~\ref{fig:compatibility_proof}. Note that if, in Figure~\ref{fig:compatibility_proof}, $m_{1}$ had been
$S\,m_{1}$ instead, $(2,1)\notin C$ since $V(S\,m_{1}) =
\{\,x\,|\,x\,:\,Nat\,\setminus\{Z\}\}$, and $V(Z)\not\subseteq V(S\,m_{1})$.

\begin{figure}
From
Definition~\ref{def:compatibility}, we must show: $(2,1)\in C$, where (2) and (1) are clauses from
Figure~\ref{fig:desugared_dupNth_step3a}.
\\\\
We consider the left-hand sides of (2) and (1), respectively:\\
$lhs_{(2)} = dupNth\,s_{(2)}\,n_{(2)}\,Z$\\
$lhs_{(1)} = dupNth\,s_{(1)}\,n_{(1)}\,m_{(1)}$\\
\\
For each sub-pattern, we define the set of matching values:\\
$V(s_{2}) = \{\,x\,|\,x\,:\,Stream\,a\}$\\
$V(n_{2}) = \{\,x\,|\,x\,:\,Nat\}$\\
$V(Z) = \{Z\}$\\
$V(s_{1}) = \{\,x\,|\,x\,:\,Stream\,a\}$\\
$V(n_{1}) = \{\,x\,|\,x\,:\,Nat\}$\\
$V(m_{1}) = \{\,x\,|\,x\,:\,Nat\}$\\
\\
Having defined these, the following must hold:
\\
$V(s_{2})\subseteq V(s_{1}) = \{\,x\,|\,x\,:\,Stream\,a\}\subseteq\,\{\,x\,|\,x\,:\,Stream\,a\}\,\qed$\\
$V(n_{2})\subseteq V(n_{1}) =
\{\,x\,|\,x\,:\,Nat\}\subseteq\,\{\,x\,|\,x\,:\,Nat\}\,\qed$\\
$V(Z)\subseteq V(m_{1}) = \{Z\}\subseteq\,\{\,x\,|\,x\,:\,Nat\}\,\qed$\\\\
Consequently, $(2,1)\in C$
  \caption{From Figure~\ref{fig:desugared_dupNth_step3a}, clause (2) is
    compatible with clause (1).}
  \label{fig:compatibility_proof}
\end{figure}


\paragraph{Challenge 3: Subsumed Copattern Clauses}
In addressing Challenge 2, we stated that different left-hand side patterns must
be preserved in the output. Nevertheless, the pattern from clause (1) in
Figure~\ref{fig:desugared_dupNth_step3a} does not occur in the
desugared version of the function in
Figure~\ref{fig:desugared_dupNth_step3b}. Due to the fact that clause (1) is
more general than all the other clauses, it is subsumed by these in the final
output, such that the right-hand side of (1) is a part of the right-hand side of
both the desugared clauses. The set of subsumed copattern clauses for a
definition are exactly the clauses which are compatible with all other clauses.  

\paragraph{Challenge 4: Non-covering Defintions} As Idris allows users to
define functions which are not covering (see
Definition~\ref{def:covering_function}), partial definitions with copatterns
should be possible as well. With our desugaring procedure, the solution to this
challenge is provided ``for free'', since the metavariables which are inserted
during the expansion phase are simply not replaced with any proper definition if
none is given. A non-covering version of the \texttt{dupNth} function is shown
in Figure~\ref{fig:dupNth_partial}, along with its desugared version.

\begin{figure}
\begin{lstlisting}[mathescape]
||| Duplicate every nth element of s, 
||| starting with the mth element.
||| @s the stream
||| @n how often
||| @m where to start
dupNth : Stream a $\to$ Nat $\to$ Nat $\to$ Stream a
(1) head       (dupNth s n m)       = head s
(2) head (tail (dupNth s n Z))      = head s
(3) head (tail (dupNth s n (S m'))) = head (tail s)
(4) tail (tail (dupNth s n Z))      = tail (dupNth s n n)
(5) tail (tail (dupNth s n (S m'))) = 
                              tail (dupNth (tail s) n m')
\end{lstlisting}
  \caption{A function duplicating the nth element of a stream. The numbers have been
  added to each clause for reference, and are not a part of the actual implementation.}
  \label{fig:dupNth}
\end{figure}

\begin{figure}
\begin{lstlisting}[mathescape]
dupNth : Stream a $\to$ Nat $\to$ Nat $\to$ Stream a
(1) head       (dupNth s n m)       =
      head (head s :: ?dupNthtail)
(2) head (tail (dupNth s n Z))      = 
      head (tail (?dupNthhead :: (head s :: ?dupNthtail)))
(3) head (tail (dupNth s n (S m'))) =
      head (tail (?dupNthhead :: 
                 (head (tail s) :: ?dupNthtail)))
(4) tail (tail (dupNth s n Z))      =
      tail (tail (?dupNthhead :: 
                 (?dupNthheadtail :: tail (dupNth s n n))))
(5) tail (tail (dupNth s n (S m'))) =
      tail (tail (?dupNthhead :: 
                 (?dupNthheadtail :: 
                  tail (dupNth (tail s) n m'))))
\end{lstlisting}
  \caption{Desugaring \texttt{dupNth}, step 1: Expansion.}
  \label{fig:desugared_dupNth_step1}
\end{figure}

\begin{figure}
\begin{lstlisting}[mathescape]
dupNth : Stream a $\to$ Nat $\to$ Nat $\to$ Stream a
(1) dupNth s n m       =
      head s :: ?dupNthtail
(2) dupNth s n Z      = 
      ?dupNthhead :: (head s :: ?dupNthtail)
(3) dupNth s n (S m') =
      ?dupNthhead :: (head (tail s) :: ?dupNthtail)
(4) dupNth s n Z      =
      ?dupNthhead :: (?dupNthheadtail :: 
                      tail (dupNth s n n))
(5) dupNth s n (S m') =
      ?dupNthhead :: (?dupNthheadtail :: 
                      tail (dupNth (tail s) n m'))
\end{lstlisting}
  \caption{Desugaring \texttt{dupNth}, step 2: Reduction.}
  \label{fig:desugared_dupNth_step2}
\end{figure}

\begin{figure}
\begin{lstlisting}[mathescape]
dupNth : Stream a $\to$ Nat $\to$ Nat $\to$ Stream a
-- The 'dupNth s n m' clause is compatible with 
-- all of the other clauses.
(1) dupNth s n m = head s :: ?dupNthtail
-- compatible with (1), (2), and (4)
(2) dupNth s n Z      = 
     ?dupNthhead :: (head s :: ?dupNthtail)
-- compatible with (1), (3), and (5)
(3) dupNth s n (S m') =
     ?dupNthhead :: (head (tail s) :: ?dupNthtail)
-- compatible with (1), (2), and (4)
(4) dupNth s n Z      =
     ?dupNthhead :: (?dupNthheadtail :: 
                     tail (dupNth s n n))
-- compatible with (1), (3), and (5)
(5) dupNth s n (S m') =
     ?dupNthhead :: (?dupNthheadtail :: 
                     tail (dupNth (tail s) n m'))
\end{lstlisting}
  \caption{Desugaring \texttt{dupNth}, step 3a: Identifying compatible
    clauses. The clauses have been numbered for reference.}
  \label{fig:desugared_dupNth_step3a}
\end{figure}

\begin{figure}
\begin{lstlisting}[mathescape]
dupNth : Stream a $\to$ Nat $\to$ Nat $\to$ Stream a
(1,2,4) dupNth s n Z = 
          head s :: (head s :: tail (dupNth s n n))
(1,3,5) dupNth s n (S m') = 
          head s :: 
          (head (tail s) :: tail (dupNth (tail s) n m'))
\end{lstlisting}
  \caption{Desugaring \texttt{dupNth}, step 3b: Merging compatible
    clauses. The numbers indicate which of the clauses from
    Figure~\ref{fig:desugared_dupNth_step3a} that have been merged into each clause.}
  \label{fig:desugared_dupNth_step3b}
\end{figure}

\begin{figure}
\begin{lstlisting}[mathescape]
dupNth : Stream a $\to$ Nat $\to$ Nat $\to$ Stream a
head (tail (dupNth s n Z))      = head s
head (tail (dupNth s n (S m'))) = head (tail s)
tail (tail (dupNth s n Z))      = tail (dupNth s n n)
tail (tail (dupNth s n (S m'))) = tail (dupNth (tail s) n m')
\end{lstlisting}
\begin{lstlisting}[mathescape]
dupNth : Stream a $\to$ Nat $\to$ Nat $\to$ Stream a
dupNth s n Z = 
 ?dupNthhead :: (head s :: tail (dupNth s n n))
dupNth s n (S m') = 
 ?dupNthhead1 :: (head (tail s) :: 
                  tail (dupNth (tail s) n m'))
\end{lstlisting}

  \caption{Above: A non-covering version of \texttt{dupNth}, where the \texttt{head}
    clause has been omitted. Below: its desugared version.}
  \label{fig:dupNth_partial}
\end{figure}


% How
\section{Implementing Copatterns in Idris}

%\subsection{Implementing Copatterns}
%#############
% Hvordan foregår desugaring? 
% Hvordan implementeres intuitionen?
%% Hvilke dele af Idris (jf. figur) berører vi?
% Hvorfor desugaring, i modsætning til elaboration? (mere code sharing)
% Hvorfor udvides TT ikke?

% Identifying copatterns
% Pattern matching
% where-blocks
% Parsing?
% with-rule
% Eksempler (simpelt + avanceret)
% Discussion
%############

\section{Discussion}
% Is this the best approach to desugaring?
% Would a TT-based approach have been better?
% Discuss 'Unnesting of copatterns' approach
% fejlmeddelelser

\begin{lstlisting}[mathescape]
dupNth : Stream a -> Nat -> Nat -> Stream a
head (dupNth s n m) = head s
tail (dupNth s n m) = dupNth' s n m

dupNth' : Stream a -> Nat -> Nat -> Stream a
dupNth' s n Z = dupNthZ s n
dupNth' s n (S m') = dupNthS s n m'

dupNthZ : Stream a -> Nat -> Stream a
head (dupNthZ s n) = head s
tail (dupNthZ s n) = tail (dupNth s n n)

dupNthS : Stream a -> Nat -> Nat -> Stream a
head (dupNthS s n m') = head (tail s)
tail (dupNthS s n m') = tail (dupNth (tail s) n m')
\end{lstlisting}

%%% Local Variables:
%%% mode: latex
%%% TeX-master: "../copatterns-thesis"
%%% End:


\chapter{Inference of Guarded Recursion}
\label{cha:infer-guard-recurs}
In Section~\ref{sec:less-restr-prod}, we provided the motivation behind
extending Idris with a less restrictive productivity checker using guarded
recursion. As we saw in Section~\ref{sec:guarded-recursion}, defining guarded
recursive programs can be quite tedious. Also, it requires the user to thoroughly
understand guarded recursion in order to write provably productive programs. In
this chapter, we propose a system where the productivity of programs can be
verified using guarded recursion with only a minimal amount of user
involvement. The main focus will be on implementation, showing how we have put theory into practice.

\section{Practical Considerations}
Applying a theoretical model for productivity to actual Idris programs means
that we must address several practical issues concerning the Idris compiler.

\subsection{Target Language}
As discussed in Chapter~\ref{cha:idris}, an Idris program has multiple different
representations throughout the compilation process, namely concrete Idris,
\IdrisM{}, TT, and machine code. Therefore, the compilation
stage and program representation chosen for the productivity analysis has great
impact on how a solution unfolds.

Since guarded recursion is a typing discipline, the system needs the ability to
both infer and check types of arbitrary terms. This is not possible prior to
elaboration, because type resolution may require evaluation. Because program
structure is important for the analysis, the machine code would be impractical
as a target language. After elaboration, the resulting TT
terms have all type information available, while program structure is still
preserved. Hence, the productivity analysis is performed on elaborated TT terms.

\subsection{Lifting Parameters} % -- Subsubsection til forrige section?
\label{sec:handling-parameters}
Recall the typing rule for the later application operator, $\tensor^\kappa$, in
Figure~\ref{fig:guarded_recursion_rules_clocks}:

\[
\frac { \Delta ;\Gamma \vdash f: \laterkappa(A\rightarrow B)\quad \quad \Delta;\Gamma \vdash e : \laterkappa A }{ \Delta;\Gamma \vdash f \tensor e : \laterkappa B } { \tensor }_{ I }
\]
%%% Local Variables: 
%%% mode: latex
%%% TeX-master: t
%%% End: 


While this rule provides us with a mechanism for sequential application of
values available later, it only works for the simply typed case. Notably, it
does not take substitution in the resulting type into account, so the function
argument to $\tensor^\kappa$, $f$, can never be applied to an argument $e$ which
refines the type of $f$, e.g. a type argument. The following function,
\texttt{repeat}, requires an application of a type argument:

\begin{lstlisting}[mathescape,title=\ttBlock]
repeat : $\forall \kappa.$ ((a : Type) $\rightarrow$ a $\rightarrow$ Stream$^{\kappa}$ a)
repeat = $\Lambda\kappa.$ fix$^{\kappa}$($\lambda{}$rec$.\lambda{}$a$.\lambda{}$n$.$ 
             StreamCons a n ((rec $\tensor ^{\kappa}$ (Next$^{\kappa}$ a)) $\tensor ^{\kappa}$ (Next$^{\kappa}$ n)))
\end{lstlisting}

By fixed point elimination, the recursive reference is under an application
of \texttt{Next$^\kappa$}, making it available later. Therefore, it must be
applied to its type argument by later application, but such an application
is not well-typed, as shown in Figure~\ref{fig:repeat_failed_typing}. In
particular, the type argument is not bound as the type variable \texttt{a},
since this connection is not expressed in the typing rule.

\begin{figure}[h]
\centering
\[
\frac { \begin{matrix} \inference { \frac { ? }{ \Gamma '\, \vdash \, (Next^{\kappa}\,repeat[\kappa])\, :\, \later
        ^{\kappa}(Type\, \rightarrow \, a\, \rightarrow \, Stream^{\kappa}\, a) } \, 
      \frac {
        \frac {  }{ \Gamma '\, \vdash \, a\, :\, Type } 
      }
      { \Gamma '\, \vdash \,
        Next^{\kappa}\, a\, :\, \later ^{\kappa}\, Type }
    }{ \, \Gamma '\, \vdash \, (Next^{\kappa}\,repeat[\kappa])\,
      \tensor ^{\kappa} \, (Next^{\kappa}\, a)\, :\, \later ^{\kappa}(a\, \rightarrow \, Stream^{\kappa}\,
      a)\ }  & \inference { \inference {  }{ \Gamma '\, \vdash \, n\, :\, a }  }{
      \Gamma '\, \vdash \, Next^{\kappa}\, n\, :\, \later ^{\kappa} \, a }  \end{matrix} }{
  \Gamma '\, \vdash \, ((Next^{\kappa}\,(repeat[\kappa]))\, \tensor ^{\kappa} \, (Next^{\kappa}\, a))\, \tensor ^{\kappa} \,
  (Next^{\kappa}\, n))\, :\, \later ^{\kappa} Stream^{\kappa}\, a }
\]
\[
\Gamma '\, =\, \Gamma ,\, repeat\, :\, \forall\kappa.((a\, :\, Type)\, \rightarrow \,
a\, \rightarrow \, Stream^{\kappa}\, a),\, a\, :\, Type,\, n\, :\, a
\]
  \caption{A failed attempt at typing the tail argument to the
    \texttt{StreamCons} constructor in the definition of \texttt{repeat}.}
  \label{fig:repeat_failed_typing}
\end{figure}

One way of solving this problem would be to change the typing rule for later
composition, such that it takes substitution in types into account, as shown in
Figure~\ref{fig:tensor_with_subst}. However, this formulation of the rule has
yet to be proven sound.

\begin{figure}[h]
\[
\inference { \Gamma \, \vdash \, t\, :\, \later^{\kappa} ((a\, :\, A)\,
  \rightarrow \, B)\quad \Gamma \, \vdash \, u\, :\, \later^{\kappa} A }{
  \Gamma \, \vdash \, t\, \tensor ^{\kappa} \, u\, :\, \later^{\kappa} B[{ u
  }/{ a }] } 
\]
  \caption{A potential rule for later application with substitution in
    types. However, the type $\later^{\kappa} B[{ u }/{ a }]$ in the conclusion
    is not well-formed according to the rules of guarded recursion presented in Figure~\ref{fig:guarded_recursion_dependent_rules}.}
  \label{fig:tensor_with_subst}
\end{figure}

Instead, we can change which arguments appear under the fixed point. Any
parameter which would have to be substituted in the type is therefore lifted out
of the fixed point operator, 
thus changing the type of the recursive reference. This gives us
the definition of \texttt{repeat} in
Figure~\ref{fig:repeat_guarded_example_new}, where the type of the recursive
reference is \texttt{$\later ^{\kappa}$(a~$\rightarrow$~Stream$^{\kappa}$
  a)}. This allows us to type part of \texttt{repeat} as exemplified in
Figure~\ref{fig:repeat_typing_new}, because the type parameter is now fixed
outside of the fixed point operator.

\begin{figure}[h]
  \begin{lstlisting}[mathescape,title=\ttBlock]
repeat : (a : Type) $\to$ $\forall \kappa.$ (a $\to$ Stream$^{\kappa}$ a)
repeat a = $\Lambda\kappa.$ fix$^{\kappa}$($\lambda{}$rec$.\lambda{}$n$.$ 
             StreamCons a n (rec $\tensor ^{\kappa}$ (Next$^{\kappa}$ n)))
\end{lstlisting}
  \caption{A definition of \texttt{repeat} where the type parameter is fixed
    outside of the fixed point operator.}
  \label{fig:repeat_guarded_example_new}
\end{figure}

\begin{figure}[h]
\[
\frac { \frac {  }{ \Gamma '\, \vdash \, rec\, :\, \later ^{\kappa}(a\, \rightarrow
    \, Stream^{\kappa}\, a) } \, \frac { \Gamma '\, \vdash \, n\, :\, a }{ \Gamma '\,
    \vdash \, Next^{\kappa}\, n\, :\, \later ^{\kappa}\, a }  }{ \, \Gamma '\, \vdash \, rec\,
  \tensor ^{\kappa} \, (Next^{\kappa}\, n)\, :\, \later ^{\kappa}Stream^{\kappa}\, a }
\]
\[
 \Gamma '\, =\, \Gamma ,\, rec\, :\, \later ^{\kappa}(a\, \rightarrow \, Stream^{\kappa}\,
 a),\, a\, :\, Type,\, n\, :\, a
\]

  \caption{Part of typing repeat with fixed type parameter.}
  \label{fig:repeat_typing_new}
\end{figure}

The problem described here applies to all names bound by dependent function
types that occur free in the body type. Therefore, any such parameter must be
lifted out of the fixed point in this manner. Note that names bound to a type
depending on a clock, e.g. $Stream^{\kappa}$, cannot be lifted out of the clock
quantification. Presently, support for clock-dependent types has been deemed out
of scope. For further elaboration, see Section~\ref{sec:guarded-types}.

\subsection{Eliminating the \texttt{fix$^\kappa$} Rule}
\label{sec:fixkappa-rule}
In the following, we describe how one can supposedly encode dependent functions
under clock quantification by eliminating the guarded recursive fixed
point. While eliminating the fixed point seems to work as intended up to polymorphic
types, the theory behind it does not cover dependent function types. This
limitation was discovered late in the process, and is therefore a part of the
current implementation. Since it does work for non-dependent functions, we
describe the technique here, and use it throughout the remainder of the
chapter. The implications of using this technique for dependent function types
is discussed in Section~\ref{sec:depend-funct-types}.

In the theoretical model of guarded recursion, all programs must be encoded
inside a special fixed point operator, as shown in
Section~\ref{sec:guarded-recursion}. Ideally, we could simply port this idea to
Idris, encoding TT programs inside a TT definition of the fixed point
operator. While this would work fine for single-clause definitions, it becomes
problematic when we consider definitions with multiple clauses, such as the
following: 

\begin{lstlisting}[mathescape,title=\ttBlock]
cycle : Nat $\to$ Nat $\to$ Stream Nat
cycle Z     m = (::) Nat Z (cycle m m)
cycle (S n) m = (::) Nat (S n) (cycle n m)
\end{lstlisting}

When transforming \texttt{cycle} to guarded recursive form, both clauses must be
part of the same fixed point definition. A seemingly obvious solution is to
convert the left-hand side pattern matching structure in \texttt{cycle} into
right-hand side \texttt{case}-expressions. For two reasons, this is not
possible: (1) TT does not have \texttt{case}-expressions, so the definition of
\texttt{cycle} is elaborated into two TT clauses, and (2) Idris
\texttt{case}-expressions only support non-dependent pattern matching. Hence,
the problem persists after elaboration.

Instead, we could attempt to define \texttt{cycle} as two functions: A
single-clause definition with the fixed point operator, and an auxiliary
function where pattern matching is performed. With this solution, all the clauses of
\texttt{cycle} are part of the same fixed point, where the guarded recursive reference,
\texttt{rec}, is given to \texttt{cycle} as an argument:

\begin{lstlisting}[mathescape, title=\ttBlock]
cycle$'$ : $\laterkappa\,$(Nat $\to$ Nat $\to$ Stream$^{\kappa}\,$Nat) $\to$ Nat $\to$ Nat $\to$ Stream$^{\kappa}$ Nat
cycle$'$ rec  Z    m = StreamCons Z  ((rec $\tensor ^{\kappa}$ (Next$^{\kappa}$ m)) $\tensor ^{\kappa}$ (Next$^{\kappa}$ m))
cycle$'$ rec (S n) m = StreamCons (S n) ((rec $\tensor ^{\kappa}$ (Next$^{\kappa}$ n)) $\tensor ^{\kappa}$ (Next$^{\kappa}$ m))

cycle : $\forall \kappa.\,$(Nat $\to$ Nat $\to$ Stream$^{\kappa}\,$Nat)
cycle = $\Lambda \kappa.\,$fix$^{\kappa}$($\lambda$rec.$\lambda$n.$\lambda$m. cycle$'$ rec n m)
\end{lstlisting}

% \begin{figure}[h]
% \begin{lstlisting}[mathescape]
% cycle$'$ : $\laterkappa\,$(Nat $\to$ Nat $\to$ Stream$^{\kappa}\,$Nat) $\to$ Nat $\to$ Nat $\to$ Stream$^{\kappa}$ Nat
% cycle$'$ rec  Z    m = StreamCons Z  ((rec $\tensor ^{\kappa}$ (Next$^{\kappa}$ m)) $\tensor ^{\kappa}$ (Next$^{\kappa}$ m))
% cycle$'$ rec (S n) m = StreamCons (S n) ((rec $\tensor ^{\kappa}$ (Next$^{\kappa}$ n)) $\tensor ^{\kappa}$ (Next$^{\kappa}$ m))

% cycle : $\forall \kappa.\,$(Nat $\to$ Nat $\to$ Stream$^{\kappa}\,$Nat)
% cycle = $\Lambda \kappa.\,$fix$^{\kappa}$($\lambda$rec.$\lambda$n.$\lambda$m. cycle rec n m)
% \end{lstlisting}
%   \caption{A guarded recursive definition of \texttt{cycle}, where the pattern
%     matching is handled by an auxiliary function \texttt{cycle'}. The
%     \texttt{Stream$^\kappa$ type is defined in Figure~\ref{fig:guarded_recursion_stream}.} }
%   \label{fig:cycle_guarded}
% \end{figure}

This approach works as expected, until we consider dependent types. In
particular, the problem is that we cannot have dependent quantification under
clock quantification. For example, in the function \texttt{prepend} from
Figure~\ref{fig:guarded_prepend}, the \texttt{n} in the type of \texttt{rec} is
fixed. But \texttt{n} is refined to \texttt{Z} in first clause and \texttt{S n}
in the the second clause, respectively, so \texttt{rec} can never be applied to
\texttt{xs} in a well-typed manner, since the type of xs is \texttt{Vect n a}.

\begin{figure}[h]
\begin{lstlisting}[mathescape,title=\ttBlock]
prepend$'$ : (n : Nat) $\to$ (a : Type) $\to$
           $\laterkappa\,$(Vect n a $\to$ Stream$^{\kappa}\,$a $\to$ Stream$^{\kappa}\,$a) $\to$ 
           Vect n a $\to$ Stream$^{\kappa}\,$a $\to$ Stream$^{\kappa}\,$a
prepend$'$ n a rec []        s = s 
prepend$'$ n a rec (x :: xs) s = 
                    StreamCons a x ((rec $\tensor$ (Next xs)) $\tensor$ (Next s))

prepend : (n : Nat) $\to$ (a : Type) $\to$ 
          $\forall\kappa.$(Vect n a $\to$ Stream$^\kappa$ a $\to$ Stream$^\kappa$ a)
prepend n a = $\Lambda\kappa.\,$fix$^\kappa$($\lambda$rec.$\lambda$xs.$\lambda$s. prepend$'$ n a rec xs s)
\end{lstlisting}
  \caption{A function prepending a vector on a stream. This definition is not well-typed.}
  \label{fig:guarded_prepend}
\end{figure}

To circumvent this issue, the current implementation eliminates the fixed point operator altogether according
to the elimination rules in Figure~\ref{fig:fix_elim_rules}. By eliminating the
fixed point, the type of the recursive reference is no longer fixed. This
elimination should allow us to write prepend as shown in
Figure~\ref{fig:guarded_prepend_bad}. However, as described in
Section~\ref{sec:depend-funct-types}, it does not.


% The
% elimination of the fixed point is well-typed, since \texttt{Next$^{\kappa}$ f}
% and the recursive reference, here denoted \texttt{rec}, always have the same
% type according to the guarded recursive typing rules in Figure~\ref{fig:guarded_recursion_rules_clocks}.

\begin{figure}[h]
\centering

\begin{prooftree}
\def\fCenter{\vdash}
\Axiom$\Delta,\kappa;\Gamma,f:A\fCenter \texttt{fix}^{\kappa}(\lambda rec.e) :
  A$
\UnaryInf$\Delta,\kappa;\Gamma,f:A\fCenter
  e[(\texttt{Next}^{\kappa}\,f)/rec]:A$
\end{prooftree}

\begin{prooftree}
\def\fCenter{\vdash}
\Axiom$\Delta,\kappa;\Gamma,f:\forall\kappa.A\fCenter
\Lambda\kappa.\,\texttt{fix}^{\kappa}(\lambda rec.e) : \forall\kappa.A$
\UnaryInf$\Delta,\kappa;\Gamma,f:\forall\kappa.A\fCenter \Lambda\kappa.\,e[(\texttt{Next}^{\kappa}\,f[\kappa])/rec]:\forall\kappa.A$
\end{prooftree}

% \def\fCenter{\vdash}
% \AXD{\Delta,\kappa;\Gamma,f:\forall\kappa.A\fCenter \texttt{fix}^{\kappa}(\lambda rec.e) :
%   A}
% \UID{\Delta,\kappa;\Gamma,f:\forall\kappa.A\fCenter
%   e[(Next^{\kappa}\,f)/rec]:A}
% \DisplayProof

% %\AXD{\Delta,\kappa;\Gamma,f:\forall\kappa.A\vdash \texttt{fix}^{\kappa}(\lambda \texttt{rec}.e :
% %  A}

%   \[
% \frac { \Delta ,\, \kappa ; \, \Gamma \, \vdash \, f\, =\, { fix }^{ \kappa  }(\lambda
%   rec.\, e)\, :\, A }{ \Delta ,\, \kappa \, ; \Gamma \vdash \, f\, =\, e[{ (Next^{ \kappa
%     }\,f) }/{ rec }]\, :\, A } fix_{E_1}
% \]

% \[
% \frac { \Delta \, ; \Gamma \vdash \, f\, =\, \Lambda \kappa .{ fix }^{ \kappa  }(\lambda
%   rec.\, e)\, :\, \forall \kappa .A }{ \Delta \, ; \Gamma \vdash \, f\, =\, \Lambda
%   \kappa .e[{ (Next^{ \kappa  }\, f[\kappa ]) }/{ rec }]\, :\, \forall \kappa .A
% } fix_{E_2}
% \]
  \caption{The rules for fixed point elimination.}
  \label{fig:fix_elim_rules}
\end{figure}
\todo{Is there a better way express the fix elimination rules?}

\begin{figure}[h]
\begin{lstlisting}[mathescape, title=\ttBlock]
prepend$'$ : (n : Nat) $\to$ (a : Type) $\to$
           Vect n a $\to$ Stream$^{\kappa}\,$a $\to$ Stream$^{\kappa}\,$a
prepend$'$ Z    $\ $ a []        $\,$s = s 
prepend$'$ (S k) a (x :: xs) s = 
   StreamCons a x (((Next$^{\kappa}$ ((prepend[$\kappa$]) n a)) $\tensor$ (Next xs)) $\tensor$ (Next s))

prepend : (n : Nat) $\to$ (a : Type) $\to$ 
          $\forall\kappa.$(Vect n a $\to$ Stream$^\kappa$ a $\to$ Stream$^\kappa$ a)
prepend n a = $\Lambda\kappa.\,$($\lambda$xs.$\lambda$s. prepend$'$ n a xs s)
\end{lstlisting}
  \caption{The implementation of \texttt{prepend} after fixed point elimination.}
  \label{fig:guarded_prepend_bad}
\end{figure}

% After elimination of the fixed point operator, the guarded recursive version of
% \texttt{prepend} can be defined as follows:

% \begin{lstlisting}[mathescape]
% prepend : $\forall\kappa.$(Vect n a $\to$ Stream$^\kappa$ a $\to$ Stream$^\kappa$ a)
% prepend []        s = $\Lambda\kappa.\,$s
% prepend (x :: xs) s = $\Lambda\kappa.\,$StreamCons x 
%                           (((Next$^\kappa$ prepend) $\tensor$ (Next$^\kappa$ xs)) $\tensor$ (Next$^\kappa$ s))
% \end{lstlisting}

% An elaboration upon this solution will be provided in Section~\ref{sec:impl-guard-recurs}.

\section{Scope}
To reduce the complexity of the implementation, we have narrowed the scope for
two central areas, namely the number of clock variables supported and the types
of inferred functions.

\subsection{A Singleton Clock}
\label{sec:singleton-clock}
% Argumenter:
%% Hvornår bruger man hvilket ur?
%% GR-regler for checkeren her
%%
% f = /\k. fix(\rec. e)
% f = /\k. e[(Next(f[k])/rec]

% e :
% e1 = rhs[Next(f[k])/rec]
% e2 = 
The theoretical model for guarded recursion with clock variables, due to Atkey
and McBride\,\citep{Atkey:2013}, imposes no limitations on the number of
simultaneous clock variables. This means that their system is very
expressive, but can lead to quite complicated definitions where the user must
explicitly specify which clock variable is applicable in a given situation to ensure
productivity. Automating the process of choosing the right clock variable for
any given scenario is not feasible in general, since for $n$ clock applications with $c$
clocks in scope, the number of possible combinations is $c^n$. Any linear
inference process would thus be in $O(c^n)$, and doing better would require
heuristics for common clock combinations. Presently, we have no knowledge of the
existence of such heuristics.

To avoid the problem of finding the correct combination of clock variables, we
simplify our system such that at most one clock is in scope at any given
time. The implication is that our productivity analysis will never return a
positive answer for programs whose guarded recursive equivalent requires more
than one clock. However, as noted by Clouston et
al.\,\citep{BirkedalL:guarded-lambda-conf}, all examples in the current literature
on guarded recursion require only one clock. This claim will be supported by our
evaluation in Chapter~\ref{cha:evaluation}, where we show a that significant number of
realistic programs have guarded recursive versions which require only one clock.

Incorporating this simplification means that we must adjust the typing rules for
guarded recursion accordingly. The adjusted rules are discussed later, in
Section~\ref{sec:guard-recurs-check}. While deriving this set of adjusted rules
from the original rules has not followed a specific methodology, a general trend
is:

\begin{itemize}
\item If the original rule required a specific clock in the environment, the
  adjusted rule requires an open clock.
\item Side conditions concerned with specific clocks not being free in the
  environment are adjusted so that no free clocks are allowed in the
  environment at all.
\end{itemize}

% Note that the $\kappa$, indicating which clock that applies for a given term,
% has been removed from most of the types and terms. Seeing as the rules can only
% express properties about the singleton clock, it is unnecessary to specify the
% current clock explicitly. For example, all $\later$ types now always operate on the
% same clock. However, $\kappa$ remains in some cases, such as $\forall \kappa$ and
% $\Lambda \kappa$. These still only mention the singleton clock, but the
% $\kappa$ has been kept in order to disambiguate $\forall \kappa$ from standard universal quantification and
% $\Lambda \kappa$ from standard lambda abstractions.

\subsection{Causal and Non-causal Functions}
\label{sec:causal-non-causal}
% Problem: The inferred type might not be the intended type, since more than one
% may be correct.
When defining a guarded recursive function, one has to consider the clock
quantification in its type. For example, in the definition of \texttt{cycle}
from Figure~\ref{fig:cycle_guarded}, the clock quantification is on the entire
type, whereas in the definition of \texttt{evens} from
Figure~\ref{fig:guarded_recursion_evens}, the clock quantification is on each
individual parameter type. As the number of arguments to a function is
increased, the number of different ways we could quantify over clocks in the
type of the function increases as well.

Since we cannot in general infer the correct combination of clock
quantifications for an arbitrary type, we simplify the problem by introducing
the idea of \emph{modality}. Modality is a property of a function definition,
specifying that it is in one of two disjoint classes: \emph{causal} or
\emph{non-causal}. 

\begin{definition}[\textit{Causality}]
\label{def:causality}
  Given a function $f : (x : A) \to B$ and an element $x : A$, where $B$ is a
  coinductive type, $f$ is \emph{causal} if the computation of the first $n$
  unfoldings of $f x$ only depends on the result of the first $n$ unfoldings of $x$.
\end{definition}

If a function is causal, the clock quantification is over the
outermost type, since this means that all function arguments operate on the same
clock. Definition~\ref{def:causality}. Informally, the
output of a causal function at time $n$ can only depend on the input at time
$n$. If a function is non-causal, the quantification is on each individual
guarded type, since the output of a non-causal function may depend arbitrarily
on its input. Therefore, the definition of \texttt{evens} in
Figure~\ref{fig:guarded_recursion_evens} is non-causal.



While inferring the modality of a function is possible, simply by trying all
possible combinations of clock quantification on its type, some functions may
have both a causal and a non-causal guarded recursive version. Since recursive
references cannot be given as input to non-causal functions, the semantics of
the two versions differ significantly. Instead of attempting to infer the
modality, we introduce the keyword \texttt{causal} as a function option, such
that it must be added to any causal definition. Function definitions which are
not marked as causal are assumed to be non-causal. This means that the
\texttt{cycle} function from Section~\ref{sec:fixkappa-rule} must be given as
follows:
\begin{lstlisting}[mathescape,title=\idrisBlock]
causal 
cycle : Nat $\rightarrow$ Nat $\rightarrow$ Stream Nat
cycle Z     m = Z :: (cycle m m)
cycle (S n) m = (S n) :: (cycle n m)
\end{lstlisting}


% \section{Implementation}
% The inference of guarded recursive versions of productive programs happens in
% two steps: (1) the inference system infers a guarded recursive TT definition
% from an elaborated TT definition, and (2) the checking system verifies that the
% inferred guarded recursive definition is well-formed according to the typing
% rules in Section~\ref{sec:guard-recurs-check}.. The two systems are implemented
% independently, and the checking system does not rely on any
% implementation-specific details of the inference system. Nevertheless, they do
% agree on the initial inference environment in which a guarded recursive definition should
% hold. Inference environments will be introduced in Section~\ref{sec:inference-system}.

\section{Preparing Idris}
% Preprocessing, how are guarded data types created?
\begin{figure}[h]
\begin{lstlisting}[mathescape,title=\idrisBlock]
namespace GuardedRecursion
  ||| A computation that is available later (''tomorrow'').
  data Later$'$ : Type $\to$ Type where
    Next : {a : Type} $\to$ a $\to$ Later$'$ a

  ||| A universal quantification over clocks.
  |||
  ||| Since the implementation uses only one clock,
  ||| no clock needs to be specified.
  data Forall : Type $\to$ Type where
    LambdaKappa : {a : Type} $\to$ a $\to$ Forall a
 
  ||| Applies the singleton clock to a value of a
  ||| universally quantified type.
  apply : {a : Type} $\to$ Forall a $\to$ a
  apply (LambdaKappa a) = a 

  ||| Specifies the time at which a computation is available.
  data Availability = Now | Tomorrow Availability

  ||| A computation that is available arbitrarily later.
  Later : Availability $\to$ Type $\to$ Type
  Later Now a = a
  Later (Tomorrow n) a = Later$'$ (Later n a)
  
  ||| Composition of two values that are available later.                             
  compose : {a : Type} $\to$ {b : Type} $\to$ 
            {n : Availability} $\to$ 
            Later (Tomorrow n) (a $\to$ b) $\to$ 
            Later (Tomorrow n) a $\to$ 
            Later (Tomorrow n) b
  compose {n = Now} t u = compose$'$ t u
    where
     compose$'$ : {a, b : Type} $\to$ Later$'$ (a $\to$ b) $\to$ Later$'$ a $\to$ Later$'$ b
     compose$'$ (Next t) (Next u) = Next (t u)
  compose {n = Tomorrow n$'$} (Next t) (Next u) = 
                                           Next (compose {n = n$'$} t u)
\end{lstlisting}
  \caption{The guarded recursion primitives in Builtins.idr}
  \label{fig:guarded_recursion_primitives}
\end{figure}

Before being able to infer guarded recursive terms, we must first define what it
means for a term to be guarded recursive within Idris. To this end, we have
added the definitions in Figure~\ref{fig:guarded_recursion_primitives} as
built-in primitives in Idris. These include a fundamental type \texttt{Later'},
modeling the later ($\laterkappa$) operator, and a type \texttt{Forall},
modeling clock quantification ($\forall\kappa$). Accordingly, \texttt{apply}
models clock application. We introduce the idea of \texttt{Availability} to
model the ``lateness'' of a given type. Concretely, this idea is used for later
composition with \texttt{compose}, in order to support later application under
arbitrarily many applications of \texttt{Later'}. We have delibrately chosen to
add these as built-in primitives, and not as a part of the standard library,
since totality checking is a fundamental feature of the compiler which should
not require a that a specific library is in scope.

Thus, a guarded recursive term is defined as a term which consists of one or more of
these added primitives. However, a well-typed guarded recursive term is not
necessarily productive, since the primitives do not model any of the side
conditions from the rules in Figure~\ref{fig:gr_rules_sin_clock}. Therefore, the
purpose of the inference system is to infer guarded recursive terms, while the
purpose of the checking system is to make sure that the guarded recursive terms
are actually productive.

% \paragraph{Guarded Recursion Library}
% The first step to adding guarded recursion to Idris is to add the guarded
% recursive types and functions. We have added these to the Idris built-in
% library, and not as a part of the prelude. This is because the user should be
% able to check functions for productivity using our guarded recursion checker
% without having to rely on the standard library.  While most of the
% implementations are straight forward, and can be found in Appendix\todo{Add ref
%   to appendix. And add appendix!}, there are interesting parts to discuss. We
% have added the notion of \emph{how much later} something is, through the idea of
% \texttt{Availability}, seen in Figure~\ref{fig:availability}. This is, in
% conjunction with \texttt{Later} done so that we can many $\later$ applications
% as a single \texttt{Later} applications.

% \begin{figure}[h]
%   \begin{lstlisting}[mathescape]
% namespace GuardedRecursion
%   ||| A computation that is available later.
%   data Later$'$ : Type $\to$ Type where
%     Next : {a : Type} $\to$ a $\to$ Later$'$ a

%   ||| A universal quantification over clocks
%   |||
%   ||| Since the implementation uses only one clock,
%   ||| no clock needs to be specified
%   data Forall : Type $\to$ Type where
%     LambdaKappa : {a : Type} $\to$ a $\to$ Forall a
 
%   ||| Applies the singleton clock to a  universally
%   ||| quantified value.
%   apply : {a : Type} $\to$ Forall a $\to$ a
%   apply (LambdaKappa a) = a 

%   ||| Specifies the time at which a computation is available
%   data Availability = Now | Tomorrow Availability

%   ||| A computation that is available arbitrarily later.
%   Later : Availability $\to$ Type $\to$ Type
%   Later Now a = a
%   Later (Tomorrow n) a = Later$'$ (Later n a)
  
%   laterDist : Later$'$ (a $\to$ b) $\to$ (Later$'$ a $\to$ Later$'$ b)
%   laterDist (Next f) = \a => case a of
%                                (Next a$'$) => Next (f a$'$)
  
%   ||| Composition of two values that are available later.                             
%   compose : {a, b : Type} $\to$ 
%             {n : Availability} $\to$ 
%             Later (Tomorrow n) (a $\to$ b) $\to$ 
%             Later (Tomorrow n) a $\to$ 
%             Later (Tomorrow n) b
%   compose {n = Now} t u = compose$'$ t u
%     where
%      compose$'$ : {a, b : Type} $\to$ Later$'$ (a $\to$ b) $\to$ Later$'$ a $\to$ Later$'$ b
%      compose$'$ (Next t) (Next u) = Next (t u)
%   compose {n = Tomorrow n$'$} (Next t) (Next u) = Next (compose {n = n$'$} t u)
% \end{lstlisting}
%   \caption{The guarded recursion primitives in Builtins.idr}
%   \label{fig:guarded_recursion_primitives_old}
% \end{figure}

\subsection{Guarded Names}
Seeing as we want to infer guarded recursive versions of input programs, while not
overwriting the names of the original names of the input programs, we have created a system for
generating new names for guarded recursive definitions. Although it is fairly
trivial, it is an important part of our system. During compilation, we build a
map from the original name for a definition, $n$, to its generated guarded recursive name,
$_gn$. If the original name is the name of a record projection function, an
additional name $_{\forall}n$ is generated for its universally quantified
projection. 

Henceforth, we will refer to the name of an original definition as
an \emph{Idris name} and its guarded recursive name as its \emph{guarded name}. For
record projections, the name for the universally quantified projection will be
referred to as the \emph{quantified name}. 

% As we want to infer new types and terms, and not override the existing ones, we
% need a system for creating a guarded name from an existing user written
% name. While this is fairly trivial, it is an important part of our system. We
% will hence forth refer to these as the \emph{guarded names} as oppose to their
% original \emph{Idris names}. We will mark these guarded names with a subscript
% $g$, such that a name $Name$ becomes $_gName$.

% During compilation we keep a map from Idris names to their guarded names. This
% means that we can for any Idris name find their guarded name, and vice
% versa. This comes in handy when we infer the guarded terms.

\subsection{Guarded Types}
\label{sec:guarded-types}
Guarded types are guarded recursive versions of coinductive data types. During
elaboration, a guarded type is constructed for each coinductive data definition,
and then elaborated separately. First, guarded names are invented for the type constructor and
each of the data constructors. Then, the later type constructor ($\laterkappa$)
is applied to the type of each recursive reference in each data
constructor. The type of the type constructor remains unchanged. An example of a
coinductive type and its guarded equivalent is shown in
Figure~\ref{fig:guarded_stream_inf}. Note that the guarded type \texttt{$_g$Stream} is not coinductive, but
inductive, since \texttt{$_g$Stream} is a positively finite encoding of the
possibly infinite type \texttt{Stream}.

\begin{figure}[h]
\begin{lstlisting}[mathescape,title=\idrisBlock]
codata Stream : Type -> Type where
  (::) : a -> Stream a -> Stream a
\end{lstlisting}
\begin{lstlisting}[mathescape,title=\idrisBlock]
data $_g$Stream : Type -> Type where
  ($_g$::) : a -> $\laterkappa$$ _g$Stream a -> $_g$Stream a
\end{lstlisting}
  \caption{Above: A coinductive type definition. Below: Its guarded equivalent.}
  \label{fig:guarded_stream_inf}
\end{figure}

The inferred applications of the later type constructor ($\laterkappa$, represented
as \texttt{Later'} in Figure~\ref{fig:guarded_recursion_primitives}) on
recursive references matches the existing encoding for lazy evaluation, where
recursive references in coinductive data constructors are guarded by the
\texttt{Inf} type constructor (see
Section~\ref{sec:coind-data-types}). Nevertheless, we have chosen to create a
separate encoding of coinductive types for guarded recursion, because the
semantics of \texttt{Inf} and later differ. The \texttt{Inf} constructor
indicates that the data must be evaluated lazily, while the later constructor
indicates availability. As such, the reading of
$\laterkappa\laterkappa _{g}Stream$ is straightforward, namely that any value of
this type is available two steps from now (i.e. ``the day after tomorrow''),
whereas \texttt{Inf (Inf Stream)} is more obscure, saying that any value of this
type may unfold infinitely unless it is delayed twice. Be that as it may, the
later constructor could be encoded with \texttt{Inf}, if all of its primitives
were ported to \texttt{Inf} as well.

\begin{figure}[h]
\begin{lstlisting}[mathescape,title=\idrisBlock]
corecord Tree a where
  left  : Tree a $\to$ Tree a
  value : Tree a $\to$ a
  right : Tree a $\to$ Tree a
\end{lstlisting}
\begin{lstlisting}[mathescape,title=\idrisBlock]
record $_g$Tree a where
  $_g$left  $\;$: $_g$Tree a $\to$ $\later_g$Tree a
  $_g$value : $_g$Tree a $\to$ a
  $_g$right : $_g$Tree a $\to$ $\later_g$Tree a
\end{lstlisting}
  \caption{Above: A coinductive record definition. Below: Its guarded equivalent.}
  \label{fig:guarded_tree_inf}
\end{figure}

Guarded types are also created for corecords. Additionally, quantified
projections are generated, which preserve clock quantification under
projections. Consider the example of the coinductive record type \texttt{Tree}
in Figure~\ref{fig:guarded_tree_inf}. For the guarded record type $_{g}Tree$, the
quantified projections defined in Figure~\ref{fig:tree_quantified_projections}
are virtually generated. In order for them to be well-typed, we use the type
isomorphism in Figure~\ref{fig:quantified_later_iso} introduced by
Møgelberg\,\citep{Mogelberg:2014} for eliminating later constructors in a
quantified context. When we say that these quantified projections are
\emph{virtually} generated, it means that only their types are generated and
added to the typing context as Idris postulates. Hence, they do not reduce
during type checking. 

\begin{figure}[h]
\begin{lstlisting}[mathescape,title=\idrisBlock]
$_{\forall}$left$\ \;$ : $\forall\kappa.\,$$_g$Stream a $\,\to\,$ $\forall\kappa.\,$$_g$Stream a
$_{\forall}$left   s = iso $\Lambda\kappa.$ $\,$($_{g}$left$\,$(s[$\kappa$]))
$_{\forall}$value : $\forall\kappa.\,$$_g$Stream a $\,\to\,$ a
$_{\forall}$value s = $_{g}$value$\,$(s[$\kappa$])
$_{\forall}$right : $\forall\kappa.\,$$_g$Stream a $\,\to\,$ $\forall\kappa.\,$$_g$Stream a
$_{\forall}$right s = iso $\Lambda\kappa.$$\,$($_{g}$right$\,$(s[$\kappa$]))
\end{lstlisting}
  \caption{The quantified projections generated for $_{g}Tree$.}
\label{fig:tree_quantified_projections}
\end{figure}

\begin{figure}[h]
\[
iso = \forall \kappa .A\cong \forall\kappa.\laterkappa A
\]
\begin{lstlisting}[mathescape,title=\idrisBlock]
postulate iso : $\forall\kappa.\,\laterkappa$a $\to$ $\forall\kappa.$a
\end{lstlisting}
  \caption{A type isomorphism for removing later types in a quantified context, and
    its right-to-left Idris implementation.}
\label{fig:quantified_later_iso}
\end{figure}

% \paragraph{Clocked Types}
% These are made by inferring a new data declaration for any codata or corecord
% declaration. This new data declaration is simply constructed during the
% elaboration of the original and then elaborated by itself. This means that for
% any user written codata or corecord declaration a guarded version is an
% intrinsic part of the program. 

% A such guarded version of a declaration is fairly straight forward. Guarded
% names are given to the type and to all constructors. Any recursive reference in
% the type of the constructor is placed under a $\later$-type, in a similar
% fashion to how Idris already handles lazy evaluation. The type constructor
% remains unchanged. An example of this inferences for a codata declaration can be
% seen in Figure~\ref{fig:guarded_stream_inf}. Note that the inferred declaration
% is a data declaration, not codata. This is because we do not need both Idris's
% built in laziness (\texttt{Inf}), and the guarded recursion laziness ($\later$)
% \todo{Explain why these are not the same}.

% \begin{figure}[h]
% \begin{lstlisting}[mathescape]
% codata Stream : Type -> Type where
%   MkStream : a -> Stream a -> Stream a

% data $_g$Stream : Type -> Type where
%   $_g$MkStream : a -> $\later_g$Stream a -> $_g$Stream a
% \end{lstlisting}
%   \caption{Inference of the guarded stream declaration.}
%   \label{fig:guarded_stream_inf}
% \end{figure}

% In the corecord case, we take a similar approach. However, instead of
% constructors we have to guard projections. Again, any recursive projection is
% guarded such that $A \rightarrow A$ becomes $A \rightarrow \later A$, and all
% projections are given a guarded name. Figure~\ref{fig:guarded_tree_inf} shows an
% example of such an inference.

% \begin{figure}[h]
% \begin{lstlisting}[mathescape]
% corecord Tree a where
%   left : Tree a -> Tree a
%   value : Tree a -> a
%   right : Tree a -> Tree a

% record $_g$Tree a where
%   $_g$left : $_g$Tree a -> $\later_g$Tree a
%   $_g$value : $_g$Tree a -> a
%   $_g$right : $_g$Tree a -> $\later_g$Tree a
% \end{lstlisting}
%   \caption{Inference of the guarded tree declaration.}
%   \label{fig:guarded_tree_inf}
% \end{figure}

% Furthermore, for each projection we also generate what we call a \emph{forall
%   projection}, henceforth denoted as $_\forall Name$. We use this for
% projections on quantified types. Consider a coinductive type $A$ with a
% projection $p$ of type $A \rightarrow A$, and a variable $x$ of type $\forall
% \kappa.A$. In order to perform $p$ on $x$ and maintain the quantification in the
% type, we must first apply the clock to $x$, apply $_gp$, and then abstract over the
% clock again using $\Lambda \kappa$. This gives us:

% \[
% \Lambda \kappa . _gp (apply\,x) : \forall \kappa . \later A
% \]

% The type of the above is isomorphic with $\forall \kappa . A$ according to the
% isomorphism described by Rasmus M\o gelberg\,\cite{Mogelberg:2014} seen in
% Figure~\ref{fig:quantified_later_iso}. As a short for all of this we simply use
% $_\forall p$ which has type $\forall A \rightarrow \forall A$, giving us the
% same result. As such these \emph{forall projections} do not add anything new,
% they are just a simpler way for us to perform projections on terms of quantified
% type. The intuition is that because of Figure~\ref{fig:quantified_later_iso} we
% can perform projections on quantified types without having to worrying about
% their lateness. Because they are quantified over clocks, they are always available.

% \begin{figure}[h]
% \[
% \forall \kappa .A\cong \forall \kappa .\rhd ^\kappa A
% \]
%   \caption{A type isomorphism describe by M\o gelberg\,\cite{Mogelberg:2014}.}
%   \label{fig:quantified_later_iso}
% \end{figure}

%\section{Inference of Guarded Recursive Definitions}

\section{Preprocessing Guarded Recursive Definitions}
\label{sec:impl-guard-recurs}
% Meta
%% Samspil mellem inferens og check
%% Hvordan behandles causal vs. non-causal
We have now defined that a guarded recursive term is a term which consists of one or
more of the built-in guarded recursive primitives. Also, we have established
that guarded recursive type definitions are created for any non-dependent
coinductive type definition, and guarded names are created for definitions with
a coinductive result type. Now, we reach the first step of the inference
process, namely preprocessing. Directed by the left-hand sides of a user-written
definition, the preprocessing step provides a setup for the inference of
right-hand side terms, which will be the subject of
Section~\ref{sec:inference-system}. As explained in Section~\ref{sec:causal-non-causal},
the system distinguishes between causal and non-causal definitions. Causal
and non-causal definitions are preprocessed differently, although the steps are
the same:

% Before
% making an attempt to infer a guarded recursive version of a definition, the
% definition is first preprocessed in order to ease the inference process. 

\begin{enumerate}
\item Create auxiliary functions to circumvent the limitations of the
  \texttt{fix$^\kappa$} rule.
\item Lift parameters out of the fixed point.
\item Eliminate the fixed point by the fixed point elimination rules from
  Figure~\ref{fig:fix_elim_rules}.
\item Eliminate the recursive argument in the auxiliary function(s).
\end{enumerate}

% \newcommand{\prepend}[5]{\ensuremath{{\boxed{#1;\causal;\phi,(#2,#3);\open;\Gamma\,\vdash\,\text{#4}\,:\,\text{#5}\,\infer
%         \text{?}\,:\,_{g}\text{#5})}}}}
% \newcommand{\prependm}[5]{\ensuremath{{\boxed{
%             \begin{matrix*}[l] #1;\causal;\phi,(#2,#3);\open;\Gamma\,\vdash \\
%                            \quad\text{#4}\,:\,\text{#5}\,\infer\,\text{?}\,:\,_{g}\text{#5})
%             \end{matrix*}
%             }}}}
\newcommand{\prepend}[5]{\ensuremath{{\boxed{IE\,\vdash\,\text{#4}\,:\,\text{#5}\,\infer
        \text{?}\,:\,_{g}\text{#5})}}}}
\newcommand{\prependm}[5]{\ensuremath{{\boxed{
            \begin{matrix*}[l] IE\,\vdash \\
                           \quad\text{#4}\,:\,\text{#5}\,\infer\,\text{?}\,:\,_{g}\text{#5})
            \end{matrix*}
            }}}}
\newcommand{\changed}[1]{\ensuremath{\fcolorbox{gray}{light-gray}{a rec}}}

On the right-hand side of the preprocessed definitions, a box will be shown
which indicates the starting point for the inference system presented in
Section~\ref{sec:inference-system},
e.g. \prepend{\text{prepend}}{\text{prepend}}{\,_{g}\text{prepend}}{s}{Stream
  a}. In particular, this example means that in a given \emph{inference
  environment}, $IE$, where \texttt{s} has type \texttt{Stream a}, we must infer
an as yet unknown term of type \texttt{$_g$Stream a}. Inference environments will
be defined along with the rest of the system in
Section~\ref{sec:inference-system}. 

% For neither the causal or the non-causal case, all the steps presented in the following are
% actually performed. Instead, we provide the setup for the inference system by
% proceeding directly to the last step. We show all the steps here in order to
% demonstrate how we have arrived
In the following, we demonstrate the theoretical steps taken in order to provide
a setup for the inference system. In practice, we proceed directly to the last
step for both the causal and the non-causal case. The intermediate steps are
merely shown here to justify our approach.

\subsection{Preprocessing Causal Functions}
\label{sec:prepr-caus-funct}
 We imagine that a user has written the following causal function
 \texttt{prepend} as a (concrete) Idris program:
\begin{lstlisting}[mathescape, title=\idrisBlock]
  total causal
  prepend : (a : Type) $\to$ List a $\to$ Stream a $\to$ Stream a
  prepend a []        s = s
  prepend a (x :: xs) s = x :: (prepend a xs s)
\end{lstlisting}
 From this point on, no user intervention is necessary.  (1) To infer a guarded
 recursive version of \texttt{$_g$prepend}, we first split the definition in
 two, such that pattern matching happens in an auxiliary function,
 \texttt{$_g$prepend$'$}, which is then called inside the fixed point on the
 right-hand side of \texttt{$_g$prepend}:

% \newcommand{\prepend}[4]{\ensuremath{{\boxed{\text{#1};\causal;\phi,\text{#2};\open;\Gamma\,\vdash\,\text{#3}\,:\,\text{#4}\,\infer
%         \\\text{?}\,:\,_{g}\text{#4})}}}}}
\begin{lstlisting}[mathescape, title=\ttBlock]
  $_g$prepend$'$ : $\laterkappa$((a$\,$:$\,$Type) $\to$ List a $\to$ $_{g}$Stream a $\to$ $_{g}$Stream a) $\to$ 
                (a$\,$:$\,$Type) $\to$ List a $\to$ $_{g}$Stream a $\to$ $_{g}$Stream a
  $_g$prepend$'$ rec a []        $\;$s = 
     $\prepend{\text{prepend}}{\text{prepend}}{\,_{g}\text{prepend}}{s}{Stream a}$
  $_g$prepend$'$ rec a (x :: xs) s = 
     $\prepend{\text{prepend}}{\text{prepend}}{\,_{g}\text{prepend}}{x :: (prepend a xs s)}{Stream a}$

  $_g$prepend : $\forall\kappa.$ (a$\,$:$\,$Type) $\to$ List a $\to$ $_{g}$Stream a $\to$ $_{g}$Stream a
  $_g$prepend = $\Lambda\kappa.\,$fix$^\kappa$($\lambda$rec.$\lambda$a.$\lambda$xs.$\lambda$s.$\,$$_g$prepend$'\;$rec a xs s)
\end{lstlisting}
As \texttt{prepend} is now defined with a fixed point, the recursive reference
\texttt{rec} is inserted in place of the original recursive reference.

(2) The type parameter \texttt{a} is lifted out of the fixed point to make up
for the lack of type substitution in the later application rule, as explained in
Section~\ref{sec:handling-parameters}. Note that this also lifts \texttt{a} out
of the type of the recursive reference, \texttt{rec}. Furthermore, \texttt{a} is
no longer bound under the clock quantification in the type of
\texttt{$_g$prepend}. After the lifting of \texttt{a}, we arrive at the following
(changes are highlighted in grey boxes):
\begin{lstlisting}[mathescape, title=\ttBlock]
  $_g$prepend$'$ : $\fcolorbox{gray}{light-gray}{(a\,:\,Type)}$ $\to$ $\fcolorbox{gray}{light-gray}{\ensuremath{\laterkappa}(List a \ensuremath{\to} \ensuremath{_{g}}Stream a \ensuremath{\to} \ensuremath{_{g}}Stream a)}$ $\to$ 
               List a $\to$ $_{g}$Stream a $\to$ $_{g}$Stream a
  $_g$prepend$'$ $\fcolorbox{gray}{light-gray}{a rec}$ []        s =
   $\prepend{\text{(prepend a)}}{\text{(prepend a)}}{\,(_{g}\text{prepend a})}{s}{Stream a}$
  $_g$prepend$'$ $\fcolorbox{gray}{light-gray}{a rec}$ (x :: xs) s =
   $\prepend{\text{(prepend a)}}{\text{(prepend a)}}{\,(_{g}\text{prepend a})}{x :: (prepend a xs s)}{Stream a}$

  $_g$prepend : $\fcolorbox{gray}{light-gray}{(a\,:\,Type)}$ $\to$ ($\forall\kappa.\,$ List a $\to$ $_{g}$Stream a $\to$ $_{g}$Stream a)
  $_g$prepend $\fcolorbox{gray}{light-gray}{a}$ = $\Lambda\kappa.\,$fix$^\kappa$($\lambda$rec.$\lambda$xs.$\lambda$s.$\,$prepend$'$ $\fcolorbox{gray}{light-gray}{a rec}$ xs s)
\end{lstlisting}
(3) The fixed point is eliminated by replacing all references to
\texttt{rec} with \texttt{Next~(($_g$prepend~a)~[$\kappa$])} in \texttt{$_g$prepend}, since this term has
exactly the same type as \texttt{rec} in step 2.
\begin{lstlisting}[mathescape, title=\ttBlock]
  $_g$prepend$'$ : (a$\,$:$\,$Type) $\to$ $\laterkappa$(List a $\to$ $_{g}$Stream a $\to$ $_{g}$Stream a) $\to$ 
               List a $\to$ $_{g}$Stream a $\to$ $_{g}$Stream a
  $_g$prepend$'$ a rec []        s =
    $\prepend{\text{(prepend a)}}{\text{(prepend a)}}{\,(_{g}\text{prepend a})}{s}{Stream a}$
  $_g$prepend$'$ a rec (x :: xs) s =
    $\prepend{\text{(prepend a)}}{\text{(prepend a)}}{\,(_{g}\text{prepend a})}{x :: (prepend a xs s)}{Stream a}$

  $_g$prepend : (a$\,$:$\,$Type) $\to$ ($\forall\kappa.\,$ List a $\to$ $_{g}$Stream a $\to$ $_{g}$Stream a)
  $_g$prepend a = $\Lambda\kappa.\,$$\lambda$xs.$\lambda$s.$\,$prepend$'$ a $\fcolorbox{gray}{light-gray}{(Next ((prepend a)[\ensuremath{\kappa}]))}$ xs s
\end{lstlisting}
(4) The \texttt{rec} argument to \texttt{$_g$prepend$'$} is eliminated, but will
be recovered by the inference system when the rule for causal recursive
references is used (see Figure~\ref{fig:epsilon_rec_causal}o):
\begin{lstlisting}[mathescape,title=\ttBlock]
  $_g$prepend$'$ : $\fcolorbox{gray}{light-gray}{(a : Type) \ensuremath{\to} List a \ensuremath{\to\,_{g}}Stream a \ensuremath{\to\,_{g}}Stream a}$
  $_g$prepend$'$ a []        s =
    $\prepend{\text{(prepend a)}}{\text{(prepend a)}}{\,(_{g}\text{prepend a})}{s}{Stream a}$
  $_g$prepend$'$ a (x :: xs) s =
    $\prepend{\text{(prepend a)}}{\text{(prepend a)}}{\,(_{g}\text{prepend a})}{x :: (prepend a xs s)}{Stream a}$

  $_g$prepend : (a$\,$:$\,$Type) $\to$ ($\forall\kappa.\,$ List a $\to$ $_{g}$Stream a $\to$ $_{g}$Stream a)
  $_g$prepend a = $\Lambda\kappa.\,$$\lambda$xs.$\lambda$s.$\,$prepend$'$ a xs s
\end{lstlisting}
After preprocessing, the inference system will attempt to build well-typed
guarded recursive right-hand sides as specified in the boxes.

% \newcommand{\evens}[5]{\ensuremath{\boxed{#1;\noncausal;\phi,(#2,#3);\open;\Gamma\,\vdash\,\text{#4}\,:\,\text{#5}\,\infer
%         \text{?}\,:\,\forall\kappa\,_{g}\text{#5})}}}
% \newcommand{\evensm}[5]{\ensuremath{\boxed{
%             \begin{matrix*}[l] #1;\causal;\phi,(#2,#3);\open;\Gamma\,\vdash \\
%                            \quad\text{#4}\,:\,\text{#5}\,\infer\,\text{?}\,:\,\forall\kappa.\,_{g}\text{#5})
%             \end{matrix*}
%             }}}
\newcommand{\evens}[5]{\ensuremath{\boxed{IE\,\vdash\,\text{#4}\,:\,\text{#5}\,\infer
        \text{?}\,:\,\forall\kappa.\,_{g}\text{#5})}}}
\newcommand{\evensm}[5]{\ensuremath{\boxed{
            \begin{matrix*}[l] IE\,\vdash \\
                           \quad\text{#4}\,:\,\text{#5}\,\infer\,\text{?}\,:\,\forall\kappa.\,_{g}\text{#5})
            \end{matrix*}
            }}}

\subsection{Preprocessing Non-causal Functions}
Recall the definition of the indexed fixed point for guarded recursion:
\begin{center}
  \AXD{\Delta;\Gamma,x:A\to \laterkappa B\vdash t:A\to B}
\UID{\Delta;\Gamma\vdash \mathit{pfix}^{\kappa}x.t:A\to B}
\end{prooftree}
\begin{center}
$\mathit{pfix}^{\kappa}x.t = \mathit{fix}^{\kappa}\,y:\laterkappa(A\to
B).t[\lambda a.y\tensor^{\kappa}next^{\kappa}(a)/x]$
\end{center}
%%% Local Variables:
%%% mode: latex
%%% TeX-master: "../copatterns-thesis"
%%% End:

\end{center}
Non-causal definitions are preprocessed slightly differently, mainly because the
recursive reference has a different form on account of the use of the indexed
fixed point. Consider the following non-causal Idris definition \texttt{evens},
provided by a user:
\begin{lstlisting}[mathescape, title=\idrisBlock]
evens : (a : Type) $\to$ Stream a $\to$ Stream a
evens a s = head s :: evens a (tail (tail s))
\end{lstlisting}
(1) Initially, two auxiliary functions are created, \texttt{$_{g}$evens$'$} and
\texttt{$_{g}$evens$''$}. The \texttt{$_{g}$evens$'$} function is necessary because its type
enables us to use the indexed fixed point, \texttt{pfix$^\kappa$}, to provide a recursive reference
without clock quantification in its result type.
\begin{lstlisting}[mathescape, title=\ttBlock]
$_g$evens$''$ : (($\forall\kappa.\,$(a$\,$:$\,$Type)) $\to$ ($\forall\kappa.\,$$_{g}$Stream a) $\to$ ($\laterkappa$$_{g}$Stream a)) $\to$ 
           ($\forall\kappa.\,$(a$\,$:$\,$Type)) $\to$ ($\forall\kappa.\,$$_{g}$Stream a) $\to$ $_{g}$Stream a
$_g$evens$''$ rec a s = 
          $\evens{\text{evens}}{\text{evens}}{\,_{g}\text{evens}}{head s :: evens a (tail (tail s))}{Stream a}$

$_g$evens$'$ : ($\forall\kappa.\,$(a$\,$:$\,$Type)) $\to$ ($\forall\kappa.\,$$_{g}$Stream a) $\to$ $_{g}$Stream a
$_g$evens$'$ = pfix$^\kappa$($\lambda$rec.$\lambda$a.$\lambda$s. evens$''$ rec s)

$_g$evens : ($\forall\kappa.\,$(a$\,$:$\,$Type)) $\to$ ($\forall\kappa.\,$$_{g}$Stream a) $\to$ ($\forall\kappa.\,$$_{g}$Stream a)
$_g$evens = $\Lambda\kappa.\,$evens$'$
\end{lstlisting}
(2) Exactly as for the causal case, the type parameter, \texttt{a}, is lifted out of the fixed point.
\begin{lstlisting}[mathescape, title=\ttBlock]
$_g$evens$''$ : $\fcolorbox{gray}{light-gray}{(\ensuremath{\forall\kappa.\,}(a :
Type))}$ $\to$ $\fcolorbox{gray}{light-gray}{((\ensuremath{\forall\kappa.\,_{g}}Stream a)
\ensuremath{\,\to} (\ensuremath{\laterkappa\, _{g}}Stream a))}$ $\to$ 
           $\forall\kappa.\,$$_{g}$Stream a $\to$ $_{g}$Stream a
$_g$evens$''$ $\fcolorbox{gray}{light-gray}{a rec}$ s =
          $\evens{\text{(evens a)}}{\text{(evens a)}}{\,(_{g}\text{evens a})}{head s :: evens a (tail (tail s))}{Stream a}$

$_g$evens$'$ : ($\forall\kappa.\,$(a$\,$:$\,$Type)) $\to$ ($\forall\kappa.\,$$_{g}$Stream a) $\to$ $_{g}$Stream a
$_g$evens$'$ $\fcolorbox{gray}{light-gray}{a}$ = pfix$^\kappa$($\lambda$rec.$\lambda$s. evens$''$ $\fcolorbox{gray}{light-gray}{a}$ rec s)

$_g$evens : ($\forall\kappa.\,$(a$\,$:$\,$Type)) $\to$ ($\forall\kappa.\,$$_{g}$Stream a) $\to$ ($\forall\kappa.\,$$_{g}$Stream a)
$_g$evens $\fcolorbox{gray}{light-gray}{a}$ = $\Lambda\kappa.\,$evens$'$ $\fcolorbox{gray}{light-gray}{a}$
\end{lstlisting}
(3.1) The indexed fixed point is replaced by the equivalent ordinary guarded
recursive fixed point. This step requires that the recursive reference is
placed inside a function where the \texttt{Next} rule is used on its
argument, \texttt{x}. Also, using \texttt{fix$^\kappa$} instead of
\texttt{pfix$^\kappa$} means that the type of the recursive reference,
\texttt{rec}, has now changed.
\begin{lstlisting}[mathescape, title=\ttBlock]
$_g$evens$''$ : ($\forall\kappa.\,$(a$\,$:$\,$Type)) $\to$ (($\forall\kappa.\,$$_{g}$Stream a) $\to$ ($\laterkappa$$_{g}$Stream a)) $\to$ 
          ($\forall\kappa.\,$$_{g}$Stream a) $\to$ $_{g}$Stream a
$_g$evens$''$ a rec s =
         $\evens{\text{(evens a)}}{\text{(evens a)}}{\,(_{g}\text{evens a})}{head s :: evens a (tail (tail s))}{Stream a}$

$_g$evens$'$ : ($\forall\kappa.\,$(a$\,$:$\,$Type)) $\to$ ($\forall\kappa.\,$$_{g}$Stream a) $\to$ $_{g}$Stream a
$_g$evens$'$ a = $\fcolorbox{gray}{light-gray}{fix\ensuremath{^\kappa}}$($\lambda$rec.$\lambda$s. evens$''$$\;$a $\fcolorbox{gray}{light-gray}{(\ensuremath{\lambda}x. rec \ensuremath{\tensor^\kappa} (Next x))}$ s)

$_g$evens : ($\forall\kappa.\,$(a$\,$:$\,$Type)) $\to$ ($\forall\kappa.\,$$_{g}$Stream a) $\to$ ($\forall\kappa.\,$$_{g}$Stream a)
$_g$evens a = $\Lambda\kappa.\,$evens$'$ a
\end{lstlisting}
(3.2) The fixed point is eliminated according to the rules in Figure~\ref{fig:fix_elim_rules}.
\begin{lstlisting}[mathescape, title=\ttBlock]
$_g$evens$''$ : ($\forall\kappa.\,$(a$\,$:$\,$Type)) $\to$ (($\forall\kappa.\,$$_{g}$Stream a) $\to$ ($\laterkappa$$_{g}$Stream a)) $\to$ 
           ($\forall\kappa.\,$$_{g}$Stream a) $\to$ $_{g}$Stream a

$_g$evens$''$ a rec s =
         $\evens{\text{(evens a)}}{\text{(evens a)}}{\,(_{g}\text{evens a})}{head s :: evens a (tail (tail s))}{Stream a}$

$_g$evens$'$ : ($\forall\kappa.\,$(a$\,$:$\,$Type)) $\to$ ($\forall\kappa.\,$$_{g}$Stream a) $\to$ $_{g}$Stream a
$_g$evens$'$ a = $\lambda$s. evens$''$ a ($\lambda$x.$\,$$\fcolorbox{gray}{light-gray}{Next (evens\ensuremath{'} a)}$ $\tensor^\kappa$ (Next x)) s

$_g$evens : ($\forall\kappa.\,$(a$\,$:$\,$Type)) $\to$ ($\forall\kappa.\,$$_{g}$Stream a) $\to$ ($\forall\kappa.\,$$_{g}$Stream a)
$_g$evens a = $\Lambda\kappa.\,$evens$'$ a
\end{lstlisting}
(4) The \texttt{rec} argument to \texttt{$_{g}$evens$''$} is eliminated, but will
be recovered by the inference system when the rules for non-causal recursive
references is used (see Figure~\ref{fig:epsilon_rec_causal}).
\begin{lstlisting}[mathescape, title=\ttBlock]
$_g$evens$''$ : $\fcolorbox{gray}{light-gray}{(\ensuremath{\forall\kappa.\,}(a : Type)) \ensuremath{\to} (\ensuremath{\forall\kappa.\,_{g}}Stream a) \ensuremath{\to\,_{g}}Stream a}$
$_g$evens$''$ a s =
         $\evens{\text{(evens a)}}{\text{(evens a)}}{\,(_{g}\text{evens a})}{head s :: evens a (tail (tail s))}{Stream a}$

$_g$evens$'$ : ($\forall\kappa.\,$(a$\,$:$\,$Type)) $\to$ ($\forall\kappa.\,$$_{g}$Stream a) $\to$ $_{g}$Stream a
$_g$evens$'$ a = $\lambda$s. $_g$evens$''$ a s

$_g$evens : ($\forall\kappa.\,$(a$\,$:$\,$Type)) $\to$ ($\forall\kappa.\,$$_{g}$Stream a) $\to$ ($\forall\kappa.\,$$_{g}$Stream a)
$_g$evens a = $\Lambda\kappa.\,$$_g$evens$'$ a
\end{lstlisting}
The inference system can now attempt to infer a guarded recursive version as
specified in the box.

\section{The Inference System}
\label{sec:inference-system}
% RULES
% Regler - forklar (også notation)
% Rekursive referencer
% Eksempel - zeros?
% Preprocessing
% Renaming
% Forskel på causal og non causal

%(1) Gennemgang af interessante regler
%(2) Eksempel
%(3) Implementation details, e.g. when Next and tensor rules clash
%%(3.1) Preprocessing, herunder renaming
%%(3.2) Forskel på causal og non causal

Inference of guarded recursive TT terms from elaborated TT terms happens by a
bottom-up, type-directed derivation according to the rules given in
Figure~\ref{fig:epsilon_rules}. Let $\Gamma$ be a typing environment and $\iota$ be the recursive
reference for the function which is currently being analysed. Then the inference
environment ($IE$) in which guarded recursive TT terms are built is defined as follows:
\begin{align*}
   \clockEnv,\,\clockEnv^\prime &::=\,\open\,|\,\closed & \text{(singleton clock)} \\
   \Psi &::=\,\causal\,|\,\noncausal & \text{(modality)} \\
   \rho &::= n \mapsto \,_gn & \text{(renaming)} \\
   \phi &::=\,\cdot\,|\,\rho\,\phi & \text{(collection of renamings)} \\
   \pi &::= \,n \mapsto (_gn,\, _{\forall}n) & \text{(projection renaming)} \\
   \Pi &::= \cdot\,|\,\pi \,\Pi & \text{(collection of projection renamings)} \\
   IE &::= \iota;\Psi;\phi;\,c;\Pi;\,\Gamma & \text{(inference environment)}
\end{align*}
The singleton clock is either open (\open) or closed (\closed). Modality is
specified by $\Psi$, and is either causal (\causal) or non-causal
(\noncausal). The intuition is that a causal function may only refer to past
values (thus \causal{} is pointing backwards), while a non-causal function may
refer to values at any point in time (thus \noncausal{} points in both
directions). A renaming, given as $\rho$, matches an Idris name $n$ to a guarded
name $_gn$. Disjoint from $\rho$, $\pi$ defines a renaming for an Idris name of
a record projection, to either a guarded name or a quantified name
($_{\forall}n$). In particular, in a given inference environment there is never
both a $rho$ and a $\pi$ for a given name $n$.

\subsection{Reading Guide}
Before presenting the inference system, we provide a reading guide for each type
of judgment. 

\subsubsection{The Inference Judgment}
The first of these, and the arguably most interesting, is the inference judgment.

\begin{center}
  ${IE\vdash e : A \infer e' : B}$
\end{center}

The inference judgment reads as follows: ``If $e$ has type $A$ in a given
inference environment $IE$, where $A$ is not
the type of types, and our goal is to infer a term of type $B$, then
$e'$ of type $B$ \emph{can be} the guarded recursive version $e$''. Since a
given term may have more than one valid guarded recursive version, we stress
that $e'$ \emph{can} be a solution. We restrict the use of the inference
judgment to terms which are not types, since we have not investigated inference
of types beyond the level of simple renaming.
% Inference (\infer) of a guarded recursive TT term $e'$ of type $B$ from an elaborated TT term $e$ of type $A$ is then defined as
% ${IE\vdash e : A \infer e' : B}$.
\subsubsection{The Projection Renaming Judgment}
Using the projection renaming judgment, we can replace a user-supplied
projection name with a guarded name or a quantified name, respectively.
\begin{center}
  ${IE\vdash \Pi(e) \mapsto (_ge,\,_{\forall}e)}$
\end{center}
This judgment reads: ``If a term $e$ is a
reference to a name $n$, where $n$ is the name of a projection, and $\Pi$ in the inference environment $IE$ maps $n$
to both a guarded name $_gn$ and a quantified name $_{\forall}n$, then $e$ must be replaced with a term $e'$ referring to
either $_gn$ or $_{\forall}n$, but not both''. The specific situation determines
which of the names is the right choice.

\subsubsection{The Renaming Equality Judgment}
Renaming equality is a structural equality on TT terms.
\begin{center}
  ${IE\vdash e \phieq e'}$
\end{center}
This rules reads: ``If $e$ and $e'$ are normalized terms in an inference environment $IE$,
then $e$ is structurally equal to $e'$ after each name $n$ in
$e$ has been substituted with their guarded name $_gn$, where ($n\mapsto\,_gn$) is in $\phi$''.

\subsubsection{The Typing Judgment}
The inference system operates with a standard typing judgment.
\begin{center}
  ${IE\vdash e : A}$
\end{center}
It reads: ``$e$ has type $A$ in the inference
environment $IE$''. Here $e$ can be any term except for a type.

% Additionally, we have the following judgment for types.
% \begin{center}
%   ${IE\vdash A\, type}$
% \end{center}
% It reads: ``A is a type, i.e. its type is the type of types, in the inference
% environment $IE$''.

\subsubsection{Side Conditions}
Most of the rules have a side condition, comparing a term $e$ with the recursive
reference, $\iota$. Whenever there is a side condition $e = \iota$, then $e$
must be structurally equal to $\iota$ after normalization. The inverse holds when
$e \not = \iota$.

This concludes the reading guide. We now proceed to present the rules of the
inference system.

% In Figure~\ref{fig:epsilon_rules}, the $\Longrightarrow_{Next^\kappa}$ rule follows directly from the $Next$ rule in Figure~\ref{fig:guarded_recursion_rules_clocks}. The
% $\Longrightarrow_{\Lambda\kappa}$ and $\Longrightarrow_{apply^\kappa}$ rules have a
% similar structure, but here the clock changes state from premise to conclusion
% (e.g. from open to closed). For handling guarded names, the
% $\Longrightarrow_{\phi}$ rule says that if some Idris name has type $A$, and its
% guarded version has type $A^\prime$, then we can insert the guarded name in its
% place. Note that the type $A^\prime$ of the guarded name is not arbitrary, but follows the
% rules for causal and non-causal function types (see Section~\todo{insert ref
%   here}). Renaming is important for the $\Longrightarrow_{App}$ rule, since it
% is often used for the transformations required in the premises. This reasoning
% also applies to the $\Longrightarrow_{\tensor^{\kappa}_{n}}$ rule. Here,
% the $n$ is a natural number denoting the required ``lateness'' of both input and
% output, where $n \ge 1$. Specifically, ${\laterkappa_{\!\!1} B \cong \laterkappa B}, {\laterkappa_{\!\!2}
% \cong \laterkappa\laterkappa B}$ and so on. For trivial cases, the
% $\Longrightarrow_{Refl}$ rule is used. \todo{Say something about abstraction and
% let-rules here, when they have been formulated.}

\begin{figure}[H]
\centering
\textbf{The Inference System}$\hfill \boxed{IE\vdash e : A \infer e' : B}$
\vspace{1em}
\subcaptionbox{
The rule for inferring \texttt{Next} applications.
}{\AXD{\IEopen\,\vdash\,e\,:\,A\,\infer\,e'\,:\,B}
\RLabel{e\not = \iota \quad (\infer_{Next^\kappa})}
\UID{\IEopen\,\vdash\,e\,:\,A\,\infer\,Next \, e'\,:\,\laterkappa B}
\DisplayProof


%%% Local Variables:
%%% mode: plain-tex
%%% TeX-master: "../copatterns-thesis"
%%% End:
}
\end{figure}
\begin{figure}[H]
\centering
\ContinuedFloat
\subcaptionbox{
The $\infer_{\Lambda\kappa}$-rule opens the closed clock. Due to the use of
a singleton clock, it is impossible to open a new clock in a context where the
clock is already open.
}{\AXD{\IEopen\vdash\,e\,:\,A\,\infer\,e'\,:\,B}
\RLabel{e \not = \iota \quad (\infer_{\Lambda\kappa})}
\UID{\IEclosed\,\vdash\,e\,:\,A\,\infer\,\Lambda\kappa.e'\,:\, \forall\kappa. B}
\DisplayProof


%%% Local Variables:
%%% mode: latex
%%% TeX-master: "../copatterns-thesis"
%%% End:
}
\end{figure}
\begin{figure}[H]
\centering
\ContinuedFloat
\subcaptionbox{
The $\infer_{[\kappa]}$-rule is dual to the  $\infer_{\Lambda\kappa}$-rule,
closing the open clock by clock application.
}{\AXD{\IEclosed\,\vdash\,e\,:\,A\,\infer\,e'\,:\,\forall\kappa.B}
\RLabel{e \not = \iota \quad (\infer_{[\kappa]})}
\UID{\IEopen\,\vdash\,e\,:\,A\,\infer\,e'[\kappa]\,:\,B }
\DisplayProof

%%% Local Variables:
%%% mode: latex
%%% TeX-master: "../copatterns-thesis"
%%% End:
}
\end{figure}
\begin{figure}[H]
\centering
\ContinuedFloat
\subcaptionbox{
This rule replaces one reference to a name with another. In particular, if the
renaming context, $\phi$, contains an entry for the name referred to by $e$,
then $_{g}e$ must be substituted for $e$.
}{\AXD{\IEc\,\vdash\,\phi(e) = \, _{g}e \quad \IEc\,\vdash\,_{g}e\,:\,A'}
\RLabel{(\infer_{\phi})}
\UID{\IEc\,\vdash\,e\,:\,A\,\infer\,_{g}eb\,:\,A' }
\DisplayProof

%%% Local Variables:
%%% mode: latex
%%% TeX-master: "../copatterns-thesis"
%%% End:
}
\end{figure}
\begin{figure}[H]
\centering
\ContinuedFloat
\subcaptionbox{
This rule renames a projection to its quantified version when both the result
type and the argument type of the inferred projection term is expected to be quantified over clocks.
}{\AXD{ \begin{matrix} 
      \IEc \, \vdash \, f \,:\,(a : A) \,\to\,B \quad
      \IEc \, \vdash \, \Pi(f)\,\mapsto (_gf,\,_{\forall}f) \\ \IEc \, \vdash
      \,_{\forall}f \, : \, (a' : \forall\kappa.A') \,\to\, \forall\kappa.B'
      \\
      \IEc \, \vdash \, x\, :\, A \, \infer \, y\, :\, \forall\kappa.A'
    \end{matrix}} 
\RLabel{f \not = \iota \quad (\infer_{\Pi_{\forall}})} 
\UID{\IEc \, \vdash \, f\, x\, :\, B[x/a] \,\infer\, _{\forall}f\  y\, :\, \forall\kappa.B'[y/a']}
\DisplayProof

%%% Local Variables:
%%% mode: latex
%%% TeX-master: "../copatterns-thesis"
%%% End:
}
\end{figure}
\begin{figure}[H]
\ContinuedFloat
\centering
\subcaptionbox{
Similar to the $\infer_{\Pi_{\forall}}$-rule, this rule renames a projection to
its guarded version when both the result type and the argument type of the
inferred projection term is \emph{not} expected to be quantified over clocks.
}{\AXD{ \begin{matrix} 
      \IEc \, \vdash \, f \,:\,(a : A) \,\to\,B \quad
      \IEc \, \vdash \, \Pi(f)\,\mapsto (_gf,\,_{\forall}f) \\ \IEc \, \vdash
      \,_{g}f \, : \, (a' : A') \,\to\, B'
      \quad
      \IEc \, \vdash \, x\, :\, A \, \infer \, y\, :\, A'
    \end{matrix}} 
\RLabel{f \not = \iota \quad (\infer_{\Pi_{g}})} 
\UID{\IEc \, \vdash \, f\, x\, :\, B[x/a] \,\infer\, _{g}f\  y\, :\, B'[y/a']}
\DisplayProof

%%% Local Variables:
%%% mode: latex
%%% TeX-master: "../copatterns-thesis"
%%% End:
}
\end{figure}
\begin{figure}[H]
\centering
\ContinuedFloat
\subcaptionbox{
This rule ensures that the system is invoked recursively on the subterms
of an application which is not expected to be in a $\laterkappa$ context.
}{\AXD{ \begin{matrix} 
      \IEc \, \vdash \, x : A \quad \IEc \, \vdash \, A \phieq A' \\
      \IEc \, \vdash \, x\, :\, A \, \infer \, y\, :\, A' \\
      \IEc \, \vdash \, f\, :\, (a \,:\, A)\, \to \, B \, \infer\, g\, :\, (a'
      \, : \, A')\, \to \, B' \\
    \end{matrix}} 
\RLabel{f \not = \iota \quad (\infer_{App})} 
\UID{\IEc \, \vdash \, f\, x\, :\, B[x/a] \,\infer\, g\  y\, :\, B'[y/a']}
\DisplayProof

%%% Local Variables:
%%% mode: latex
%%% TeX-master: "../copatterns-thesis"
%%% End:
}
\end{figure}
\begin{figure}[H]
\centering
\ContinuedFloat
\subcaptionbox{
The $\infer_{\tensor^\kappa_n}$ rule is used when the guarded recursive version
of an application is expected to be in a later ($\laterkappa$) context. Here, $n$ is a natural number denoting the
required availability of both inputs. Specifically, $\laterkappa_{\!\!n}$B
signifies and $n$-iterated application of the $\laterkappa$ type constructor,
such that ${\laterkappa_{\!\!1} B \cong \laterkappa B}, {\laterkappa_{\!\!2}B
\cong \laterkappa\laterkappa B}$ and so on. Therefore, $\tensor^{\kappa}_1$
denotes the standard later application, as presented in Section~\ref{sec:guarded-recursion}. Note that as discussed in
Section~\ref{sec:fixkappa-rule}, this rule does not apply to dependent function spaces.
}{\AXD{\begin{matrix} \IEopen \, \vdash \, A \phieq A' \\
                    \IEopen \, \vdash \, f\, :\, A\, \to \, B\, \infer \, g\,:\, \laterkappan (A'\, \to \, B') \\ 
                    \IEopen \, \vdash \, x\, :\, A\, \infer \, y\,:\,\laterkappan A'
     \end{matrix}} 
\RLabel{n \geq 1, f \not = \iota \quad (\infer_{\tensor^{\kappa}_{n}})} 
\UID{\IEopen \, \vdash \, f\, x\, :\, B\, \infer \, g\, \tensor^{\kappa}_{n} \, y\, :\, \laterkappan B'}
\DisplayProof

%%% Local Variables:
%%% mode: latex
%%% TeX-master: "../copatterns-thesis"
%%% End:
}
\end{figure}
\begin{figure}[H]
\centering
\ContinuedFloat
\subcaptionbox{
The reflexivity rule simply states that any term can be inferred to itself. This
rule is used for terms which occur in both the elaborated TT term and the guarded
recursive TT term, e.g. constants.
}{\AXD{}
\RLabel{(\infer_{Refl})}
\UID{\IEc\,\vdash\,e\,:\,A\,\infer\,e\,:\,A}
\DisplayProof
%%% Local Variables:
%%% mode: latex
%%% TeX-master: "../copatterns-thesis"
%%% End:
}

\end{figure}
\begin{figure}[H]
\centering
\ContinuedFloat
\subcaptionbox{
  The $\infer_\lambda$-rule handles the recursive case by adding a new renaming
  entry to the renaming environment, $\phi$, such that all occurrences of the
  bound variable $a$ in $e$ can be renamed to $_{g}a$.
}{\AXD{ \begin{matrix} 
                     \iota;\Psi;\phi,(a,\,_{g}a);\clockEnv;\Pi;\Gamma, (a:A),(_{g}a
                     : A') \, \vdash \, e\, : \, B \, \infer\, e'\, :\, B'
    \end{matrix}} 
\RLabel{(\infer_{\lambda})} 
\UID{\IEc \, \vdash \, \lambda (a\,:\,A). \, e \, :\, (x\,:\,A) \, \to \, B  \,\infer\, \lambda (_{g}a\,:\,A'). \, e \, :\, (x \,:\,A') \, \to \, B'}
\DisplayProof


% \[
% \frac { \begin{matrix} \phi ;c;\Gamma \, \vdash A\, :\, Type\, \overset { Type }{ \Longrightarrow  } \, A^{ \prime  }\, :\, Type \\ \phi ;c;\Gamma ,x:A\, \vdash \, e\, :\, B\, x\, \overset { B^{ \prime  }y }{ \Longrightarrow  } \, \phi ;c;\Gamma ,y:A^{ \prime  }\, \vdash \, e{ ^{ \prime  } }\, :\, B^{ \prime  }y \\ \phi ;c;\Gamma \, \vdash A,A^{ \prime  },B,B^{ \prime  }\, :\, Type \end{matrix} }{ \phi ;c ;\Gamma \, \vdash \, \lambda x.\, e\, :\, (x\, :\, A)\, \to \, B\, x\, \Longrightarrow \, \phi ;c;\Gamma \, \vdash \, \lambda x.\, e^{ \prime  }\, :\, (y\, :\, A^{ \prime  })\, \to \, B^{ \prime  }\, y\, :\, B^{ \prime  }y } \Longrightarrow _{ \lambda  }
% \]


%%% Local Variables:
%%% mode: latex
%%% TeX-master: "../copatterns-thesis"
%%% End:
}

\end{figure}
\begin{figure}[H]
\centering
\ContinuedFloat
\subcaptionbox{
As for the $\infer_\lambda$-rule, the $\infer_{let}$-rule uses the renaming
environment for the recursive case for $b$.  
}{\AXD{\begin{matrix} \IEc \, \vdash \, \phi(A)\,=\,A' \quad \IEc \, \vdash \, A'\,:\,Type \\
                    \IEc \, \vdash \, e\,:\,A \,\infer\, e'\,:\,A' \quad
                    \IEc \, \vdash \, _{g}x \,:\, A' \\
                    \iota;\Psi;\phi,(x,\,_{g}x);\clockEnv;\Gamma, (x:A) \, \vdash \,b\,:\,B[e/x] \,\infer\, b'\,:\, B[e'/_{g}x]
     \end{matrix}} 
\RLabel{(\infer_{let})} 
\UID{\IEc \, \vdash \, let\,x\,\mapsto\,e\,:\,A\, in\, b \, : \, B[e/x] \, \infer \,let\,_{g}x\,\mapsto\,e'\,:\,A' \, in\, b' \, : \, B[e'/_{g}x]}
\DisplayProof

%%% Local Variables:
%%% mode: latex
%%% TeX-master: "../copatterns-thesis"
%%% End:
}
\end{figure}

\begin{figure}[H]
\centering
\ContinuedFloat
\subcaptionbox{
Due to the elimination of the indexed fixed point, this rule ensures that no
arguments to the non-causal recursive reference has a type in a $\laterkappa$
context. The $e\, \overset{\rightarrow}{x_n}$ notation denotes the recursive reference,
$e$, applied to arguments $x_0,\,x_1\,\cdots\,x_n$.
}{\AXD{\begin{matrix}
                    \IEopennoncausal \, \vdash \, e\, \overset{\rightarrow}{x_{n-1}}\, :\, A\, \to \, B\, \infer \, e'\,:\, \laterkappa (A'\, \to \, B') \\ 
                    \IEopennoncausal \, \vdash \, a_n\, :\, A\, \infer \, a_n^{'}\,:\,A'
     \end{matrix}} 
\RLabel{n > 0,\, e = \iota \quad (\infer_{Rec_{\noncausal_n}})} 
\UID{\IEopennoncausal \, \vdash \, e\, \overset{\rightarrow}{a_n}\, :\, B\,
  \infer \, e'\, \tensor^{\kappa}_{1} \, Next \, a_n^{'}\, :\, \laterkappa B'}
\DisplayProof

%%% Local Variables:
%%% mode: latex
%%% TeX-master: "../copatterns-thesis"
%%% End:
}

\end{figure}
\begin{figure}[H]
\centering
\ContinuedFloat
\subcaptionbox{
Handles the non-causal recursive reference applied to one argument. This rule
is necessary since the recursive case for $e$ is different from the $\infer_{Rec_{\noncausal{_n}}}$-rule.
}{\AXD{\begin{matrix} \IEopennoncausal \, \vdash \, e\, \, :\, A\, \to \, B\, \infer \, Next \, e'\,:\, \laterkappa (A'\, \to \, B') \\ 
                    \IEopennoncausal \, \vdash \, x_0\, :\, A\, \infer \, x_0^{'}\,:\,A'
     \end{matrix}} 
\RLabel{e = \iota \quad (\infer_{Rec_{\noncausal_0}})} 
\UID{\IEopennoncausal \, \vdash \, e\, x_0\, :\, B\,
  \infer \, Next \, e'\, \tensor^{\kappa}_{1} \, Next \, x_0^{'}\, :\, \laterkappa B'}
\DisplayProof

%%% Local Variables:
%%% mode: latex
%%% TeX-master: "../copatterns-thesis"
%%% End:
}

\end{figure}
\begin{figure}[H]
\centering
\ContinuedFloat
\subcaptionbox{
This rule ensures that a \texttt{Next} is always applied to any non-causal recursive reference.
}{\AXD{\begin{matrix} \IEopennoncausal \, \vdash \, e \phieq e' \quad
                    \IEopennoncausal \, \vdash \, e' \, : \, A'                    
     \end{matrix}} 
\RLabel{e = \iota \quad (\infer_{Rec_{\noncausal}})} 
\UID{\IEopennoncausal \, \vdash \, e \, : \, A \, \infer \, Next \, e' \, : \, \laterkappa {A'}}
\DisplayProof

%%% Local Variables:
%%% mode: latex
%%% TeX-master: "../copatterns-thesis"
%%% End:
}
\end{figure}
\begin{figure}[H]
\centering
\ContinuedFloat
\subcaptionbox{
  The rule for recursive references for causal functions. All types
    used in typing judgments are assumed to be well-formed. This rule recovers
    the eliminated recursive reference for causal functions, since it is the
    only rule that applies for the recursive reference.}
{\AXD{\begin{matrix} \IEopencausal \, \vdash \, e \phieq e' \quad
                    \IEopencausal \, \vdash \, e' \, : \, \forall\kappa. A'                    
     \end{matrix}} 
\RLabel{e = \iota \quad (\infer_{Rec_{\causal}})} 
\UID{\IEopencausal \, \vdash \, e \, : \, A \, \infer \, Next \, (e'[\kappa]) \, : \, \laterkappa {A'}}
\DisplayProof

%%% Local Variables:
%%% mode: latex
%%% TeX-master: "../copatterns-thesis"
%%% End:
}
  
  \caption{Rules for inferring guarded recursive terms, excluding the rules for
    recursive references. All types used in typing judgments are assumed to be
    well-formed.}
  \label{fig:epsilon_rec_causal}
\label{fig:epsilon_rules}
\end{figure}

\subsection{Choosing an Inference Rule}


\subsection{An Example}
As an example, we show how to infer the guarded recursive version of the
function \texttt{repeat} in Figure~\ref{fig:repeat_proof_example_program}. The
proof is given in Figure~\ref{fig:repeat_inference_proof}. Here, all side
conditions have been elided, but can be easily proven by testing for equality
with the recursive reference, \texttt{repeat a}. The proof proceeds by
continually taking apart the input term, until the inference of the subterms can
be resolved by either renaming ($\infer_\phi$) or reflexivity
($\infer_{Refl}$). As soon as the type of one subterm has been inferred, its
type trickles down into the rest of the proof tree, leading more inferences to
resolution. None of the inferred terms are given prior to the inference, but the
type of the first step is known after preprocessing. Additional examples of the
inference of guarded terms can be found in Appendix~\ref{app:infer-guard-recurs}.

%All side conditions have been elided, but can be easily proven for each rule.
%Should be read top-down, but must be understood bottom-up. 
%Only the right-hand side of _grepeat' is inferred
%A full example, along with a more advanced one, can be found in
%Appendix~\ref{app:infer-guard-recurs}.\todo{Make this appendix}
%The derivation proceeds by continually pushing type requirements down into the syntax tree

\begin{figure}[h]
\begin{lstlisting}[mathescape, title=\ttBlock]
repeat : (a : Type) -> a -> Stream a
repeat a n = (::) a n (repeat a n)
\end{lstlisting}
\begin{lstlisting}[mathescape, title=\ttBlock]
$_g$repeat$'$ : (a$\,$:$\,$Type) $\to$ a $\to$ $_{g}$Stream a
$_g$repeat$'$ a n = $(_g::)$ a n $(\onk{\text{Next}}\,((_g$repeat$\ $a$)[\kappa]) \tensor^\kappa_1 (\onk{\text{Next}}\,
$n$))$

causal 
$_g$repeat : (a$\,$:$\,$Type) $\to$ ($\forall\kappa.\,$ a $\to$ $_{g}$Stream a)
$_g$repeat a = $\Lambda\kappa.\,$$\lambda$n.$\,$$_g\text{repeat}'$ a n
\end{lstlisting}
  \caption{Above: A simple causal function \texttt{repeat}. The data constructor for
    Stream, \texttt{(::)} is used in a prefix manner. Below: The guarded
    recursive version. Note that only the right-hand side of
    \texttt{$_g$repeat$'$} is inferred by the inference system, while the
    remaining changes happen by preprocessing.}
\label{fig:repeat_proof_example_program}
\end{figure}


% Figure~\ref{fig:epsilon_zeros} shows how a guarded recursive version of
% \texttt{zeros} can be inferred using the rules presented in Figure~\ref{fig:epsilon_rules}. The
% derivation proceeds by continually pushing type requirements down into the
% syntax tree, until these can be resolved either trivially (using the
% $\Longrightarrow_{Refl}$ rule), or by renaming (using the
% $\Longrightarrow_{\phi}$ rule). 

% Note that all premises ensuring type consistency
% have been omitted for brevity. 


\begin{figure}[H]
\centering

\subcaptionbox{
The derivation proceeds within this inference environment. From top to bottom:
The recursive reference is \texttt{repeat a}, and \texttt{repeat} is causal
(\causal). The renaming environment, $\phi$, is extended with entries for the
stream constructor \texttt{(::)} and \texttt{repeat a}, and the clock is
initially open (\open) due to preprocessing (see
Section~\ref{sec:prepr-caus-funct}). Finally, the typing environment, $\Gamma$,
is extended as shown.
}{
$\begin{matrix*}[l]IE = & \text{repeat\ a};
  \\ & \causal ;\\ &
\phi, (::,\ _g::), (\text{repeat\ a},\, _g\text{repeat\ a});\\ &
 \open;\\ &
\Gamma,\text{a} : \text{Type},\ \text{n} : \text{a},\,_g\text{repeat} : (\text{a}\ :\ \text{Type}) \to \forall \kappa.(\text{a} \to \onk{\text{Stream}}\ \text{a}),\ 
\\ & \ \ \ _{g}::\ : (\text{a}\ :\ \text{Type})
\to \text{a} \to \later^\kappa_1 \onk{\text{Stream}}\ \text{a} \to \onk{\text{Stream}}\ \text{a}
\end{matrix*}$
}
\end{figure}
\begin{figure}[H]
\centering
\ContinuedFloat
\subcaptionbox{
We begin with the conclusion, namely that the guarded recursive version of the
right-hand side of \texttt{repeat} is the term shown to the right of the double arrow (\infer) in
the conclusion. In this first step, we use the $\infer_{App}$-rule to take apart the two outermost applications.
}{
\AxiomC{(1)}
\AxiomC{}
\RightLabel{$\infer _{Refl}$}
\UnaryInfC{$IE \vdash n : a \infer n : a$}
\RightLabel{$\infer _{App}$}
\BinaryInfC{$\begin{matrix} IE \vdash (::)\,a\,n \ : Stream\,a \to Stream\,a \infer \\
  (_g::)\,a\,n \  : \later ^\kappa_1  \onk{Stream}\,a \to
  \onk{Stream}\,a\end{matrix}$}
\AxiomC{(2)}
\RightLabel{$\infer _{App}$}
\BinaryInfC{ $\begin{matrix}IE \vdash (::)\,a\,n\,(repeat\ a\ n) : Stream\,a\infer
\\ \,(_g::)\,a\,n\,((\onk{Next}\,((_grepeat\ a)[\kappa]) \tensor^\kappa_1 (\onk{Next}\,
n) : Stream^\kappa \,a\end{matrix}$}
\DisplayProof
}
\end{figure}
\begin{figure}[H]
\centering
\ContinuedFloat
\subcaptionbox{
The third application is taken apart in the same manner.
}{
\AxiomC{(3)}
\AxiomC{}
\RightLabel{$\infer _{Refl}$}
\UnaryInfC{$IE \vdash a : Type \infer a : Type$}
\LeftLabel{(1)}
\RightLabel{$\infer _{App}$}
\BinaryInfC{$
\begin{matrix}IE \vdash (::)\,a : a \to Stream\,a \to Stream\,a \infer
  \\ (_g::)\,a \  : a \to \later ^\kappa_1  \onk{Stream}\,a \to
  \onk{Stream}\,a\end{matrix}$}
\DisplayProof
}
\end{figure}
\begin{figure}[H]
\centering
\ContinuedFloat
\subcaptionbox{
Here, we need to show that the recursive reference can be applied to the $n$
argument by using the rule for later application (\tensor).
}{
\AxiomC{(4)}
\AxiomC{}
\RightLabel{$\infer _{Refl}$}
\UnaryInfC{$IE \vdash n : a \infer n : a$}
\RightLabel{$\infer _{\onk{Next}}$}
\UnaryInfC{$IE \vdash n : a \infer \onk{Next}\ n : \later^\kappa_1 a$}
\LeftLabel{(2)}
\RightLabel{$\infer _{\tensor^\kappa_n}$}
\BinaryInfC{$\begin{matrix}IE \vdash repeat\ a\ n : Stream\ a \infer \\ (\onk{Next}\,  (_grepeat\ a)[\kappa])
  \tensor^\kappa_1 (\onk{Next}\ n) : \later^\kappa_1 \onk{Stream} a\end{matrix}$}
\DisplayProof
}
\end{figure}
\begin{figure}[H]
\centering
\ContinuedFloat
\subcaptionbox{
The $\infer_\phi$-rule is used to show that the data constructor
\texttt{($_g$::)} for $_g$Stream is the guarded recursive
version of the \texttt{(::)} constructor for \texttt{Stream}.
}{
  \AxiomC{}
  \UnaryInfC{$IE \vdash \phi(::) =\ (_g::) $}
\AxiomC{}
  \UnaryInfC{$\begin{matrix}IE \vdash (_g::) : (a : Type) \to a \to \\
    \later^\kappa_1\onk{Stream}\,a \to \onk{Stream}\,a\end{matrix}$}
\RightLabel{$\infer _{\phi}$}
\LeftLabel{(3)}
\BinaryInfC{$\begin{matrix}IE \vdash (::) : (a : Type) \to a \to Stream\,a \to
    Stream\,a \infer \\
    (_g::) : (a : Type) \to a \to \later^\kappa_1\onk{Stream}\,a \to \onk{Stream}\,a
\end{matrix}$}
\DisplayProof
}
\end{figure}
\begin{figure}[H]
\centering
\ContinuedFloat
\subcaptionbox{
The rule for the recursive reference in the causal case is used to eliminate the
recursive reference \texttt{repeat a}.
}{
  \AxiomC{(5)}
\AxiomC{}
\UnaryInfC{$\begin{matrix} IE \vdash _grepeat : (a\ :\ Type) \to  \forall
    \kappa.( a \to \onk{Stream}\ a) \end{matrix}$}
\AxiomC{}
\UnaryInfC{$IE \vdash a : Type$}
\RightLabel{App}
\BinaryInfC{$IE \vdash \, _grepeat\ a : \forall \kappa.( a \to \onk{Stream}\ a)$}
\RightLabel{$\infer _{Rec\causal}$}
\LeftLabel{(4)}
\BinaryInfC{$\begin{matrix}IE \vdash repeat\ a : a \to Stream\ a \infer \\ \onk{Next} (_grepeat\ a)[\kappa]
  : a \to \onk{Stream}\ a\end{matrix}$}
\DisplayProof
}
\end{figure}
\begin{figure}[H]
\centering
\ContinuedFloat

\subcaptionbox{
Follows directly from the inference environment.
}{
\AxiomC{}
\LeftLabel{(5)}
\UnaryInfC{$IE \vdash \phi(repeat\ a) = \,_grepeat\ a$}
\DisplayProof
}

\caption{An inference of the guarded recursive version of repeat.}
\label{fig:repeat_inference_proof}
\end{figure}


% \begin{figure}
% \begin{lstlisting}[mathescape]
% causal zeros : Stream Nat
% zeros = Z :: zeros

% $_g$zeros : $\forall\kappa.$ $_g$Stream Nat
% $_g$zeros = $\Lambda\kappa.$ Z $_g$:: (Next (apply $_g$zeros))
% \end{lstlisting}
%   % \[
%   % \frac{x}{\eps{IE}{Z :: zeros}{Stream Nat}{IE}{Z :: gzeros}{gStream Nat}}
%   % \]
% %\input{figures/epsilon_zeros}
%   \caption{Above: The user-provided program \texttt{zeros}, along with the
%     inferred program $_g$\texttt{zeros}. Below: Inferring $_g$\texttt{zeros} from \texttt{zeros} using the rules from
%   Figure~\ref{fig:epsilon_rules}.}
%   \label{fig:epsilon_zeros}
% \end{figure}

% \subsubsection{Implementing the Inference System}
% \label{sec:impl-infer-syst}

% \paragraph{Preprocessing} Two aspects of the inference system, renaming and
% delaying the recursive reference, are mandatory, and these 
% are therefore handled in a separate preprocessing step. Both must take modality
% into account.

% Renaming, i.e. all applications of the $\Longrightarrow_{\phi}$ rule, is
% performed by substituting all Idris names in $\phi$ with their guarded
% versions. Special care must be taken when substituting names of projection
% functions, since the modality of the function in question, $f$, influences the choice
% of guarded name. If $f$ is causal, 

% Delaying the recursive reference
%Preprocessing
%Modality and Recursive reference
%When rules clash

\section{The Guarded Recursion Checker}
\label{sec:guard-recurs-check}
The term inferred by the inference system is potentially ill-typed according to the
guarded recursion typing rules. If this is the case, the input term can not be
guaranteed to be productive. We therefore need a way of checking types of
guarded recursive terms.

The checking algorithm is based on the typing rules defined by M\o
gelberg\,\citep{Mogelberg:2014}, although they have been adjusted according to
the restrictions imposed by the singleton clock. In Figure~\ref{fig:gr_rules_sin_clock}, we
present the adjusted typing rules. Each rule is presented alongside the original
rule, and the translation is explained. Note that the
checking environment(\emph{CE}) is expanded with a recursive reference, $\iota$, a modality, $\Psi$,
and a singleton clock environment. These extensions to the environment 
have the same definitions as those given for the inference environment in
Section~\ref{sec:inference-system}. Also, the $fix^\kappa$-rule from M\o
gelberg's set of rules is missing. Due to fixed point elimination, discussed in
Section~\ref{sec:fixkappa-rule}, the rule for the fixed point operator is
replaced by rules for the recursive reference, for both the causal and the non-causal case.

As the input to the guarded recursion checking system is TT terms, the checking system is a
decoupled extension to the existing TT type system presented in Section~\ref{sec:tt-core-type}. In any situation where
both a standard TT rule and a guarded recursion rule applies, the guarded
recursion rule must be chosen in order to ensure that all side conditions are checked.

% the standard Idris typing
% rules\,\citep{BradyIdrisImpl13} do also apply, however the guarded recursive
% rules take precedence and will be applied if possible.

\subsubsection{The Typing Judgment}
The checking system operates with a standard typing judgment.
\begin{center}
  ${CE\vdash e : A}$
\end{center}
It reads: ``$e$ has type $A$ in the checking environment $CE$''. 

% New things in env
\begin{figure}
\textbf{The Typing Rules}
\begin{center}
\subcaptionbox{
Looking at the clock environment in the premise $\Delta, \kappa$ it is clear
that this translates to the open clock ($\open$). In the conclusion we translate
$\Delta$ to the closed clock, because $\Delta$ without $\kappa$ is empty as $\kappa$ is
the only clock in a singleton clock system. The side condition $\kappa \notin fc(\Gamma)$ translates to
$nofc(\Gamma)$, because if the only clock, $\kappa$, is not allowed to
be free in $\Gamma$ then no clock is allowed free in $\Gamma$.
}{

\AxiomC{$\Delta, \kappa ; \Gamma \vdash A : Type$}
\AxiomC{$\kappa \notin fc(\Gamma)$}
\BinaryInfC{$\Delta ; \Gamma \vdash \forall \kappa . A : Type$}
\DisplayProof
\quad
    \AxiomC{$\CEopen \vdash A:Type$}
    \AxiomC{$nofc(\Gamma)$}
    \RightLabel{$I_\forall$}
    \BinaryInfC{$\CEclosed \vdash \forall \kappa . A : Type$}
\DisplayProof
}
\vspace{1em}
\subcaptionbox{
The interesting thing to note in this rule is $\kappa \in \Delta$. If $\kappa$ is in
$\Delta$ and $\kappa$ is the only clock, then $\Delta$ must translate to the
open clock for this side condition to hold. This eliminates the side condition, as $\kappa \in \open$ is always true.
}{
\AxiomC{$\Delta ; \Gamma \vdash A : Type$}
\AxiomC{$\kappa \in \Delta$}
\BinaryInfC{$\Delta ; \Gamma \vdash \laterkappa A : Type$}
\DisplayProof
\quad
    \AxiomC{$\CEopen \vdash A:Type$}
    \RightLabel{$I_{\laterkappan}$}
    \UnaryInfC{$\CEopen \vdash \laterkappan A : Type$}
\DisplayProof  
\vspace{1em}
}
\vspace{1em}
\subcaptionbox{
The only change in this rule is how we consider the environment. We translate
$\Delta$ to the open clock, as $\onk{Next}$ implies a need for
a clock.
}{
\AxiomC{$\Delta ; \Gamma \vdash t : A$}
\UnaryInfC{$\Delta ; \Gamma \vdash \onk{Next}\ t : \laterkappa A$}
\DisplayProof
\quad
    \AxiomC{$\CEopen \vdash t : A$}
    \RightLabel{$I_{\onk{Next}}$}
    \UnaryInfC{$\CEopen \vdash \onk{Next}\ t : \laterkappan A$}
\DisplayProof
\vspace{1em}
}
\vspace{1em}
\subcaptionbox{
For the same reason as in the $I_\forall$-rule, $\Delta, \kappa$ becomes the open clock, and
just $\Delta$ becomes the closed clock. As before, $\kappa \notin fc(\Gamma)$
becomes $nofc(\Gamma)$.
}{
\AxiomC{$\Delta, \kappa ; \Gamma \vdash t : A$}
\AxiomC{$\kappa \notin fc(\Gamma)$}
\BinaryInfC{$\Delta ; \Gamma \vdash \Lambda \kappa. t : \forall \kappa. A$}
\DisplayProof
\quad
\AxiomC{$\CEopen \vdash t : A$}
\AxiomC{$nofc(\Gamma)$}
    \RightLabel{$I_{\Lambda \kappa}$}
\BinaryInfC{$\CEclosed \vdash \Lambda \kappa . t : \forall \kappa . A$}
\DisplayProof
}

\AxiomC{$\kappa \notin fc(\Gamma)$}
\AxiomC{$\Delta, \kappa ; \Gamma \vdash A : Type$}
\AxiomC{$\Delta, \kappa' ; \Gamma, \Gamma' \vdash t : \forall \kappa. A$}
\TrinaryInfC{$\Delta, \kappa' ; \Gamma , \Gamma ' \vdash t[\kappa'] : A[{  \kappa'}/{\kappa  }]$}
\DisplayProof
\vspace{1em}

\subcaptionbox{
Since, originally, this rule mentions two different clocks $\kappa$ and
$\kappa'$, we have to look at the premises and the conclusion  one by one to
make sense of them in a singleton clock environment. If $\forall \kappa . A$ is
to be a type, then $A$ must be a type under the open clock. If $t$ is to be of type
$\forall \kappa . A$ it must open a clock, according to $I_{\Lambda \kappa}$,
and thusly be checkable under the closed clock. And finally the
conclusion, if we apply a clock on a term, then a clock exist, thus a clock must
be open. Furthermore, the clock substitution on $A$ makes no sense with only one
clock. As usual,  $\kappa \notin fc(\Gamma)$ becomes $nofc(\Gamma)$.
}{
\AxiomC{$nofc(\Gamma)$}
\AxiomC{$\CEopen \vdash A : Type$}
\AxiomC{$\CEclosed ,\Gamma' \vdash t : \forall \kappa . A$}
    \RightLabel{$I_{[\kappa]}$}
\TrinaryInfC{$\CEopen ,\Gamma ' \vdash t[\kappa] : A$}
\DisplayProof
}
\end{center}
\vspace{2em}
\end{figure}

\begin{figure}
\begin{center}
\ContinuedFloat
\AxiomC{$\Delta ; \Gamma \vdash t : \laterkappa (A \to B)$}
\AxiomC{$\Delta ; \Gamma \vdash u : \laterkappa A$}
\BinaryInfC{$\Delta ; \Gamma \vdash t \tensor^\kappa u : \laterkappa B$}
\DisplayProof

\vspace{1em}
\subcaptionbox{
While there is nothing that states anything about the clock environment in the
original rule, both the later types ($\later$) and later composition ($\tensor$)
requires a clock to make sense. Because of this we require the clocks to be
open.
}{

    \AxiomC{$\CEopen \vdash t : \laterkappan (A \to B) $}
    \AxiomC{$\CEopen \vdash u : \laterkappan A$}
    \RightLabel{$I_{\tensorkappan}$}    
    \BinaryInfC{$\CEopen \vdash t \tensorkappan u : \laterkappan B$}
\DisplayProof  
}
\vspace{1em}
\subcaptionbox{
This rule does not have a direct counter part in the original model, but is
rather a consequence of our fixed point elimination from
Section~\ref{sec:fixkappa-rule}. A causal recursive reference must be on the
form $\onk{Next}\ e[\kappa]$, according to the rule in
Figure~\ref{fig:fix_elim_rules}. This ensures the correct lateness compared to
the one of $\iota$, given that $e = \iota$ and $\iota : \forall \kappa . A$.
}{
\AxiomC{$\CEopencausal \vdash \iota : \forall \kappa . A$}
\RightLabel{$e = \iota \quad Rec_{\causal}$}
\UnaryInfC{$\CEopencausal \vdash \onk{Next}\ e[\kappa] : \laterkappan A$}
\DisplayProof
}
\vspace{1em}
\subcaptionbox{To recover from eliminating the indexed fixed point, this rule ensures that
  all arguments to the non-causal recursive references has a $\later^\kappa_1$
  type.}{
\AxiomC{$\begin{matrix}\CEopennoncausal \vdash (\onk{Next}\ e) \tensor^\kappa_1
    (\onk{Next}\ x_0) \tensor^\kappa_1 \\
    \cdots \tensor^\kappa_1 (\onk{Next}\ x_{m-1}) : \later^\kappa_1(A \to B)
\\ \CEopennoncausal \vdash (\onk{Next}\ x_n) : \later^\kappa_1 A\end{matrix}$}
\RightLabel{$\begin{matrix} n > 0 \\ e = \iota\end{matrix} \quad Rec_{\noncausal_n}$}
\UnaryInfC{$\CEopennoncausal \vdash (\onk{Next}\ e) \tensor^\kappa_1 (\onk{Next}\ x_0) 
  \cdots \tensor^\kappa_1 (\onk{Next}\ x_n) : \later^\kappa_1B$}
\DisplayProof
}
\vspace{1em}
\subcaptionbox{Here, the non-causal references with one argument is
  handled. This rule is necessary as a base case for the $Rec_{\noncausal_n}$ rule.}{
\AxiomC{$\begin{matrix}\CEopennoncausal \vdash (\onk{Next}\ e) : \later^\kappa_1(A \to B)\end{matrix}$}
\AxiomC{$\CEopennoncausal \vdash (\onk{Next}\ x_0) : \later^\kappa_1 A$}
\RightLabel{$e = \iota \quad Rec_{\noncausal_0}$}
\BinaryInfC{$\CEopennoncausal \vdash (\onk{Next}\ e) \tensor^\kappa_1 (\onk{Next}\ x_0) : \later^\kappa_1B$}
\DisplayProof
}
\vspace{1em}
\subcaptionbox{
This rule ensures that the recursive reference has the correct lateness compared
to $\iota$.
}{
\AxiomC{$\CEopennoncausal \vdash \iota : A$}
\RightLabel{$e = \iota \quad Rec_{\noncausal}$}
\UnaryInfC{$\CEopennoncausal \vdash \onk{Next}\ e : \laterkappan A$}
\DisplayProof
}
\vspace{1em}

\subcaptionbox{
This is an extension of the Idris $Var$ rules. They state that types with a free
clock can only occur if clock is open.
}{
\AxiomC{$(\lambda t : \onk{A}) \in \Gamma$}
\RightLabel{$Var_{1_\open}$}
\UnaryInfC{$\CEopen \vdash t : \onk{A}$}
\DisplayProof

\AxiomC{$(\forall t : \onk{A}) \in \Gamma$}
\RightLabel{$Var_{2_\open}$}
\UnaryInfC{$\CEopen \vdash t : \onk{A}$}
\DisplayProof
}

\end{center}
  \caption{Guarded Recursion rules with singleton clock.}
  \label{fig:gr_rules_sin_clock}

\end{figure}
%%% Local Variables:
%%% mode: latex
%%% TeX-master: "../copatterns-thesis"
%%% End:

\subsection{An Example}
In Figure~\ref{fig:repeat_inference_proof} an inference of the guarded recursive
version of the function \texttt{repeat}'s guarded-recursive counterpart was shown. In continuation of this, we
show that $_g$\texttt{repeat} is productive according to the guarded recursion
checking rules. The derivation is given in Figure~\ref{fig:repeat_typing_example}. Additional examples of such proofs can be found in Appendix~\ref{app:check-guard-recurs}.

\begin{figure}[H]
\centering
\subcaptionbox{
The type check takes place in this checking environment. From the top the
environment is: The recursive reference is $repeat\ a$, which is a causal
(\causal) function, and due to the earlier described preprocessing, the clock starts as
open (\open). Finally the typing environment $\Gamma$, is extended as shown.
}{
$\begin{matrix*}[l]CE = & _g\text{repeat\ a};
  \\ & \causal ;\\ &

 \open;\\ &
\Gamma,\text{a} : \text{Type},\ \text{n} : \text{a},\,_g\text{repeat} : (\text{a}\ :\ \text{Type}) \to \forall \kappa.(\text{a} \to \onk{\text{Stream}}\ \text{a}),\ 
\\ & \ \ \ _{g}::\ : (\text{a}\ :\ \text{Type})
\to \text{a} \to \later^\kappa_1 \onk{\text{Stream}}\ \text{a} \to \onk{\text{Stream}}\ \text{a}
\end{matrix*}$
}
\end{figure}
\begin{figure}[H]
\centering
\ContinuedFloat
\subcaptionbox{
Starting with the inferred term in the conclusion, we first apply the $App$
rule, an Idris typing rule, as we have a regular application between $(_g::)\ a\
n$ and $(\onk{Next}\,((_grepeat\ a)[\kappa])) \tensor^\kappa_1
(\onk{Next}\,n)$. Repeating the application rule we end up having to show that
$n$ is of type $a$ which we know from the environment.
}{
\AxiomC{(1)}
\AxiomC{}
\RightLabel{$Var_1$}
\UnaryInfC{$CE \vdash n : a$}
\RightLabel{$App$}
\BinaryInfC{$CE \vdash (_g::)\ a\ n : \later^\kappa_1 \onk{Stream}\ a \to
  \onk{Stream}\ a$}
\AxiomC{(2)}
\RightLabel{$App$}
\BinaryInfC{$CE \vdash \,(_g::)\,a\,n\,((\onk{Next}\,((_grepeat\ a)[\kappa])) \tensor^\kappa_1 (\onk{Next}\,
n)) : Stream^\kappa \,a$}
\DisplayProof
}
\end{figure}
\begin{figure}[H]
\centering
\ContinuedFloat
\subcaptionbox{
This application is taken apart in the same way, leaving us with two things we
know from the environment.
}{
\AxiomC{}
\RightLabel{$Var_{1_\open}$}
\UnaryInfC{$CE \vdash (_g::) : (a : Type) \to a \to \later^\kappa_1 \onk{Stream}\ a \to
  \onk{Stream}\ a$}
\AxiomC{}
\RightLabel{$Var_1$}
\UnaryInfC{$CE \vdash a : Type$}
\LeftLabel{(1)}
\RightLabel{$App$}
\BinaryInfC{$CE \vdash (_g::)\ a : a \to \later^\kappa_1 \onk{Stream}\ a \to
  \onk{Stream}\ a$}
\DisplayProof
}
\end{figure}
\begin{figure}[H]
\centering
\ContinuedFloat
\subcaptionbox{
Here, we start by applying the $\tensorkappan$-rule, leaving us with two new
obligations, both $\laterkappan$. To the first we can apply the rule for the
recursive references rule in the causal case. Lastly, the $\onk{Next}\ a$-case
is trivially shown by the $I_{\onk{Next}}$-rule and from the environment.
}
{
\AxiomC{}
\RightLabel{$Var_1$}
\UnaryInfC{$CE \vdash \ _grepeat\ a : \forall \kappa . (a \to \onk{Stream}\ a)$}
\RightLabel{$Rec_\causal$}
\UnaryInfC{$CE \vdash (\onk{Next} (_grepeat\ a))[\kappa] : \later^\kappa_1 (a \to \onk{Stream}\ a)$}
\AxiomC{}
\RightLabel{$Var_1$}
\UnaryInfC{$CE \vdash n : a$}
\RightLabel{$I_{\onk{Next}}$}
\UnaryInfC{$CE \vdash \onk{Next}\ n : \later^\kappa_1 a$}
\LeftLabel{(2)}
\RightLabel{$I_{\tensorkappan}$}
\BinaryInfC{$CE \vdash (\onk{Next} (_grepeat\ a))[\kappa] \tensor^\kappa_1 (\onk{Next}\,
n) : \later^\kappa_1 \onk{Stream}\ a$}
\DisplayProof
}
  \caption{Proof that $_g$repeat is well-typed.}
  \label{fig:repeat_typing_example}
\end{figure}
%%% Local Variables:
%%% mode: latex
%%% TeX-master: "../copatterns-thesis"
%%% End:

%%% Local Variables:
%%% mode: latex
%%% TeX-master: "../copatterns-thesis"
%%% End:



\section{Discussion}
\label{sec:discussion}
Although the system presented throughout this chapter can prove productive quite
a few realistic programs (see Chapter~\ref{cha:evaluation}), it currently has
several shortcomings and limitations.
%\subsection{Guarded Recursion and Dependent Types}

\subsection{Erasure}


\subsection{Dependent Function Types and Fixed Point Elimination}
\label{sec:depend-funct-types}
As described in Section~\ref{sec:fixkappa-rule}, our current implementation uses
fixed point elimination in order to enable inference of guarded recursive
programs with dependent types. However, this leads the system to make an
unfortunate assumption. Recall the example of \texttt{prepend} from
Section~\ref{sec:fixkappa-rule}:
\begin{lstlisting}[mathescape, title=\ttBlock]
prepend$'$ : (n : Nat) $\to$ (a : Type) $\to$
           $\laterkappa\,$(Vect n a $\to$ Stream$^{\kappa}\,$a $\to$ Stream$^{\kappa}\,$a) $\to$ 
           Vect n a $\to$ Stream$^{\kappa}\,$a $\to$ Stream$^{\kappa}\,$a
prepend$'$ n a rec []        s = s 
prepend$'$ n a rec (x $_g$:: xs) s = 
                    ($_g$::) a x ((rec $\tensor$ (Next xs)) $\tensor$ (Next s))

prepend : (n : Nat) $\to$ (a : Type) $\to$ 
          $\forall\kappa.$(Vect n a $\to$ Stream$^\kappa$ a $\to$ Stream$^\kappa$ a)
prepend n a = $\Lambda\kappa.\,$fix$^\kappa$($\lambda$rec.$\lambda$xs.$\lambda$s. prepend$'$ n a rec xs s)
\end{lstlisting}
Here, the type of \texttt{rec} is {$\laterkappa\,$(Vect n a $\to$
  Stream$^{\kappa}\,$a $\to$ Stream$^{\kappa}\,$a)}. Through fixed point
elimination, we wish to substitute \texttt{rec} with \texttt{{(Next$^{\kappa}$
    ((prepend[$\kappa$]) n a)}}, such that we arrive at the following definition
of \texttt{prepend}:
\begin{lstlisting}[mathescape, title=\ttBlock]
prepend$'$ : (n : Nat) $\to$ (a : Type) $\to$
           Vect n a $\to$ Stream$^{\kappa}\,$a $\to$ Stream$^{\kappa}\,$a
prepend$'$ n a []        s = s 
prepend$'$ n a (x $_g$:: xs) s = 
   ($_g$::) a x (((Next$^{\kappa}$ ((prepend[$\kappa$]) n a)) $\tensor$ (Next xs)) $\tensor$ (Next s))

prepend : (n : Nat) $\to$ (a : Type) $\to$ 
          $\forall\kappa.$(Vect n a $\to$ Stream$^\kappa$ a $\to$ Stream$^\kappa$ a)
prepend n a = $\Lambda\kappa.\,$fix$^\kappa$($\lambda$xs.$\lambda$s. prepend$'$ n a xs s)
\end{lstlisting}
Because \texttt{n} and \texttt{a} are universally quantified in the type of
\texttt{prepend}, we are allowed to specialise the type of \texttt{prepend} on
the right-hand side of \texttt{prepend$'$}, making the implementation
well-typed. In the type of \texttt{rec}, however, \texttt{n} and \texttt{a} are
fixed. Hence, we have by fixed point elimination assumed that a recursive
reference with fixed parameters can be substituted with a term with universally
quantified parameters. In general, we do not know whether such a substitution is sound, since we
cannot recover the fixed point version of the program afterwards.

Consequently, the analysis should not rely on fixed point elimination. Instead,
the inference should simply proceed after step 2 of preprocessing for causal
functions, and after step 3.1 for non-causal functions (see
Section~\ref{sec:impl-guard-recurs}). For the inference system, fixed point
elimination means that recursive references must be handled differently for
causal and non-causal functions, namely by the $\infer_{Rec_\causal}$-rule in
the causal case and by the $\infer_{Rec_\noncausal_{n}}$,
$\infer_{Rec_\noncausal_{0}}$, and $\infer_{Rec_\noncausal}$ rules in the non-causal
case. Without fixed point elimination, the recursive reference can be handled by
the inference system with one uniform rule for both the causal and the noncausal case:
 
\begin{prooftree}
\AxiomC{$\IEc\vdash \phi(\iota)=e$}
\AxiomC{$\IEc\vdash e:A'$}
\RLabel{\infer_{Rec}}
\BinaryInfC{$\IEc\vdash \iota : A \infer e : A$'$$}
\end{prooftree}

This would simplify the inference system considerably. From the perspective of
the checking system, the absence of fixed point elimination would mean that all
the checking rules for recursive references (both causal and noncausal) could be
replaced by a single fixed point rule:

\begin{prooftree}
\def\fCenter{\vdash}
\Axiom$\IEopen,x:\laterkappa A\fCenter t : A$
\RLabel{fix_I}
\UnaryInf$\IEopen\fCenter fix^{\kappa} x. t : A$
\end{prooftree}

Since fixed point elimination in general is has not yet been proven sound,
although it seems to work in most cases up to polymorphic types, it should not be
part of a future implementation.

\subsection{Preservation of Semantics}
\label{sec:pres-semant}
In Section~\ref{sec:inference-system}, we presented the rules for inferring
guarded recursive TT terms from elaborated TT terms. Even though we are
confident in our approach, we have no proof that the semantics of the inferred
guarded recursive terms are the same as the semantics of the input
terms. Naturally, the guarded recursive version must have the same semantics in
order for the productivity analysis to have any value. We imagine that such a
proof could be provided by showing that the input term and the output term of each
rule reduce to the same or equivalent TT semantics. The TT
semantics are defined by Brady\,\citep{BradyIdrisImpl13}.

\subsection{Type Class Instances}
\label{sec:type-class-instances}
The inference system does not support function definitions given in type class
instances at present, as these require special treatment. Type classes in Idris
are implemented by dictionary passing, and type class constraints are modeled as
additional parameters. Consequently, Idris expects that for each type class
instance for a coinductive type $A$, its guarded type $_gA$ has an equivalent
instance. Type class instances for guarded types are not derived at the moment,
so no guarded recursive definition involving type classes can be
well-typed. Deriving type class instances should be possible during elaboration,
since it is merely a question of inferring guarded recursive versions of each
instance function, taking modality into account.

% An alternative to deriving guarded type class instances is to eliminate the idea of
% type classes entirely in the inference system, such that 

\subsection{Mutual Recursion}
\label{sec:mutual-recursion}
Since our productivity analysis emerges from the use of the guarded recursive
fixed point defined in Section~\ref{sec:guarded-recursion}, mutually recursive
definitions are not (yet) supported. The reason is that our analysis prevents
infinite recursion by ensuring that all direct recursive references are not
immediately available, seeing as they must have a type under a specific type
constructor, later ($\laterkappa$). If other constructs can lead to infinite
recursion, these must be identified separately. This is deferred as future work
(see Chapter~\ref{cha:future-work}).

\subsection{Totality Dependencies}
\label{sec:total-depend}
Currently, the system does not keep track of totality dependencies between
functions. Consider the following program:
\begin{lstlisting}[mathescape, title=\idrisBlock]
nats : Stream Nat
nats = Z :: map S nats
\end{lstlisting}
Here, \texttt{nats} depends on the totality of \texttt{map}. If \texttt{map}
turns out to be partial, it is not currently detected that \texttt{nats} is also
partial as a consequence thereof. The reason is that the productivity analysis
is a type-level check, so the analysis of \texttt{nats} relies on the promise that
a total version of \texttt{map} has been found, and not on the actual
implementation of \texttt{map}. This can quite easily be fixed by checking that
any non-recursive reference has already been proven total before it is accepted
as the guarded recursive version of a term.

\subsection{User-written Guarded Recursion}
\label{sec:user-written-guarded}
% Problem: Non-causal functions require two functions
%% However, the checker checks non-causal definitions in an open clock
Our system does not currently allow the user to circumvent the inference
system. This means that users cannot have the checking system verify whether
their hand-written guarded recursive definitions are productive, since the
checking system assumes that all input definitions have been preprocessed
according to the procedure described in Section~\ref{sec:inference-system}.

However, allowing users to write guarded recursive programs is not merely a
question of adapting the checking system. Consider the following well-typed
Idris program, potentially written by a user:
\begin{lstlisting}[mathescape, title=\idrisBlock]
total force : Later (Tomorrow Now) a $\to$ a
force (Next a) = a

total f : Stream Nat
f = $\Lambda\kappa.\,$fix$^\kappa$($\lambda$rec.$\,$force$\,$rec)
\end{lstlisting}
Here, \texttt{fix$^\kappa$} is used correctly in \texttt{f}, but the use of
\texttt{force} means that \texttt{f} is not productive. Although the recursive
reference is well-typed, infinite recursion results from the application of
\texttt{force}. In practice, this leads to a situation where users can add
arbitrary axioms to the checking system, since the checking system cannot
readily dispute the correctness of the well-typed and total definition of
\texttt{force}.

In order to allow users to write guarded recursive programs manually,
restrictions must therefore be implemented in the Idris compiler. Concretely,
the example above is problematic because the user is allowed to define pattern
matching on a value of type \texttt{Later}. Restricting users from writing such
definitions would therefore be necessary. Further exploration into this area is
needed in order to determine whether additional restrictions are required.

\subsection{Multiple Clocks}
\label{sec:multiple-clocks}
In Section~\ref{sec:singleton-clock}, we explained that the inference system has
been restricted to only inferring guarded recursive terms that require at most
one clock variable. The $\infer_{\Lambda\kappa}$ and $\infer_{[\kappa]}$ rules
take advantage of this restriction, since they assume that the clock in question
is always the singleton clock. Extending these rules to multiple clocks would
most likely be hard, because the $\infer_{[\kappa]}$ rule would have to infer
which clock to apply in any given situation.

In contrast, the checking system could probably be successfully extended to
accommodate multiple clocks. It merely has to check whether a given situation is
allowed according to the rules in Figure~\ref{fig:gr_rules_sin_clock}, and does
not perform any inference whatsoever. Therefore, user-written guarded recursive
definitions with multiple clocks could most likely be supported by extending the
checking system.

\subsection{Error Reporting}
\label{sec:error-reporting}
Giving helpful error messages concerning guarded recursion can be difficult,
mainly because it may require that the user is familiar or at least partially
familiar with the underlying theory. Aside from that, it should be possible to
generate quite detailed error messages, considering that each rule may fail
independently. We have not yet put any effort into providing helpful error
messages, but we imagine that such messages could have the form: ``The causal
definition $f$ is not productive because $x$ at line $n$, column $c$, has no
guarded recursive form of type $A'$ (inferred from type $A$)''. All information
used in this example is directly available in each of the inference rules.






%%% Local Variables:
%%% mode: latex
%%% TeX-master: "../copatterns-thesis"
%%% End:


\chapter{Evaluation}
\label{cha:evaluation}
\section{Guarded Recursion}
In Section~\ref{sec:less-restr-prod} we discussed the motivation behind
implementing a less restrictive productivity checker. In
Chapter~\ref{cha:infer-guard-recurs} we introduced a system for
productivity checking functions utilizing guarded recursion. In the following we
present the result of implementing this system in Idris. We look into
what can now be automatically proven total, but also what can not. The
implementation of all functions discussed in this section can be found in
Appendix~\ref{app:example-programs}. 

We divide productive functions into three classes: 
\begin{itemize}
\item Functions we can prove total.
\item Functions we should be able to prove total, but can not.
\item Functions we should not and can not prove total.
\end{itemize}

The first class is self-explanatory. The second stems from our implementation. While
we do indent to fully implement the system described in
Chapter~\ref{cha:infer-guard-recurs}, our current implementation has not quite
reached full coverage yet. The last covers functions that are productive, but that we
can not infer a guarded recursive definition for.
% Intro with link back to motivation.
% Three classes of functions:
% What we can check
% What we should be able to check
% What we should not be able to check.
% Why can we check what we can?
% Why can't we check what we should be able to?
% Why should we not be able to check what we shouldn't?
% Modality

Figure~\ref{fig:productivity_table_1} shows examples of functions falling into the
first category. While some functions are already proven productive under the current Idris
syntactic guardedness principle, we can see that we have added to the set of
Idris functions automatically proven productive. We can, for example, now prove
productivity for functions where the recursive is given as an argument to
another function.

\begin{figure}[h]
\begin{center}
  \begin{tabular}{| l | c | c |} \hline
    Function & Syntactic Guardedness & Guarded Recursion \\ \hline
    \texttt{zeros} & Productive & Productive \\ \hline
    \texttt{repeat} & Productive & Productive \\ \hline
    \texttt{map} & Productive & Productive \\ \hline
    \texttt{zipWith} & Productive & Productive \\ \hline
    \texttt{toggle} & Productive & Productive \\ \hline
    \texttt{interleave} & Productive & Productive \\ \hline
    \texttt{unfold} & Productive & Productive \\ \hline
    \texttt{prepend} & Productive & Productive \\ \hline
    \texttt{cycle} & Productive & Productive \\ \hline
    \texttt{nats} & Not Productive & Productive \\ \hline
    \texttt{fib} & Not Productive & Productive \\ \hline
    \texttt{fac} & Not Productive & Productive \\ \hline
    \texttt{paperfold} & Not Productive & Productive \\ \hline
    \texttt{calculateWilfully} & Productive & Productive \\ \hline
    \texttt{tmap} & Productive & Productive \\ \hline
    \texttt{tzip} & Productive & Productive \\ \hline
    \texttt{trepeat} & Productive & Productive \\ \hline
    \texttt{carry} & Not Productive & Productive \\ \hline
    \texttt{pingserver'} & Not Productive & Productive \\ \hline
    \texttt{imap} & Productive & Productive \\ \hline
    \texttt{counter} & Not Productive & Productive \\ \hline
  \end{tabular}
\end{center}
  \caption{Examples of functions we can prove productive using our guarded
    recursive system. Note that the header ``Syntactic Guardedness'' referes to the current
    Idris implementation of syntactic guardedness, and ``Guarded Recursion''
    refers to our implementation of guarded recursion in Idris.}
  \label{fig:productivity_table_1}
\end{figure}

Figure~\ref{fig:productivity_table_2} shows examples of functions falling into
the second category. These are functions for which our implementation can not
yet infer a guarded term. We are, however, convinced the problem lies within the
implementation of our rules, and not the rules themselves. As such, a better
implementation should judge these functions as productive. We will evaluate on
our implementation in Section~\ref{sec:our-implementation}.

\begin{figure}[h]
\begin{center}
  \begin{tabular}{| l | c | c |} \hline
    Function & Syntactic Guardedness & Guarded Recursion \\ \hline
    \texttt{never} & Not Productive & Not Productive \\ \hline
    \texttt{bind} & Not Productive & Not Productive \\ \hline
    \texttt{multmachine} & Not Productive & Not Productive \\ \hline
    \texttt{pingserver} & Not Productive & Not Productive \\ \hline
  \end{tabular}
\end{center}
  \caption{Examples of functions we can not yet prove productive. Productivity
    of these
    functions should be provable within our system.}
  \label{fig:productivity_table_2}
\end{figure}

The last class we described are functions that we do not think our system can
prove productive. While only a few examples are listed in
Figure~\ref{fig:productivity_table_3}, we expect for there to be more of such
definitions. The problem with these functions is that they require a very specific
type to be proven productive.

\subsection{Function Types}
\label{sec:function-types}
To simplify inference we categorized functions into causal and
non-causal functions. We used this to help us infer the guarded type of a function. There
are, however, functions for which the guarded type falls outside this
categorization. The examples in Figure~\ref{fig:productivity_table_3} both need
more sophisticated types to be proven total.

\begin{figure}[h]
\begin{center}
  \begin{tabular}{| l | c | c |} \hline
    Function & Syntactic Guardedness & Guarded Recursion \\ \hline
    \texttt{mergef} & Not Productive & Not Productive \\ \hline
    \texttt{paperfold'} & Not Productive & Not Productive \\ \hline
  \end{tabular}
\end{center}
  \caption{Examples of functions that we can not prove productive. Productivity
    of these functions should not be provable within our system.}
  \label{fig:productivity_table_3}
\end{figure}

Atkey and McBride\,\citep{Atkey:2013} prove \texttt{mergef} productive by
changing the type of the input function. They alter it to be:

\begin{lstlisting}[mathescape, title=\idrisBlock]
mergef : (Nat $\to$ Nat $\to \laterkappan$Stream$^\kappa$ Nat $\to$ Stream$^\kappa$) $\to$
         Stream$^\kappa$ Nat $\to$ Stream$^\kappa$ Nat $\to$ Stream$^\kappa$ Nat
mergef f s t = f (head s) (head t) 
                (mergef $\tensorkappan$ (Next$^\kappa$ f) $\tensorkappan$ (tail s) $\tensorkappan$ (tail t))
\end{lstlisting}

This follows the intuition that the \texttt{Stream} argument to \texttt{f} must be used in a
later context. It allows us to give the recursive reference, which is always
later, as an argument to \texttt{f}.

In a similar fashion, Clouston et al.\,\citep{BirkedalL:guarded-lambda-conf}
prove \texttt{paperfold'} productive by using a differently typed \texttt{interleave}:

\begin{lstlisting}[mathescape, title=\idrisBlock]
interleave : Stream$^\kappa$ Nat $\to$ $\laterkappan$ Stream$^\kappa$ Nat $\to$ Stream$^\kappa$ Nat
\end{lstlisting}

This type captures that while the first argument is needed straight away, the
second argument is always only needed later. With this type for interleave they
can define a guarded \texttt{paperfolds'}:

\begin{lstlisting}[mathescape, title=\idrisBlock]
paperfolds' : Stream$^\kappa$ Nat
paperfolds' = interleave toggle paperfolds'
\end{lstlisting}

Inferring these types would be no easy task, and has been deemed out of scope of
this project. With better support for user interaction with the
inference system, one could imagine a system where the user could specify these
types instead of attempting to infer them. The inference system would still be
inferring the term, but instead of attempting to infer a type, it will use the
type defined by the user.

% \begin{figure}[h]
% \begin{lstlisting}[mathescape]
% semiinterleave : $\forall \kappa$.Stream$^\kappa$ a $\to$ $\forall \kappa$(Stream$^\kappa$ a $\to$ Stream$^\kappa$ a)
% semiinterleave s = $\Lambda \kappa.$ semiinterleave'(s)

% semiinterleave' : $\forall \kappa$.Stream$^\kappa$ a $\to$ Stream$^\kappa$ a $\to$ Stream$^\kappa$ a
% semiinterleave' = pfix(f)

% f : ($\forall \kappa$.Stream$^\kappa$ a $\to$ $\later$$^{\kappa}$(Stream$^\kappa$ a $\to$ Stream$^\kappa$ a)) $\to$ 
%        $\forall \kappa$.Stream$^\kappa$ a $\to$ Stream$^\kappa$ a $\to$ Stream$^{\kappa}$ a
% f g xs s = StreamCons (head xs)
%     (Next (StreamCons (head s) 
%                       ((g (tail (tail xs))) $\tensor^\kappa$(tail s))))
% \end{lstlisting}
% \caption{A function interleave a stream with the even indices of another
%   stream.}
% \label{fig:semiinterleave}
% \end{figure}

\subsection{Our Implementation}
\label{sec:our-implementation}
As discussed, the implementation of our system is not yet complete. There are
definitions that we should be able to check that we can not. The cause of this is
simply a buggy implementation. In the later parts of the project we have
prioritized fine-tuning our system on a theoretical level, rather than ironing
out bugs in our implementation. This has resulted in a stronger system, but a
weaker implementation.

Our implementation, while flawed, demonstrates that implementing
the system described in Chapter~\ref{cha:infer-guard-recurs} is feasible. It
shows that, through a set of inference rules, it is possible to infer a guarded
term. This term is then checked against the guarded recursion rules, to ensure
productivity.

We are therefore convinced that it is possible to improve our implementation, to closer
resemble our system. There are definitely improvements to be made, but what we
have done thus far has already improved the productivity checker in Idris,
allowing for more total definitions to be written.
%%% Local Variables:
%%% mode: latex
%%% TeX-master: "../copatterns-thesis"
%%% End:


%!TEX root = ../copatterns-thesis.tex
\chapter{Related Work}
\label{cha:related-work}

\section{Copatterns}
\label{sec:related_work_copatterns}
\todo{Relate to 'Unnesting of Copatterns' by Setzer et al.}

\section{Size-change Termination}
Since Idris already has a working totality checker using size-change termination\,\citep{BradyIdrisImpl13}, it would be interesting if the size-change principle could be used for determining the productivity of corecursive functions as well. The size-change principle for termination was first proposed for a strict first-order functional language (without loop constructs) by Lee, Jones, and Ben-Amram\,\citep{LeeJones01SizeChange}. The principle essentially states that if infinitely many recursive calls to a function would lead to infinite decrease in some parameter value, then the function must be terminating, since any value of an inductive type must have finite size. This last condition is of particular importance, as the size-change principle cannot in general recognize functions as being terminating if they have parameters that do not exhibit a well-founded order. Lee, Jones, and Ben-Amram present two realizations of the principle, one using automata and one using a call graph. In the graph formulation, termination is determined by identifying any recursive calls (both direct and indirect) through cycles in the call graph, and then constructing a ``size-change graph'' for each of these. The size-change graphs are then used to find out whether infinite descent in some parameter value is present. One of the limitations of this approach is that parameter values must decrease monotonically: Values cannot at any point become structurally larger, even though the total change in size in a call chain would ultimately lead a to a decreasing value. Two examples of size-change terminating functions are shown in Figure~\ref{fig:sizechange_plus_map}. In both examples, the recursive call happens on structurally smaller input.

\begin{figure}
\begin{lstlisting}[mathescape]
plus : Nat $\to$ Nat $\to$ Nat
plus Z      m = m
plus (S n') m = plus n' m

map : (a $\to$ b) $\to$ Vect n a $\to$ Vect n b
map f []        = []
map f (x :: xs) = f x :: map f xs
\end{lstlisting}
\caption{Two size-change terminating functions.}
\label{fig:sizechange_plus_map}
\end{figure}

Since its first-order formulation, the principle has been proven to be applicable to more expressive cases. Jones and Bohr\,\citep{Jones04Untyped} showed that size-change termination can be applied to the untyped lambda calculus using abstract interpretation\,\citep{Jones:1995}. A set of safe size-change graphs for a program is generated by defining evaluation rules without an environment component, thus overapproximating the number of possible values for any given variable. Given these rules, a corresponding overapproximated set of safe size-change graphs is generated for further termination analysis. Note that due to the undecidability of the halting problem, it is impossible in general to compute the exact set of safe size-change graphs for a given function call.

Following the work on the untyped lambda calculus, Sereni and Jones generalized the size-change principle to handle a higher-order functional language with user-defined data types and general recursion\,\citep{Sereni05terminationanalysis,Sereni06Phd}. Here, a termination criterion is presented which works for arbitrary control-flow graphs, and in turn is able to give an approximation of termination for both strict and lazy functional programs. A key point in this work is how different approaches to control-flow and call graph construction may influence the preciseness of the termination analysis.

All of the previously mentioned implementations of the size-change principle can only approximate termination for programs involving data which exhibits some well-founded order. Nevertheless, Avery\,\citep{Avery06} presented a formulation in which it is possible to detect size-change termination for non-well-founded data types --- in particular, this formulation is shown to work for a language with an integer type. Instead of identifying infinite descent using a well-founded partial order on parameter, as proposed by Lee at al., Avery's analysis is based on a decrease in invariants which are found to hold for each program point. The idea is that if the value of some invariant (which can involve arbitrarily many values) can be shown to decrease on every passage of a program point, then the program terminates. A simple example of a size-change terminating program in Avery's implementation is given in Figure~\ref{fig:avery_example}. The invariant for the inner loop is \texttt{n - j + i}, while the one for the outer loop is \texttt{n - i}. These cannot decrease indefinitely, and therefore the program is size-change terminating.

\begin{figure}
\begin{lstlisting}
for (i = 0; i <= n; i++) {
  for (j = 0; j-i <= n; j++);
}
return;
\end{lstlisting}
\caption{An example program involving integers written in a subset of C, which is size-change terminating in Avery's formulation.}
\label{fig:avery_example}
\end{figure}

While size-change termination for non-well-founded data could be a step towards using a size-change approach for an approximation of productivity, it is unclear which invariants one would have to infer for corecursive programs. A first attempt might emerge from the observation that any productive corecursive program cannot consume any more data than it produces. Based solely on syntax, such an analysis would quickly approach the idea behind syntactic guardedness checkers, which will be discussed in Section~\ref{sec:synt-guard}.

In more recent work, Hyvernat\,\citep{Hyvernat13} has proposed a formulation of the size-change principle for functional languages which to a certain degree solves the problem of non-monotonic decrease in parameter values. The motivation behind this work is to incorporate size-change termination into the PML language\,\cite{PMLLanguage}. Non-monotonic decrease is detected by tracking the size of a parameter throughout the entire control-flow graph, instead of merely recording whether each call in isolation leads to a decrease in some value.

The termination criterion most similar to the size-change principle predates the original article by Lee, Jones, and Ben-Amram\,\citep{LeeJones01SizeChange}. This criterion was developed by Abel for the \texttt{foetus} termination checker\,\citep{Abel98foetus}, and forms the basis of the totality checker implemented in Agda\,\citep{Norell:thesis}. Analogous to the size-change principle, Abel identifies recursive calls in a call graph and performs a termination analysis by tracking changes in parameter sizes. This initial presentation of \texttt{foetus} makes no mention of productivity for corecursive programs, although such an extension has since been presented by Altenkirch and Danielsson\,\citep{AltenkirchNAD10} (for further discussion, see Section~\ref{sec:synt-guard}).

\section{Syntactic Guardedness}
\label{sec:synt-guard}
% Telford and Turner nævnes mere
% Hyvernat flyttes herned

Due to the duality between inductive and coinductive types, it may be compelling to imagine a productivity analysis which, dual to the size-change principle, works by identifying structurally larger values. Such a dual notion of ``size-change productivity'' is exactly the idea behind the syntactic guardedness checkers found in Idris, Agda, and Coq\,\citep{Coq:manual}. Where values become structurally smaller by pattern matching, they become structurally larger by constructor application. Therefore, the \emph{guardedness principle} states that a coinductive definition is guarded if all corecursive calls appear directly under a coinductive constructor. This has the implication that the productivity of corecursive functions can be detected by a purely syntactic check.

The guardedness principle was first proposed by Coquand\,\citep{Coquand94} has a tool for constructive reasoning about infinite objects. Coquand argues that previous coinductive proof methods rely on impredicative definitions, which he deems to be ``unsatisfactory'' for constructive reasoning. Based on Milner's work on process calculi\,\citep{Milner:1989}, Coquand defines an infinite object to be productive if a (not necessarily well-founded) computation tree can be associated with it, through an analogy between infinite proof objects and processes. This observation is then synthesized into a ``guarded induction principle''. In practice, the guarded induction principle can be used for constructive proofs through a syntactic check, which verifies that all corecursive references appear directly under a coinductive constructor.

In continuation of Coquand's efforts, Gim\'{e}nez\,\citep{Gimenez95} formalized an extension of the Calculus of Constructions\,\citep{Coquand:1988}, in which a modification of Coquand's original formulation of the guarded induction principle is provided. In particular, the guarded induction principle is modified such that it can be applied to types with second-order quantification.

The guarded induction principle described by Coquand and Gim\'{e}nez was intended to be used within a proof system (e.g. Coq) for easier reasoning about infinite objects. Confirming Coquand's own observation, Telford and Turner\,\citep{Telford98ensuringthe} argue that the guarded induction principle is too conservative in a programming setting. In their system of \emph{Elementary Strong Functional Programming} (ESFP)\,\citep{Telford97ensuringstreams,Telford98ensuringthe,Telford:jucs_6_4:ensuring_termination_in_esfp}, Telford and Turner therefore extend practical use of the guarded induction principle to detect a wider range of function as productive. They achieve this by considering guardedness more abstractly over a domain of ``guardedness levels'', such that productive corecursion is bounded by a given ``depth''. Intuitively, their system works by detecting that a program produces more elements that it consumes, i.e. that recursive references are subject to more introduction forms (constructor applications) than elimination forms (case analysis). Since any program can be higher-order, the guardedness level of a function is calculated by a ``guardedness function'',  where the guardedness levels of both bound and free variables are taken into account. As an advanced example, Telford and Turner show that the program generating the Hamming numbers (a problem first discussed by Dijkstra\,\citep{Dijkstra:1997}) is accepted as a productive program by their system.

Hyvernat\,\citep{Hyvernat13} discusses an approach similar to the idea behind the work of Telford and Turner. He proposes that productivity can be detected by counting the number of abstractions and applications in a program, respectively. Given that one does not reduce under abstractions, a function can then be considered productive if it has more abstractions than applications, since this necessarily means that the program will not diverge.

Another approach to coping with the conservative nature of the original guardedness principle is to consider alternative programming styles, making productive definitions easier to write. Following the incorporation of the guardedness condition into the Agda totality checker as described by Altenkirch and Danielsson (they attribute their approach to Abel)\,\citep{AltenkirchNAD10}, Danielsson\,\citep{Danielsson10beatingthe} described a method for working around the guardedness condition whenever it rejects a productive program. When a corecursive reference appears under a call to a function which is not a constructor, he designs an embedded domain-specific language in which said function is implemented as a constructor, making the guarded induction principle applicable. He is then able to define an interpreter for this language, such that the original program is ultimately accepted as productive. None of these steps happen automatically, but must be done manually. Although useful, Danielsson argues that efficiency is a concern, and that the best solution might be to entirely move away from using guardedness for productivity. Sized types have since been implemented in Agda.

\section{Guarded Recursion}
%Nakano
%neelk
%Atkey and McBride
%Flere papers fra Aarhus + Mogel

As presented in Section~\ref{sec:guarded-recursion}, guarded recursive type systems can be used to define necessarily productive programs. The general notion underlying guarded recursion was introduced by Nakano\,\citep{Nakano:2000}, although he does not himself use the term ``guarded recursion''. He introduces a modal typing system with recursive types, $\lambda\!\bullet\!\mu$, in which it is possible to encode provably non-diverging programs. This is achieved partly by giving a modal type to the Y-combinator (see Figure~\ref{fig:nakano_Y}), and partly by defining types such that recursive occurrences are only available in the context of a modality $\bullet$, e.g. infinite binary trees as a type $\mu X.A\times\bullet X\times\bullet X$. Nakano suggests that his system could be related to temporal logic, where the $\bullet$ modality reprensents notions of ``previous time''. Additionally, he notes that the system is not intended to be used as a type system for programming languages, but serves to widen the range of programs to which the idea of proofs-as-programs can be applied.

\begin{figure}
\[
\vdash \lambda f. (\lambda x. f (x x)) (\lambda x. f (x x)) : (\bullet X \to X) \to X
\]
\caption{The Y-combinator as derived in Nakano's $\lambda\!\bullet\!\mu$ system\,\citep{Nakano:2000}.}
\label{fig:nakano_Y}
\end{figure}

Inspired by Nakano's approach, Atkey and McBride\,\citep{Atkey:2013} present a similar typing discipline based on the simply typed lambda calculus, paving the way for a more practical use of guarded recursion. They introduce applicative programming\,\citep{Mcbride:2008} over modalities, which coupled with Nakano's fixed point combinator makes programming with modalities more accessible, and provides a more compositional system for productivity checking. Their main contribution is the introduction of clock variables into a type system with guarded recursion, such that it is possible to distinguish between values that are in the process of construction and values that are fully constructed. Fully constructed values can be identified by universal quantification over clock variables, enabling the user to extract values from the modal world of guarded recursion.

Following the work by Atkey and McBride, M\o gelberg\,\citep{Mogelberg:2014} extends the idea of a type system with guarded recursion and clock variables to a dependently typed setting. Based on work by Birkedal and M\o gelberg\,\citep{BirkedalL:sgdtuniverse-conf} and Birkedal, M\o gelberg, Schwinghammer, and St\o vring\,\citep{BirkedalL:sgdt-journal}, he models a dependently typed lambda calculus with guarded recursion and clocks using the topos of trees model. This results in an extensional type theory for reasoning about guarded recursive types. Although no intensional formulation of the model is given, it is expected that such a formulation exists.

% In extensional type theory, definitional and propositional equality is the same
Typing systems with modalities on types as introduced by Nakano has also been applied in the field of functional reactive programming. Krishnaswami\,\citep{Krishnaswami13} presents a typing system where both time leaks (depending on past values for arbitrary intervals of time) and space leaks (non-permanent memory leaks from capturing too much history) are prevented by construction. This is accomplished by typing all values with \emph{temporal recursive types}, using modalities on types to indicate the time at which a value becomes available. In the context of a global clock, programs are then evaluated according to two operational semantics: one giving the semantics for a program at the current clock tick, and one for advancing the global clock. Whenever the global clock is advanced, all values which can no longer be referenced due to time constraints are discarded from the environment, thus eliminating space leaks. Accompanying this mechanism are rules which make references to unavailable data impossible. Time leaks are prevented by unfolding all values at the moment they become available. Since Krishnaswami's approach enforces causality (that the first \emph{n} outputs only depend on the first \emph{n} input values), it can essentially be used to encode guarded recursion, at least in the simply typed formulation by Atkey and McBride. But where Atkey and McBride emphasize that clock variables localize the encoding of productivity, Krishnaswami introduces a global clock to which all programs must adhere.

\section{Sized Types}
An alternative approach to type-based totality checking is sized types. Where the intuitive abstraction of guarded recursion is time, sized types attach sizes to values at the type level. The motivation behind sized types is to express totality proofs using induction on sizes, such that the structure of the term in question can be disregarded. With size-change termination, termination proofs are constructed by ensuring that all recursive arguments happen on structurally smaller terms. Analogously, totality proofs using sized types are constructed by ensuring that all values are defined in terms of data with a structurally smaller \emph{size}.  Hence, the term ``size'' implies no specific term structure on the data involved in the totality proof.

Sized types for totality checking were first proposed by Hughes, Pareto, and Sabry\,\citep{Hughes96} for reactive systems, where each data type introduced into a program is associated with a family of sized types indicating the bounds of a value of that type. A similar idea was developed by Amadio and Coupet-Grimal\,\citep{Amadio98}, where guard conditions are introduced into the type system to ensure the productivity of coinductive data, following the work of Coquand\,\citep{Coquand94} and Gim\'{e}nez\,\citep{Gimenez95}.

Eduarde Gim\'{e}nez also presented a system for typing recursive definitions in an extension of the Calculus of Constructions using sized types\,\citep{Gimenez98structuralrecursive}. A notable result of this work is that any well-typed term in the proposed extension is normalizing with respect to lazy evaluation, substantially widening the domain of functions to which type-based termination is applicable. In the wake of this extension, Abel\,\citep{Abel99terminationchecking} wrote a quite accessible paper on using sized types for totality checking, showing that bidirectional type checking\,\citep{Pierce00} is suitable for a system with sized types. In addition, it is described how infinite streams defined as coinductive types can be encoded in a language as functions on natural numbers, and that the productivity of a stream can be understood in terms of its \emph{definedness}, meaning the number of times it can safely be unfolded.

The notion of definedness is also an important part of Abel and Pientka's work on applying sized types to a system with copatterns\,\citep{Abel13Wellfounded}, although here it is defined more precisely as the \emph{depth} of a value of coinductive type. By directly using the notion of depth in the type system, they show that totality proofs for both coinductive definitions with copatterns and inductive definitions (as well as mixed inductive-coinductive definitions) can be constructed by well-founded induction on sizes within the type system. The method is shown to work for System F\textsubscript{$\omega$}, and has later been implemented in Agda along with copatterns.

%structurally smaller

%%% Local Variables:
%%% mode: latex
%%% TeX-master: "../copatterns-thesis"
%%% End:


%!TEX root = ../copatterns-thesis.tex
\chapter{Future Work}
\label{cha:future-work}
<<<<<<< Updated upstream
=======
\section{Copatterns}
\subsection{With-rule}
\subsection{Pattern matching on Codata}
\subsection{Improved Syntax}
>>>>>>> Stashed changes
%### Copatterns ###
% Copatterns in with-rule
% Pattern matching på codata i Idris?
% Måske bedre syntaks
% 
%##################
\section{Guarded Recursion}
\label{sec:guarded-recursion-1}
In this section we will go over how we will improve on our guarded recursion
system and develop it to be a better productivity checker.
\subsection{Low Hanging Fruit}
There are a number of improvements which do not require great loads of work to
implement, but that we have not done due to a lack of time. These \emph{low
  hanging fruits} are mostly Idris technicalities that require a better
understanding of how certain things are handled in Idris.
\paragraph{Instance Functions}
% Instans Funktioner

\paragraph{Dependent Productivity}
% Funktioner hvis produktivitet afhænger af hinanden
If a function calls another function, then its productivity depends on the
totality of the called function. We do not take this into account when
judging the productivity of functions.

This should be a fairly trivial task. Simply keep track of what functions depend
on others and when everything has been checked, check the dependencies. Mutual
recursion is a separate case which we will discuss in
Section~\ref{sec:mutual-recursion}.
\subsection{Mutual Recursion}
\label{sec:mutual-recursion}
We have not been able to find any theoretical solution for mutually recursive
functions, so we have not given them any special treatment, not have we deeply
explored the area. We did see in Figure~\ref{fig:productivity_table} that our
system was able to identify mutually recursive functions as productive, but this
is probably more due to coincidence.

An initial, conservative approach could be to treat all mutually recursive
references in the same way we treat recursive references now. This can be
thought of as making the fixed point on a product of functions rather than a
single function.
\subsection{Clocks}
While we have not been able to identify the use for more than one clock,
extending our system to have multiple clocks would be bring our implementation
closer to the theory, thus its desirable. As discussed in
Chapter~\ref{cha:infer-guard-recurs} inferring multiple clocks is not
trivial. There seems to be no better way of doing it than simple trial and
error, and even then we are not sure if we found the right inference. 

The checker, however, could be changed to handle multiple clocks in a way that
makes sense. While this obviously would not have much affect with no changes to
the inference system, adding multiple clocks to the checking algorithm could
be interesting for user written guarded recursion.
\subsection{User written Guarded Recursion}
To increase the usefulness of guarded recursion in Idris, it is important to
continually improve and support writing guarded recursion without the inference
system.
\paragraph{Partial Inference}
Another interesting aspect could be to allow the programmer to only write parts
of the guarded recursion, and then infer the missing parts. This could be seen
as sort of a \emph{hinting} system, where the programmer guides the inference system.
\subsection{Preservation of Semantics}

\subsection{Error Messages}

%### GUARDED recursion ####
% Flere ure: Hvor stort et problem er det kun at have eet ur?
% Mutual recursion
% (Instans-funktioner)
% Kan man skrive guarded recursion selv?
%% Delvis inferens?
% Bevis for preservation of semantics
%##########################

%%% Local Variables:
%%% mode: latex
%%% TeX-master: "../copatterns-thesis"
%%% End:


%!TEX root = ../copatterns-thesis.tex
\chapter{Conclusion}
% Copatterns virker i princippet, men en desugaring var nok ikke den bedste
% løsning
% Subject reduction
% Hvornår virker copatterns

% Dependent types og guarded recursion
% Fixed point elimination was a bad idea
% Singleton clock var en succes
% Modality var en succes
% Opdeling mellem inferens og checking
% Flere funktioner kan bevises at være totale i Idris

% Et godt udgangspunkt for videre arbejde
We set out to implement both copatterns and a better productivity checker for
Idris than the existing implementation of the syntactic guardedness
principle. To a certain degree, both of these goals were achieved.

The addition of copatterns was motivated by examining
dependent pattern matching on coinductive data in Idris. As it has been the case
for Coq and Agda, Idris also loses subject reduction when faced with dependent
pattern matching. Our implementation of copatterns was realized as a desugaring from \IdrisM{}
definitions to \IdrisM{} definitions. While this decision seemed like the best
solution at first, it does come with some limitations. Notably, we found that
our transformation had to rely on a somewhat awkward notion of compatibility
between pattern clauses. Also, since the desugaring happens prior to
elaboration, the present implementation does not allow us to disambiguate names
from types. Desugaring copatterns as a pre-elaboration step additionally means
it may become difficult to make the \texttt{with}-rule work. Consequently, we
conclude that an implementation of copatterns as a separate elaboration to TT is
probably a better solution. However, our implementation of copatterns was, for
the most part, a success, seeing as it is now possible to write definitions
using copatterns in Idris.

We implemented a productivity checker based on guarded recursion, which can
automatically construct productivity proofs for programs which are not
productive according to the syntactic guardedness principle. The productivity
checker was implemented as two disjoint systems, an inference system an a
checking system. The inference system is in charge of inferring guarded
recursive definitions from user-written definitions. The checking system ensures
that the inferred definition is well-typed according to the guarded recursive typing rules. This separation of concerns was a success, since it made
the inference system simpler and increased our confidence in the correctness of our
approach. 

One of our goals was to provide a more expressive productivity
analysis than the syntactic guardedness principle, while at the same time
requiring as little user involvement as possible. This was quite successfully
achieved, since users only have to specify the modality (i.e. causal or
non-causal) of each function definition returning coinductive data. Also, we
found that the use of a singleton clock greatly simplified the analysis, without
any perceived loss of expressiveness. 

The choice of eliminating the guarded
recursive fixed point in order to generalize the productivity analysis turned
out to be unfortunate, since this elimination may not be sound. However, for
function definitions without dependent types, fixed point elimination does not
seem to have an impact on the result of the analysis.

In continuation of the above, we find that a dependently typed system does not
fully benefit from the approach, since the theory behind guarded recursion
is not expressive enough. In particular, the rule for later
application does not (yet) hold for dependent function spaces.

Although our productivity analysis has its flaws, our implementation has widened the range of
programs that can be proven productive by the Idris compiler. In conclusion, we
find that while our approach is not yet fully refined, it is a good starting
point for further investigation into the area of automating productivity proofs
based on guarded recursion.


%%% Local Variables:
%%% mode: latex
%%% TeX-master: "../copatterns-thesis"
%%% End:


\bibliography{bibliography}

\appendix

\chapter{Example Programs}
\label{app:example-programs}

\lstinputlisting[
  title=\idrisBlock,
  frame=l,
  numbers=left,                   
  numbersep=5pt,                  
  numberstyle=\tiny\color{mygray}
]{examples/CoinductionExamplesReport.idr}



%%% Local Variables:
%%% mode: latex
%%% TeX-master: "../../copatterns-thesis"
%%% End:


\chapter{Record Types by Observation in Idris}
\label{app:record-types-observ}
In Sections~\ref{sec:copatterns} and \ref{sec:coinductive-types} we discussed
the concept of coinductive data defined by observations. In the following we
discuss how we have implemented a new way of defining recod types. Our
implementation allows the programmer to define records by observation, rather
than by a single constructor. Our new record syntax replaces the old record
syntax in Idris. We also explain how we have expanded the idea of records to
also include coinductive records, or just \emph{corecords}.

\section{Intuition}
To understand our implementation, let us first look at the current
implementation of records in Idris. In Figure~\ref{fig:records_in_idris} an
example of a product type, here a \texttt{Pair}, is defined as a
record. Syntactically, this looks a lot like a data declaration, except for the
\texttt{record} keyword. Also, we are required to define exactly one
constructor, unlike data declarations, where we can have zero or many.

\begin{figure}[h]
\begin{lstlisting}
record Pair : Type -> Type -> Type where
  MkPair : (fst : a) -> (snd : b) -> Pair a b
\end{lstlisting}
  \caption{Current record syntax in Idris.}
  \label{fig:records_in_idris}
\end{figure}

The difference between a record and a data declaration does not appear until just
before elaboration. Here the record is elaborated as a data declaration
(literally Figure~\ref{fig:records_in_idris}, with the keyword \texttt{data}
instead of \texttt{record}), and a set of functions generated. Both a
\emph{projection} and an \emph{update} function is generated for each named
parameter in the constructor, in the \texttt{Pair} example \texttt{fst} and
\texttt{snd}. These update and projection functions are then placed in a namespace with the
name of the type, in this case \texttt{Pair}. 

\begin{figure}[h]
\begin{lstlisting}
data Pair : Type -> Type -> Type where
  MkPair : (fst : a) -> (snd : b) -> Pair a b

namespace Pair
  fst : Pair a b -> a
  fst (MkPair x _) = x

  snd : Pair a b -> b
  snd (MkPair _ y) = y

  set_fst : Pair a b -> a -> Pair a b
  set_fst (MkPair _ y) x = MkPair x y

  set_snd : Pair a b -> b -> Pair a b
  set_snd (MkPair x _) y = MkPair x y
\end{lstlisting}
  \caption{\texttt{Pair} from Figure~\ref{fig:records_in_idris} desugared. Note
    that the above definitions differ slightly from what is actually inferred,
    but they are essentially the same.}
  \label{fig:pair_desugared}
\end{figure}

An example of how records are desugared can be seen in
Figure~\ref{fig:pair_desugared}. Records are currently a high-level level
purely syntactic construct. We aim for our implementation to be as similar as
the current one, to reuse as much of the existing record infrastructure as
possible. Therefore, our implementation should also be syntactic sugar.

\subsection{Coinductive Records}
Just as records are desugared to data declarations, coinductive records should be
desugared to coinductive data, or \texttt{codata}, declarations. From the parser, the difference between data and
codata declarations is a flag given to the elaborator. We can apply the same
approach to coinductive records, with a flag denoting whether the declaration should be
desugared to a data or codata declaration. 

\section{Implementation}
Our implementation is a syntactic desugaring. In the following we will describe
how each component of the new syntax translates to what in a desugared
version. So without further ado, we present the following record syntax: 

\begin{lstlisting}[mathescape]
record <name> <parameters> where
  <field$_1$ name> : <field$_1$ type>
  <field$_2$ name> : <field$_2$ type>
  ...
  <field$_n$ name> : <field$_n$ type>
  constructor <constructor name>
\end{lstlisting}

\paragraph{Parameters}
The parameters, or indices, desugar to the type constructor declaration. A parameter can be either (1) a
name bound to a type, e.g. \texttt{(n : Nat)} or (2) a name by itself,
e.g. \texttt{a}. If a name is not given a specific type it is assumed to be of type
\texttt{Type}, the type of types. 

The parameters are explicit arguments to the type
constructor and implicit arguments to the data constructor. A parameter can be
marked as implicit by surrounding it with curly braces, replacing any existing
parentheses.

\paragraph{Fields}
The new records fields are specified as any number of colon-separated name-type pairs. All
fields are explicit arguments to the data constructor. Similar to the original Idris-record implementation, projection and
update functions are generated for each field.
\paragraph{Data Constructor}
The last line of the declaration is an, optional, custom data constructor
name. This allows the programmer to define a specific name for the data
constructor. This can be omitted, in which case a name is generated during
elaboration. The elaborator will try to make a sensible constructor name based
on the type name, for example \texttt{MkT} for a type with name \texttt{T}.

\begin{figure}[h]
\begin{lstlisting}
record Pair a b where
  fst : a
  snd : b

record Foo (param : Nat) where
  num : Int

record C {a} (n : Nat) where
  {unhelpful : a}
  xs : Vect n a

corecord Stream a where
  head : a
  tail : Stream a
  constructor (::) 

\end{lstlisting}
  \caption{Record examples.}
  \label{fig:new_record_examples}
\end{figure}

\begin{figure}[h]
\begin{lstlisting}
data Pair : (a : Type) -> (b : Type) -> Type where
  MkPair : {a : Type} -> {b : Type} -> 
                (fst : a) -> (snd : b) -> Pair a b

data Foo : (param : Nat) -> Type where
  MkFoo : {param : Nat} -> (num : Int) -> Foo param

data C : {a : Type} -> (n : Nat) -> Type where
  MkC {a : Type} -> {unhelpful : a} -> 
           (n : Nat) -> (xs : Vect n a) -> C {a=a} n

codata Stream : (a : Type) -> Type where
  (::) : {a : Type} -> (head : a) -> (tail : Stream a) -> Stream a
\end{lstlisting}
  \caption{The (co)data declaration resulting from desugaring the (co)record
    declarations from Figure~\ref{fig:new_record_examples}}
  \label{fig:new_record_examples_desugared}
\end{figure}

In Figure~\ref{fig:new_record_examples} a few examples of the above described
syntax can be seen, and Figure~\ref{fig:new_record_examples_desugared} depicts how
they desugar. For the sake of simplicity only the desugared data and codata
declarations are shown, and the generated projection and update functions are omitted.

%%% Local Variables:
%%% mode: latex
%%% TeX-master: "../copatterns-thesis"
%%% End:


\begin{landscape}
\chapter{Inference of Guarded Recursion: Proofs}
\label{app:infer-guard-recurs}
\label{cha:example-inferences}

  \begin{prooftree}
\AxiomC{(1)}
\AxiomC{}
\RightLabel{$\infer _{Refl}$}
\UnaryInfC{$IE \vdash Z : Nat \infer Z : Nat$}
\RightLabel{$\infer _{app}$}
\BinaryInfC{$\begin{matrix} IE \vdash (::)\,Nat\,Z \ : Stream\,Nat \to Stream\,Nat \infer \\
  (_g::)\,Nat\,Z \  : \later ^\kappa_1  \onk{Stream}\,Nat \to
  \onk{Stream}\,Nat\end{matrix}$}
\AxiomC{}
\RightLabel{?}
\UnaryInfC{$IE \vdash \phi(zeros) =\ _gzeros$}
\AxiomC{}
\RightLabel{Var}
\UnaryInfC{$IE \vdash\ _gzeros : \forallk{Stream}\,Nat$}
\RightLabel{$\infer _{Rec\causal}$}
\BinaryInfC{$\begin{matrix} IE \vdash zeros : Stream\,Nat \infer \\ \onk{Next}\,apply\,
  _gzeros : \later ^\kappa_1  \onk{Stream}\,Nat\end{matrix}$}
\RightLabel{$\infer _{app}$}
\BinaryInfC{ $IE \vdash (::)\,Nat\,Z\,zeros : Stream\,Nat\infer \,(_g::)\,Nat\,Z\,(\onk{Next}\,apply\,
  _gzeros) : \forall \kappa.Stream^\kappa \,Nat$}
  \end{prooftree}

\begin{prooftree}
\AxiomC{}
\RightLabel{?}
  \UnaryInfC{$IE \vdash \phi(::) =\ (_g::) $}
\AxiomC{}
\RightLabel{Var}
  \UnaryInfC{$IE \vdash (_g::) : (a : Type) \to a \to
    \later^\kappa_1\onk{Stream}\,a \to \onk{Stream}\,a$}
\RightLabel{$\infer _{\phi}$}
\BinaryInfC{$\begin{matrix}IE \vdash (::) : (a : Type) \to a \to Stream\,a \to
    Stream\,a \infer \\
    (_g::) : (a : Type) \to a \to \later^\kappa_1\onk{Stream}\,a \to \onk{Stream}\,a
\end{matrix}$}
\AxiomC{}
\RightLabel{$\infer _{Refl}$}
\UnaryInfC{$IE \vdash Nat : Type \infer Nat : Type$}
\LeftLabel{(1)}
\RightLabel{$\infer _{app}$}
\BinaryInfC{$
IE \vdash (::)\,Nat : Nat \to Stream\,Nat \to Stream\,Nat \infer (_g::)\,Nat \  : Nat \to \later ^\kappa_1  \onk{Stream}\,Nat \to
  \onk{Stream}\,Nat$}
\end{prooftree}

$IE = zeros;\, \causal ;\,\phi, (::,\ _g::), (zeros,\, _gzeros);\, \open;\,
\Gamma,Z : Nat,\ zeros : Stream Nat,\,_gzeros : \forallk{Stream} Nat$

\newpage


\end{landscape}


%%% Local Variables:
%%% mode: latex
%%% TeX-master: "../../copatterns-thesis"
%%% End:


\begin{landscape}
\chapter{Checking of Guarded Recursion: Proofs}
\label{app:check-guard-recurs}
\newcommand{\inCE}[1]{$CE \vdash$ #1}

\begin{figure}[h]
\begin{prooftree}
\AxiomC{(1)}
\AxiomC{\inCE{$Nat \ \texttt{type}$}}
\RightLabel{App}
\BinaryInfC{\inCE{$ (_g::) Nat : Nat \to \later^\kappa_1 \onk{Stream}\ Nat \to
    \onk{Stream}\ Nat$}}
\AxiomC{}
\RightLabel{Var}
\UnaryInfC{\inCE{$ Z : Nat$}}
\RightLabel{App}
\BinaryInfC{\inCE{$ (_g::)\,Nat\,Z \  : \later ^\kappa_1  \onk{Stream}\,Nat \to \onk{Stream}\,Nat$}}
\AxiomC{}
\RightLabel{Var}
\UnaryInfC{\inCE{$_gzeros : \forall \kappa . \onk{Stream}\ Nat$}}
\RightLabel{$Ref_{\causal}$}
\UnaryInfC{\inCE{$\onk{Next}\,
  _gzeros[\kappa] : \later ^\kappa_1  \onk{Stream}\,Nat$}}
\RightLabel{App}
\BinaryInfC{\inCE{$  \,(_g::)\,Nat\,Z\,(\onk{Next}\,
  _gzeros[\kappa]) : Stream^\kappa \,Nat$}}
\end{prooftree}

\begin{prooftree}
\AxiomC{}
\LeftLabel{(1)}
\RightLabel{Var$_{1_\open}$}
\UnaryInfC{\inCE{$(_g::) : (a : Type) \to a \to \later \kappa_1 \onk{Stream}\ a \to
    \onk{Stream}\ a$}}
\end{prooftree}

$CE = _gzeros;\ \causal;\ \open; \ \Gamma,Z : Nat,\ _gzeros
: \forallk{Stream} Nat$

  \caption{Checking of zeros.}
  \label{fig:check_zeros}
\end{figure}

\newpage

\begin{figure}[h]
  \begin{prooftree}
\AxiomC{}

\RightLabel{Var$_{1_\open}$}
\UnaryInfC{\inCE{$(_g::) : (a : Type) \to a \to \later \kappa_1 \onk{Stream}\ a \to
    \onk{Stream}\ a$}}
\AxiomC{\inCE{$Nat \ \texttt{type}$}}
\RightLabel{App}
\BinaryInfC{\inCE{$ (_g::) Nat : Nat \to \later \kappa_1 \onk{Stream}\ Nat \to
    \onk{Stream}\ Nat$}}
\AxiomC{}
\RightLabel{Var}
\UnaryInfC{\inCE{$ Z : Nat$}}
\RightLabel{App}
\BinaryInfC{\inCE{$ (_g::)\,Nat\,Z \  : \later ^\kappa_1  \onk{Stream}\,Nat \to \onk{Stream}\,Nat$}}
\AxiomC{(1)}
\RightLabel{App}
\BinaryInfC{\inCE{$(_g::)\,Nat\,Z\,(\onk{Next} ((\, _gmap[\kappa])\ Nat \ Nat\,S) \tensor^\kappa_1
  \,(\onk{Next}(\, _gnats[\kappa]))) : \onk{Stream}\ Nat$}}
  \end{prooftree}
  
  \begin{prooftree}
    \AxiomC{(2)}
    \AxiomC{}
\RightLabel{Var}
\UnaryInfC{\inCE{$S : Nat \to Nat$}}
\RightLabel{App} 
\BinaryInfC{\inCE{$(\, _gmap[\kappa])\ Nat\ Nat\ S : \onk{Stream}\ Nat \to
    \onk{Stream}\ Nat$}}
\RightLabel{$I_{\onk{Next}}$}
\UnaryInfC{\inCE{$\onk{Next} ((\, _gmap[\kappa])\ Nat\ Nat\ S) : \later^\kappa_1 (\onk{Stream}\ Nat \to \onk{Stream}\ Nat)$}}
\AxiomC{}
\RightLabel{Var}
\UnaryInfC{\inCE{$_gnats : \forall \kappa . \onk{Stream}\ Nat$}}
\RightLabel{$Rec_{\causal}$}
\UnaryInfC{\inCE{$(\onk{Next}(\, _gnats[\kappa])) : \onk{Stream}\ Nat$}}
\LeftLabel{(1)}
\RightLabel{$I_{\tensorkappan}$}
\BinaryInfC{\inCE{$    \,\onk{Next} ((\, _gmap[\kappa])\ Nat \ Nat\,S) \tensor^\kappa_1 \,(\onk{Next}(\, _gnats[\kappa])) :
    \later^\kappa_1 Stream^\kappa \,Nat$}}
  \end{prooftree}

  \begin{prooftree}
\AxiomC{(3)}
\AxiomC{\inCE{$Nat \ \texttt{type}$}}
\RightLabel{App}
    \BinaryInfC{\inCE{$(\, _gmap[\kappa])\ Nat : (b : Type) \to (Nat \to b) \to \onk{Stream}\ Nat \to \onk{Stream}\ b$}}
\AxiomC{\inCE{$Nat \ \texttt{type}$}}
\RightLabel{App}
\LeftLabel{(2)}
    \BinaryInfC{\inCE{$(\, _gmap[\kappa])\ Nat\ Nat : (Nat \to Nat) \to \onk{Stream}\ Nat \to \onk{Stream}\ Nat$}}
  \end{prooftree}

  \begin{prooftree}
\AxiomC{}
\RightLabel{Var}
    \UnaryInfC{\inCE{$\ _gmap : \forall \kappa.((a : Type) \to (b : Type) \to
        (a \to b) \to \onk{Stream}\ a \to \onk{Stream}\ b)$}}
\RightLabel{$I_{[\kappa]}$}
\LeftLabel{(3)}
    \UnaryInfC{\inCE{$\, _gmap[\kappa] : (a : Type) \to (b : Type) \to (a \to b) \to \onk{Stream}\ a \to \onk{Stream}\ b$}}    
  \end{prooftree}

$CE =\ _gnats;\ \causal;\ \open; \ \Gamma,Z : Nat,\ _gnats :
\forallk{Stream} Nat, \ _gmap : \forall \kappa.((a : Type) \to (b : Type) \to
        (a \to b) \to \onk{Stream}\ a \to \onk{Stream}\ b)$

  \caption{Checking of nats.}
  \label{fig:check_nat}
\end{figure}
\end{landscape}
%%% Local Variables:
%%% mode: latex
%%% TeX-master: "../../copatterns-thesis"
%%% End:




\end{document}

%%% Local Variables: 
%%% mode: latex
%%% TeX-master: t
%%% End: 
