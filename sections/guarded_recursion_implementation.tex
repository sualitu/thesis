\chapter{Inference of Guarded Recursion}
\todo{Should we mention somewhere that we don't plan on running these? And can
  we even do that? What about evaluation in types?}
\label{cha:infer-guard-recurs}
As mentioned in Section~\ref{sec:less-restr-prod} we wish to extend Idris with
a less restrictive productivity checker using guarded recursion. However, as
we saw in Section~\ref{sec:guarded-recursion} defining guarded recursive
programs can be quite tedious and requires the programmer to understand guarded
recursion in order to write productive programs. We therefore propose a system
where we can check productivity of programs using guarded recursion without
bothering the user.

In this chapter we describe a such system with focus on how it is implemented,
and how we have put theory into practice.
\section{Intuition}
%##########
% Type-directed (undersøg præcis hvad dette betyder) 
% Overgang fra teori til Idris - Hvilke praktiske hensyn tages der?
% Fix-punktet - hvordan forstås det?
% Kun eet ur!
% Kausale og ikke-kausale funktioner
%% Problemet bag: hvorfor er vi nødt til at skelne?
% Eksempel på inferens
%##########
Our system divides into two parts: (1) An inference system that builds a
guarded recursive term from a user written term, and (2) an algorithm that
checks the structure of a term and its type according to the rules from 
Figure~\ref{fig:guarded_recursion_rules_clocks}. While the general idea is that
these two parts work together, such that the inference system first builds a term
for the checking algorithm to check, it is our goal that these two should work
as disjoint modules. This is important as we want our checking algorithm to be a
general-purpose guarded recursion checker, not just an algorithm for checking
the output of the inference system.

Furthermore it is important to note that our system is an extension of
Idris. This means that we do not add or change any core elements of the language, and
therefore do not jeopardise or tamper with the correctness of the Idris type system.

\subsection{Practical Considerations}
When realising a theoretical model such as guarded recursion in an existing
system, there are practical considerations to be taken. Because of the way Idris
is implemented, how we construe the theory might not be immediately obvious. In
the following we will lay out these practical considerations, and how they
relate to the theory.

\paragraph{Target}
As discussed in Chapter~\ref{cha:idris}, an Idris program is represented in
multiple different ways throughout compilation. The time and representation on
which to perform our analysis has great impact on how a solution
unfolds.

Discarding the representations obviously not useful for this analysis (namely
concrete Idris, IBC, and binaries), we are left with two options: \IdrisM and
TT. As guarded recursion is a typing discipline our system needs access to
the types of any term, and since \IdrisM does not contain all type information
our analysis is performed on the core language, TT.

\paragraph{Parameters}
In Section~\ref{sec:guarded-recursion} we discussed and used a guarded fixpoint
operator. Due to the type of this operator ($(\later A \rightarrow A)
\rightarrow A$) the recursive reference will always have a $\later$-type. This
means that any application of a recursive reference to an argument must be done
using $\tensor ^{\kappa}$.

Idris, like many other languages, have named parameters which influence the type
of a function. Looking at the rule for $\tensor ^{\kappa}$ in
Figure~\ref{fig:guarded_recursion_rules} we see that it does not take any type
substitution into account unlike the rules for regular application in Idris from
Figure~\ref{fig:TT_typing_rules}. This means that we cannot use $\tensor ^{\kappa}$ for
parameter application. To better understand this, consider the following example.

\begin{figure}[h]
  \begin{lstlisting}[mathescape]
repeat : $\forall \kappa$ ((a : Type) $\rightarrow$ a $\rightarrow$ Stream$^{\kappa}$ a)
repeat = $\Lambda \kappa$ fix$^{\kappa}$($\lambda{}$rec$.\lambda{}$a$.\lambda{}$n$.$ 
             StreamCons a n ((rec $\tensor ^{\kappa}$ (Next$^{\kappa}$ a)) $\tensor ^{\kappa}$ (Next$^{\kappa}$ n)))
\end{lstlisting}
  \caption{An infinite stream of the same element.}
  \label{fig:repeat_guarded_example}
\end{figure}

In Figure~\ref{fig:repeat_guarded_example} an infinite stream \texttt{repeat} is
defined. It simply just repeats the same element infinitely. Note that even
though this problem also applies to implicit arguments, all arguments in the
example are explicit for simplicity. This means that \texttt{StreamCons} has
type \texttt{(a : Type) $\rightarrow$ a $\rightarrow$ $\later^{\kappa}$ Stream$^{\kappa}$ a $\rightarrow$ Stream$^{\kappa}$ a}.

When checking if this implementation conforms to the guarded rules we have to
check that \texttt{(rec $\tensor ^{\kappa}$ (Next$^{\kappa}$ a)) $\tensor ^{\kappa}$ (Next$^{\kappa}$ n)} has type
\texttt{$\later ^{\kappa}$Stream$^{\kappa}$a} we run into trouble. In order for this to hold,
according to the $\tensor ^{\kappa}$ rule, \texttt{rec $\tensor ^{\kappa}$ (Next$^{\kappa}$ a)} must have
type \texttt{$\later^{\kappa}$(a $\rightarrow$ Stream$^{\kappa}$ a)}. As shown in
Figure~\ref{fig:repeat_typing} we cannot show this as \texttt{rec} has a
different type in our environment to what we are trying to show.

\begin{figure}[h]
\[
\frac { \begin{matrix} \inference { \frac { ? }{ \Gamma '\, \vdash \, rec\, :\, \later
        ^{\kappa}(Type\, \rightarrow \, a\, \rightarrow \, Stream^{\kappa}\, a) } \, 
      \frac {
        \frac {  }{ \Gamma '\, \vdash \, a\, :\, Type } 
      }
      { \Gamma '\, \vdash \,
        Next^{\kappa}\, a\, :\, \later ^{\kappa}\, Type }
    }{ \, \Gamma '\, \vdash \, rec\,
      \tensor ^{\kappa} \, (Next^{\kappa}\, a)\, :\, \later ^{\kappa}(a\, \rightarrow \, Stream^{\kappa}\,
      a)\ }  & \inference { \inference {  }{ \Gamma '\, \vdash \, n\, :\, a }  }{
      \Gamma '\, \vdash \, Next^{\kappa}\, n\, :\, \later ^{\kappa} \, a }  \end{matrix} }{
  \Gamma '\, \vdash \, (rec\, \tensor ^{\kappa} \, (Next^{\kappa}\, a))\, \tensor ^{\kappa} \,
  (Next^{\kappa}\, n))\, :\, \later ^{\kappa} Stream^{\kappa}\, a }
\]
\[
\Gamma '\, =\, \Gamma ,\, rec\, :\, \later ^{\kappa}((a\, :\, Type)\, \rightarrow \,
a\, \rightarrow \, Stream^{\kappa}\, a),\, a\, :\, Type,\, n\, :\, a
\]
  \caption{Part of an attempt to type repeat.}
  \label{fig:repeat_typing}
\end{figure}

One way of solving this could be to have the $\tensor ^{\kappa}$-rule take substitution in
types into account as seen in Figure~\ref{fig:tensor_with_subst}. This, however,
is not yet backed by the theory, and is therefore not an option.

\begin{figure}[h]
\[
\inference { \Gamma \, \vdash \, t\, :\, \later^{\kappa} ((a\, :\, A)\,
  \rightarrow \, B)\quad \Gamma \, \vdash \, u\, :\, \later^{\kappa} A }{
  \Gamma \, \vdash \, t\, \tensor ^{\kappa} \, u\, :\, \later^{\kappa} B[{ u
  }/{ a }] } 
\]
  \caption{Tensor with substitution in types.}
  \label{fig:tensor_with_subst}
\end{figure}

Instead we change what is fixed by the fixpoint. We pull any parameter out,
changing the fixed type, such that \texttt{rec} now has type \texttt{$\later
  ^{\kappa}$(a $\rightarrow$ Stream$^{\kappa}$ a)}. This gives us the definition seen in
Figure~\ref{fig:repeat_guarded_example_new}. Here the \texttt{a} is fixed and
not part of the recursive reference. This allows us to type part of
\texttt{repeat} as exemplified in Figure~\ref{fig:repeat_typing_new}.

\begin{figure}[h]
  \begin{lstlisting}[mathescape]
repeat : (a : Type) $\rightarrow$ $\forall \kappa$ (a $\rightarrow$ Stream$^{\kappa}$ a)
repeat a = $\Lambda \kappa$ fix$^{\kappa}$($\lambda{}$rec$.\lambda{}$n$.$ 
             StreamCons a n (rec $\tensor ^{\kappa}$ (Next$^{\kappa}$$^{\kappa}$ n)))
\end{lstlisting}
  \caption{Repeat with the type parameter fixed.}
  \label{fig:repeat_guarded_example_new}
\end{figure}

\begin{figure}[h]
\[
\frac { \frac {  }{ \Gamma '\, \vdash \, rec\, :\, \later ^{\kappa}(a\, \rightarrow
    \, Stream^{\kappa}\, a) } \, \frac { \Gamma '\, \vdash \, n\, :\, a }{ \Gamma '\,
    \vdash \, Next^{\kappa}\, n\, :\, \later ^{\kappa}\, a }  }{ \, \Gamma '\, \vdash \, rec\,
  \tensor ^{\kappa} \, (Next^{\kappa}\, n)\, :\, \later ^{\kappa}Stream^{\kappa}\, a }
\]
\[
 \Gamma '\, =\, \Gamma ,\, rec\, :\, \later ^{\kappa}(a\, \rightarrow \, Stream^{\kappa}\,
 a),\, a\, :\, Type,\, n\, :\, a
\]

  \caption{Part of typing repeat with fixed type parameter.}
  \label{fig:repeat_typing_new}
\end{figure}

\paragraph{Multiple clauses}
Another problem occurs when we consider functions with multiple clauses,
e.g. functions defined with pattern matching. To understand the problem we must
first step out of the guarded recursive world for a moment and consider the
function \texttt{cycle} in Figure~\ref{fig:cycle_non_guarded}. Note that it is
not a guarded recursive definition, but a standard Idris definition.

\begin{figure}[h]
  \begin{lstlisting}[mathescape]
cycle : Nat $\rightarrow$ Nat $\rightarrow$ Stream Nat
cycle Z m = Z :: (cycle m m)
cycle (S n) m = (S n) :: (cycle n m)
\end{lstlisting}
  \caption{A function cycle in standard Idris}
  \label{fig:cycle_non_guarded}
\end{figure}

When transforming this to a guarded recursive form we want to make the fixed
point over both clauses. An intuitive way to get around this could be to
convert the pattern matching match into case-expressions. This is not possible
as TT does not have case-expressions, only top-level pattern
matching. Furthermore, a case-expression based solution would make even less
sense when we consider dependent pattern matching, since we by removing the
dependent pattern match change the type of the left hand side.

Instead, we can define cycle as two functions: One with the fixed point, and one
with the pattern match, as seen in Figure~\ref{fig:cycle_guarded}. This way, all
clauses of \texttt{cycle} are part of the same fixpoint. The recursive reference
\texttt{ref} is handed to \texttt{cycle} as a parameter. Note that this way of
handling the fixed point also works for functions with only one clause.

\begin{figure}[h]
\begin{lstlisting}[mathescape]
cycle : $\later ^{\kappa}$ (Nat $\rightarrow$ Nat $\rightarrow$ Stream$^{\kappa}$ Nat) $\rightarrow$ 
             Nat $\rightarrow$ Nat $\rightarrow$ Stream$^{\kappa}$ Nat
cycle rec    Z  m = 
         StreamCons Z     (rec $\tensor ^{\kappa}$ (Next$^{\kappa}$ m)) $\tensor ^{\kappa}$ (Next$^{\kappa}$ m)
cycle rec (S n) m = 
         StreamCons (S n) (rec $\tensor ^{\kappa}$ (Next$^{\kappa}$ n)) $\tensor ^{\kappa}$ (Next$^{\kappa}$ m)

cycle$'$ : $\forall \kappa$ (Nat $\rightarrow$ Nat $\rightarrow$ Stream$^{\kappa}$ Nat)
cycle$'$ = $\Lambda \kappa$ fix$^{\kappa}$($\lambda$rec.$\lambda$n.$\lambda$m. cycle rec n m)
\end{lstlisting}
  \caption{A guarded definition of cycle}
  \label{fig:cycle_guarded}
\end{figure}

Again, this solution has short comings. We fix the recursive
reference as \texttt{ref} and thereby also fix its type. This works out fine in
the simply typed case, but once we consider higher-order and especially
dependent types we run into problems. If we consider the function prepend from
Figure~\ref{fig:guarded_prepend}, the \texttt{n} in the type of \texttt{rec} is
fixed, e.g. in the second clause to \texttt{S n}. Since \texttt{xs} has type
\texttt{Vect n a}, this term is not well-typed.

\begin{figure}[h]
\begin{lstlisting}[mathescape]
prepend : (Later' (Vect n a -> Stream a -> Stream a)) -> Vect n a -> Stream a -> Stream a
prepend rec [] s = s 
prepend rec (x :: xs) s = StreamCons x (compose {n=Now} (compose {n=Now} rec (Next xs)) (Next s))

prepend' : Forall (Vect n a -> Stream a -> Stream a)
prepend' = LambdaKappa (fix(\rec, xs, s => prepend rec xs s))
\end{lstlisting}
  \caption{A function prepending a vector on a stream.}
  \label{fig:guarded_prepend}
\end{figure}\todo{Prettify and verify figure}

To circumvent this we eliminate the \texttt{fix} operator all together according
to the elimination rules in Figure~\ref{fig:fix_elim_rules}. By eliminating the
fixpoint, we no longer fix the type of the recursive reference. Because of the
type of the guarded fixed point operator, the type of \texttt{rec} is
$\later^{\kappa}$A, which is also the type of \texttt{Next f}.

\begin{figure}[h]
  \[
\frac { \Delta ,\, \kappa ; \, \Gamma \, \vdash \, f\, =\, { fix }^{ \kappa  }(\lambda
  rec.\, e)\, :\, A }{ \Delta ,\, \kappa \, ; \Gamma \vdash \, f\, =\, e[{ (Next^{ \kappa
    }\, f) }/{ rec }]\, :\, A } fix_{E_1}
\]

\[
\frac { \Delta \, ; \Gamma \vdash \, f\, =\, \Lambda \kappa .{ fix }^{ \kappa  }(\lambda
  rec.\, e)\, :\, \forall \kappa .A }{ \Delta \, ; \Gamma \vdash \, f\, =\, \Lambda
  \kappa .e[{ (Next^{ \kappa  }\, f[\kappa ]) }/{ rec }]\, :\, \forall \kappa .A
} fix_{E_2}
\]
  \caption{Rules for fix elimination}
  \label{fig:fix_elim_rules}
\end{figure}

\paragraph{Singleton Clock}
To simplify the implementation we have limited our system to only have one
clock, rather than a clock environment. Either the clock is open, or the clock
is closed. This changes the rules we discussed in
Section~\ref{sec:guarded-recursion}, as seen in
Figure~\ref{fig:gr_rules_sin_clock}\todo{insert one clock rules}. Here
$\sqcup$ denotes the open clock, and $\sqcap$ the closed. \todo{Explain how we
  came to these rules}

\begin{figure}[h]
  
  \caption{Guarded Recursion rules with singleton clock.}
  \label{fig:gr_rules_sin_clock}
\end{figure}

While deriving this rule set from the original one has not followed a specific
methodology, a general trend is:

\begin{itemize}
\item If a rule required a specific clock in the environment, it now requires an
  open clock.
\item Side conditions talking about a specific clock being not being free in the
  environment are transformed to that no free clocks are allowed in the
  environment at all.
\end{itemize}

Note that the $\kappa$ has gone from a lot of the types and terms. This is
because we no longer talk about specific clocks, but about the
clock. I.e. $\later$ is always on the same clock and as such there is no need to
specify which. Also note that the $\kappa$ remains on some $\forall \kappa$ and
$\Lambda \kappa$. While these still only mention the singleton clock, the
$\kappa$ is kept to disambiguate these from regular quantification and
abstraction. 

This change of course limits the set of programs for which we can infer guarded
recursion, and thereby ensure productivity. In Section~\ref{cha:evaluation} we
discuss how this limitation affects our system.

\paragraph{Modality of Functions}
In general, when defining a guarded recursive function, one has to consider how
to quantify over clocks. For example, in the definition of \texttt{repeat} from
Figure~\ref{fig:repeat_guarded_example_new} the clock quantification is on the
entire type, where as in the definition of \texttt{evens} from
Figure~\ref{fig:guarded_recursion_evens}, the clock quantification is on each
individual \texttt{Stream}. As there we increase the number of arguments we also
increase the number of different way we could quantify the function, making it
increasingly difficult to infer how to place them.

Because of this we have simplified to the problem. A function is either
\emph{causal} or \emph{not causal}. We call this the \emph{modality} of a
function. If a function is causal it means that the quantification is on the
entire type. In this sense \texttt{repeat} is said to be causal. If a function
is not causal the quantification is on each individual guarded type, meaning
that \texttt{evens} is \emph{not} causal.

While inferring the modality of a function is indeed be possible, simply by
trial and error, there might exist not guarded function with both a causal and
not causal guarded implementation. In such a situation the system would have no
way of knowing which modality to assign the function.

Instead of attempting to infer the modality, we let the user assign a modality
to a function. By doing this we ensure that the system does not infer a function
with the wrong modality.\todo{Explain what this means for recursive ref}

\section{Implementation}
In this section we go over how we have realized the above intuition into an
actual Idris implementation. We start by presenting the overall additions to the
Idris compiler environment, and then how the inference and checker systems have
been implemented.

\paragraph{Guarded Recursion Library}
The first step to adding guarded recursion to Idris is to add the guarded
recursive types and functions. We have added these to the Idris built-in
library, and not as a part of the prelude. This is because the user should be
able to check functions for productivity using our guarded recursion checker
without having to rely on the standard library.  While most of the
implementations are straight forward, and can be found in Appendix\todo{Add ref
  to appendix. And add appendix!}, there are interesting parts to discuss. We
have added the notion of \emph{how much later} something is, through the idea of
\texttt{Availability}, seen in Figure~\ref{fig:availability}. This is, in
conjunction with \texttt{Later} done so that we can many $\later$ applications
as a single \texttt{Later} applications.

\begin{figure}[h]
  \begin{lstlisting}[mathescape]
data Availability = Now | Tomorrow Availability

data Later' : Type -> Type where
  Next : {a : Type} -> a -> Later' a

Later : Availability -> Type -> Type
Later Now a = a
Later (Tomorrow n) a = Later' (Later n a)
\end{lstlisting}
  \caption{Availability.}
  \label{fig:availability}
\end{figure}

\paragraph{Guarded Names}
As we want to infer new types and terms, and not override the existing ones, we
need a system for creating a guarded name from an existing user written
name. While this is fairly trivial, it is an important part of our system. We
will hence forth refer to these as the \emph{guarded names} as oppose to their
original \emph{Idris names}. We will mark these guarded names with a subscript
$g$, such that a name $Name$ becomes $_gName$.

During compilation we keep a map from Idris names to their guarded names. This
means that we can for any Idris name find their guarded name, and vice
versa. This comes in handy when we infer the guarded terms.

\paragraph{Clocked Types}
These are made by inferring a new data declaration for any codata or corecord
declaration. This new data declaration is simply constructed during the
elaboration of the original and then elaborated by itself. This means that for
any user written codata or corecord declaration a guarded version is an
intrinsic part of the program. 

A such guarded version of a declaration is fairly straight forward. Guarded
names are given to the type and to all constructors. Any recursive reference in
the type of the constructor is placed under a $\later$-type, in a similar
fashion to how Idris already handles lazy evaluation. The type constructor
remains unchanged. An example of this inferences for a codata declaration can be
seen in Figure~\ref{fig:guarded_stream_inf}. Note that the inferred declaration
is a data declaration, not codata. This is because we do not need both Idris's
built in laziness (\texttt{Inf}), and the guarded recursion laziness ($\later$).

\begin{figure}[h]
\begin{lstlisting}[mathescape]
codata Stream : Type -> Type where
  MkStream : a -> Stream a -> Stream a

data $_g$Stream : Type -> Type where
  $_g$MkStream : a -> $\later_g$Stream a -> $_g$Stream a
\end{lstlisting}
  \caption{Inference of the guarded stream declaration.}
  \label{fig:guarded_stream_inf}
\end{figure}

In the corecord case, we take a similar approach. However, instead of
constructors we have to guard projections. Again, any recursive projection is
guarded such that $A \rightarrow A$ becomes $A \rightarrow \later A$, and all
projections are given a guarded name. Figure~\ref{fig:guarded_tree_inf} shows an
example of such an inference.

\begin{figure}[h]
\begin{lstlisting}[mathescape]
corecord Tree a where
  left : Tree a -> Tree a
  value : Tree a -> a
  right : Tree a -> Tree a

record $_g$Tree a where
  $_g$left : $_g$Tree a -> $\later_g$Tree a
  $_g$value : $_g$Tree a -> a
  $_g$right : $_g$Tree a -> $\later_g$Tree a
\end{lstlisting}
  \caption{Inference of the guarded tree declaration.}
  \label{fig:guarded_tree_inf}
\end{figure}

Furthermore, for each projection we also generate what we call a \emph{forall
  projection}, henceforth denoted as $_\forall Name$. We use this for
projections on quantified types. Consider a coinductive type $A$ with a
projection $p$ of type $A \rightarrow A$, and a variable $x$ of type $\forall
\kappa.A$. In order to perform $p$ on $x$ and maintain the quantification in the
type, we must first apply the clock to $x$, apply $_gp$, and then abstract over the
clock again using $\Lambda \kappa$. This gives us:

\[
\Lambda \kappa . _gp (apply\,x) : \forall \kappa . \later A
\]

The type of the above is isomorphic with $\forall \kappa . A$ according to the
isomorphism described by Rasmus M\o gelberg\,\cite{Mogelberg:2014} seen in
Figure~\ref{fig:quantified_later_iso}. As a short for all of this we simply use
$_\forall p$ which has type $\forall A \rightarrow \forall A$, giving us the
same result. As such these \emph{forall projections} do not add anything new,
they are just a simpler way for us to perform projections on terms of quantified
type. The intuition is that because of Figure~\ref{fig:quantified_later_iso} we
can perform projections on quantified types without having to worrying about
their lateness. Because they are quantified over clocks, they are always available.

\begin{figure}[h]
\[
\forall \kappa .A\cong \forall \kappa .\rhd ^\kappa A
\]
  \caption{A type isomorphism describe by Rasmus M\o gelberg\,\cite{Mogelberg:2014}.}
  \label{fig:quantified_later_iso}
\end{figure}

\subsection{Inference}
\subsection{Checking}
To check that what we infer actually conforms to the guarded recursion rules, we
have implemented a checker. This algorithm simply checks a term to have a given
type within the rule from Figure~\ref{fig:gr_rules_sin_clock}. To the algorithm
less error prone it is implemented in the simplest fashion possible and as
attempts to stick strictly to the typing rules. This means that the checking
algorithm is very simple, conservative, and rigid. It expect types and terms to
be of a specific structure, and it discards anything it can guarantee the
correctness off. 

The checker traverses a term, and unfolds them according to the rules. Any rules
that requires a type equality is done by asking the Idris type system for
conversion equality, and if the type of a term is needed it is checked using the
Idris type checker.

The structure of the code itself aims to take care of as much as the check using
pattern matching. While not always possible, ideally the conclusion of a rule
should be represented by a pattern match, and the premise(s) by a right-hand
side. By doing this we can, through different patterns, be very selective about
what is accepted by the checker and discard anything that does not match the
conclusion of a rule.

Getting more into the actual implementation the checking algorithm is a function
\texttt{check}, with the type signature seen in Figure~\ref{fig:check_type}. The
parameters are as follows:\todo{Is this too low level?}

\begin{itemize}
\item \texttt{Clock} is the clock under which we are checking the term. A clock
  is simply either open or closed.
\item \texttt{Env} is the local environment, usually generated from the
  left-hand side of the clause being checked.
\item \texttt{(Name, Type)} is name and the type of the recursive
  reference. 
\item \texttt{Term} is the term being checked.
\item \texttt{Type} is the type the term is being checked to have.
\item \texttt{Idris Totality} is the result of the algorithm wrapped in the
  Idris monad\todo{Is the Idris monad explained at this point?}. 
\end{itemize}

\begin{figure}[h]
\begin{lstlisting}
check :: Clock -> 
         Env -> 
         (Name, Type) -> 
         Term -> 
         Type -> 
         Idris Totality
\end{lstlisting}  
  \caption{Checking algorithm type.}
  \label{fig:check_type}
\end{figure}

The majority of the conclusions of the rules can be reflected as patterns. If we
wish to e.g. delay a term, the $Next_i$-rule requires that, (1) the clock is
open, (2) the term is on the form $Next t$, and (3) the type is on the form
$\later A$. These requirements are expressed as a pattern in
Figure~\ref{fig:next_pattern}. The view-patterns can be read as regular pattern
match, such that \texttt{next -> Extracted t} is \texttt{Next t}. The premise
from the rule is reflected on the right-hand side. simply check that given the
same environments \texttt{t} has type \texttt{a}.

\begin{figure}[h]
\begin{lstlisting}
check Open g n (next -> Extracted t) (later -> Extracted a) 
     =
     check Open g n t a
\end{lstlisting}
  \caption{$Next_{i}$ pattern. Note the use of view-patterns (the -> syntax).}
  \label{fig:next_pattern}
\end{figure}

\paragraph{No Free Clocks in Environment}
% No free clocks in finished definitions.
% Check everything in local environment with closed clock.
Some rules have a side condition dictating that there must no free clocks in the
variable environment. These are checked by checking that every term in the
variable can be checked under a closed clock. This is the same as no clocks
being free, because any term or type with a free clock would require the clock
to be open in order to be well-formed. 
\paragraph{Identifying the Recursive Reference}
\todo{This paragraph needs more implementation to be written}
%##########
% To faser: Inferens og check - bør kunne beskrives separat
% Inferens - regler
% Check - regler
% Correctness claim - Hvordan kan vi være sikre på at vi inferer guarded
% recursion?
% Preservation of semantics - Har den infererede term samme semantik som
% input-termen?
% Tensor-regel og parametre
%##########

%%% Local Variables:
%%% mode: latex
%%% TeX-master: "../copatterns-thesis"
%%% End:
