%!TEX root = ../copatterns-thesis.tex
\chapter{Implementation}

\section{Desugaring}
% Why do we want to desugar?
Codata definitions and record definitions with projections, along with function definitions described with copatterns, can all be written using existing Idris constructs. In particular, we consider a function such as \texttt{dup} given in Figure\,\ref{fig:dup_copatterns} to be syntactic sugar for the definition given in Figure\,\ref{fig:dup_desugared}. Likewise, the definition of the coinductively defined Stream type with projections is desugared into a constructor-based definition with exactly one constructor.

\begin{figure}
\begin{lstlisting}
corecord Stream : Type -> Type where
  head : Stream a -> a
  tail : Stream a -> Stream a
  constructor mkStream

--| dupNth s n m duplicates every nth element of s, 
--| starting with the mth element
dupNth : Stream a -> Nat -> Nat -> Stream a
head       (dupNth s n m)       = head s
head (tail (dupNth s n Z))      = head s
head (tail (dupNth s n (S m'))) = head (tail s)
tail (tail (dupNth s n Z))      = tail (dupNth s n n)
tail (tail (dupNth s n (S m'))) = tail (dupNth (tail s) n m')
\end{lstlisting}
\caption{A duplication function on streams, defined with copatterns.}
\label{fig:dup_copatterns}
\end{figure}

\begin{figure}
\begin{lstlisting}
codata Stream : Type -> Type where
  mkStream : a -> Stream a -> Stream a

head : Stream a -> a
head (mkStream h _) = h

tail : Stream a -> a
tail (mkStream _ t) = t

dupNth : Stream a -> Nat -> Nat -> Stream a
dupNth s n Z      =
  mkStream (head s)
           (mkStream (head s)
                     (tail (dupNth s n n)))
dupNth s n (S m') =
  mkStream (head s)
           (mkStream (head (tail s))
                     (tail (dupNth (tail s) n m')))
\end{lstlisting}
\caption{A desugared version of the program described in Figure \ref{fig:dup_copatterns}.}
\label{fig:dup_desugared}
\end{figure}

In order to define the desugaring mechanism which allows us to straightforwardly transform programs from a projection-based form to a constructor-based form, we recall that each projection in a type definition defines an elimination rule for that type. For a general coinductive type with product structure $\nu A. B_1 \times B_2 \times \cdots \times B_n$ where $A$ may occur in any of the types $B_i$ (for $1 \leq i \leq n$), we can define \textit{n} elimination rules in the style of Harper\,\cite[Ch. 15]{Harper:2012} as shown in Figure\,\ref{fig:nuA_elim_rules}. Since these rules define all of the ways we can eliminate a value of the given type, we can synthesize them into a single introduction rule which has \textit{n} premises; one for the result of each elimination rule. This property, as illustrated in Figure\,\ref{fig:nuA_intro_rule}, is obviously reversible, such that we can always use our elimination rules on an introduction form and vice versa.

\begin{figure}
\caption{Elimination rules $\pi_1, \pi_2 \ldots \pi_n$ for a type $\nu A. B_1 \times B_2 \times \ldots \times B_n$.}
\label{fig:nuA_elim_rules}
\end{figure}

\begin{figure}
\caption{Introduction rule for a type $\nu A. B_1 \times B_2 \times \ldots \times B_n$ based on information synthesized from the elimination rules.}
\label{fig:nuA_intro_rule}
\end{figure}

Using the intuition of going back and forth between elimination and introduction, we now present our high-level desugaring mechanism which takes constructs from projection-based form to constructor-based form. Note that all desugaring operations happen during or before elaboration.



\subsection{Desugaring (co)record definitions}
Record and corecords definitions are desugared according to the rule defined in Figure\,\ref{desugaring_records}. The type of each projection becomes an input type to the constructor with the given name.

\begin{figure}
\caption{Desugaring rule for record definitions.}
\label{fig:desugaring_records}
\end{figure}

Definitions of records are desugared at 

\subsection{Desugaring left-hand side projections}
Left-hand side projections for values of both record and corecord type can be desugared in the same way. A clause with left-hand side projections is desugared in three steps: expand, reduce, and merge. To illustrate each step, we will show the steps involved in desugaring the \textit{dupNth} function from Figure\,\ref{fig:dup_copatterns}. 
\begin{figure}
\begin{lstlisting}[mathescape]
dupNth : Stream a -> Nat -> Nat -> Stream a
head (dupNth s n m) = 
  head (mkStream (head s) ?tail) 
head (tail (dupNth s n Z)) = 
  head (tail (mkStream ?head (mkStream (head s) ?tailtail)))
head (tail (dupNth s n (S m'))) = 
  head (tail (mkStream ?head (mkStream (head (tail s)) ?tailtail)))
tail (tail (dupNth s n Z)) = 
  tail (tail (mkStream ?head (mkStream ?headtail 
                                       (tail (dupNth s n n)))))
tail (tail (dupNth s n (S m'))) = 
  tail (tail (mkStream ?head (mkStream ?headtail 
                                       (tail (dupNth (tail s) n m')))))
\end{lstlisting}
\caption{Desugaring, step 1: Expand right-hand sides to make projections on a constructor.}
\label{fig:desugar_step1}
\end{figure}

The first step, expansion, is shown in Figure\,\ref{fig:desugar_step1}. Since we desugar in the elaboration phase, we assume that the constructor for a given type,  \textit{mkStream} in this example, is available in the context. During expansion, the right-hand side of a clause with one or more left-hand side projections is expanded such that projections happen on an appropriate constructor, where the original right-hand side is injected as a parameter in the position specified by left-hand side projections. Due to the intuition given above, we do not have a well-formed introduction form (i.e. constructor application) before we know the result of all eliminations. Because each clause only defines one elimination rule, holes (e.g. ?tail) must therefore be left for constructor arguments unknown to a given clause. However, we defer resolving these holes to the merging step. None of the clauses try to access information unknown to them, so no undefined behaviour can arise as a result of this step.

\begin{figure}
\begin{lstlisting}[mathescape]
dupNth : Stream a -> Nat -> Nat -> Stream a
dupNth s n m      = 
  mkStream (head s) ?tail
dupNth s n Z      = 
  mkStream ?head (mkStream (head s) ?tailtail)
dupNth s n (S m') = 
  mkStream ?head (mkStream (head (tail s)) ?tailtail)))
dupNth s n Z))    = 
  mkStream ?head (mkStream ?headtail (tail (dupNth s n n)))
dupNth s n (S m') = 
  mkStream ?head (mkStream ?headtail (tail (dupNth (tail s) n m')))
\end{lstlisting}
\caption{Desugaring, step 2: Reduce clauses by removing projections appearing in the same position on left and right-hand sides.}
\label{fig:desugar_step2}
\end{figure}

Reduction is the second step of desugaring, and is exemplified in Figure\,\ref{fig:desugar_step2}. In this step, each clause is reduced by removing equivalent projections on both side of a clause, similar to how one would reduce a mathematical equation. We are allowed to perform such reductions at this point because all the clauses make right-hand side projections directly on the (as yet unmerged) output of the function, while all the left-hand side projections happen directly on the input.

\begin{figure}
\begin{lstlisting}[mathescape]
dupNth : Stream a -> Nat -> Nat -> Stream a
dupNth s n Z      = 
  mkStream (head s) (mkStream (head s) 
                              (tail (dupNth s n n)))
dupNth s n (S m') = 
  mkStream (head s) 
           (mkStream (head (tail s)) 
                     (tail (dupNth (tail s) n m')))
\end{lstlisting}
\caption{Desugaring, step 3: Merge right-hand sides of compatible clauses.}
\label{fig:desugar_step3}
\end{figure}

The third and final step is merging, shown in Figure\,\ref{fig:desugar_step3}, where the right-hand sides of compatible clauses are merged into a single clause. Two clauses are considered to be compatible if (1) they make equivalent pattern matches on the same arguments or (2) a more specific clause is merged with a more general clause. The first condition means that \texttt{dupNth s n Z} is compatible with \texttt{dupNth s n Z}, but not \texttt{dupNth s n (S~m')}. The second condition is more subtle, expressing that \texttt{dupNth s n Z} is compatible with \texttt{dupNth s n m}, since the former is more specific than the latter. Accordingly, \texttt{dupNth s n m} is not compatible with \texttt{dupNth s n Z}.

\begin{figure}
\begin{lstlisting}[mathescape]
--| Expands the right-hand side to make 
--| projections on a constructor definition
expand ($\pi_i$ (f x)) rhs = $\pi_i$ (mkA ?arg$_1$ ?arg$_2$ $\cdots$ rhs$_i$ $\cdots$ ?arg$_n$)
expand ($\pi_j$ lhs) rhs = 
  $\pi_j$ (expand lhs (mkA ?arg$_1$ ?arg$_2$ $\cdots$ rhs$_j$ $\cdots$ ?arg$_n$))
expand lhs rhs = rhs

--| Reduce an expression by removing equal projections 
--| on the left and right-hand sides of a clause
reduce (f x)    rhs     = f x = rhs
reduce ($\pi_i$ lhs) ($\pi_i$ rhs) = reduce lhs rhs
reduce _ _ = error -- Trying to reduce non-compatible projections

--| Merges the first clause with the second clause
--| if their left-hand sides are compatible
mergeWithClause (lhs$_1$ = rhs$_1$) (lhs$_2$ = rhs$_2$) =
  if compatible (lhs$_1$) (lhs$_2$)
  then (lhs$_1$ = merge (rhs$_1$) (rhs$_2$))
  else (lhs$_1$ = rhs$_1$)

--| Merges the first constructor with the second
merge (mkA ?arg$_1$ ?arg$_2$ $\cdots$ ?arg$_i$ $\cdots$ ?arg$_n$)
      (mkA ?arg$_1$ ?arg$_2$ $\cdots$ rhs$_i$  $\cdots$ ?arg$_n$) 
      = (mkA ?arg$_1$ ?arg$_2$ $\cdots$ rhs$_i$ $\cdots$ ?arg$_n$)
merge (mkA ?arg$_1$ ?arg$_2$ $\cdots$ rhs$_i$ $\cdots$ ?arg$_n$)
      (mkA ?arg$_1$ ?arg$_2$ $\cdots$ ?arg$_i$  $\cdots$ ?arg$_n$) 
      = (mkA ?arg$_1$ ?arg$_2$ $\cdots$ rhs$_i$ $\cdots$ ?arg$_n$)
merge (mkA ?arg$_1$ ?arg$_2$ $\cdots$ rhs$_{i1}$ $\cdots$ ?arg$_n$)
      (mkA ?arg$_1$ ?arg$_2$ $\cdots$ rhs$_{i2}$  $\cdots$ ?arg$_n$) 
      = error -- Trying to merge two different RHS
merge (mkA ?arg$_1$ ?arg$_2$ $\cdots$ mkA($\cdots$)$_{i1}$ $\cdots$ ?arg$_n$)
      (mkA ?arg$_1$ ?arg$_2$ $\cdots$ mkA($\cdots$)$_{i2}$  $\cdots$ ?arg$_n$) 
      = (mkA ?arg$_1$ ?arg$_2$ $\cdots$ merge (mkA($\cdots$)$_{i1}$) (mkA($\cdots$)$_{i2}$) $\cdots$ ?arg$_n$)
merge xs ys = xs

--| Desugars a list of clauses
desugar ((lhs = rhs) :: clauses) =
  let expandedRhs = expand lhs rhs
  in let reduced = reduce lhs expandedRhs
  in foldr mergeWithClause reduced clauses
desugar [] = []
\end{lstlisting}
\caption{High-level formalization of desugaring, given in Haskell-like syntax. The constructor name \texttt{mkA} signifies a constructor for an arbitrary type A, and $\pi_i$ denotes projections.}
\label{fig:desugar_formalization}
\end{figure}

A formalization of the three steps is shown in Figure\,\ref{fig:desugar_formalization}. Expansion and reduction is straightforward. Merging tries to merge one clause with all other clauses of a definition, although only compatible clauses will lead to an actual merge. This presentation makes no mention of implicit arguments, since these are treated in the same way as explicit arguments.

% merge : Constructor A -> Constructor A -> Constructor A
% merge (Constr name args) (Constr name' args') =
%   if name == name'
%   then (Constr name (mergeArgs args args'))
%   else error

% mergeArgs : [Args] -> [Args] -> [Args]
% mergeArgs (? :: xs) (y :: ys) = y :: mergeArgs xs ys
% mergeArgs (x :: xs) (? :: ys) = x :: mergeArgs xs ys
% mergeArgs ((Constr n args) :: xs) ((Constr m args') :: ys) = 
%   if n == m
%   then (Constr n (mergeArgs args args')) :: mergeArgs xs ys
%   else error
% mergeArgs (x :: xs) (y :: ys) = error

% mkRHS : Projection -> LHS -> RHS -> RHS
% mkRHS ($\pi$_i) (fun n) (r_i) = mk 

\subsection{Desugaring right-hand side projections}

%%% Local Variables: 
%%% mode: latex
%%% TeX-master: "../copatterns-thesis"
%%% End: 
