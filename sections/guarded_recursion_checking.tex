\section{The Guarded Recursion Checker}
\label{sec:guard-recurs-check}
The term inferred by the inference system is potentially ill-typed according to the
guarded recursion typing rules. If this is the case, the input term can not be
guaranteed to be productive. We therefore need a way of checking types of
guarded recursive terms.

The checking algorithm is based on the typing rules defined by M\o
gelberg\,\citep{Mogelberg:2014}, although they have been adjusted according to
the restrictions imposed by the singleton clock. In Figure~\ref{fig:gr_rules_sin_clock}, we
present the adjusted typing rules. Each rule is presented alongside the original
rule, and the translation is explained. Note that the
environment is expanded with a recursive reference, $\iota$, a modality, $\Psi$,
and a singleton clock environment. These extensions to the environment 
have the same definitions as those given for the inference environment in
Section~\ref{sec:inference-system}. Also, the $fix^\kappa$-rule from M\o
gelberg's set of rules is missing. Due to fixed point elimination, discussed in
Section~\ref{sec:fixkappa-rule}, the rule for the fixed point operator is
replaced by rules for the recursive reference, for both the causal and the non-causal case.

As the input to the guarded recursion checking system is TT terms, the checking system is a
decoupled extension to the existing TT type system presented in Section~\ref{sec:tt-core-type}. In any situation where
both a standard TT rule and a guarded recursion rule applies, the guarded
recursion rule must be chosen in order to ensure that all side conditions are checked.

% the standard Idris typing
% rules\,\citep{BradyIdrisImpl13} do also apply, however the guarded recursive
% rules take precedence and will be applied if possible.

% New things in env
\begin{figure}
\textbf{The Typing Rules}
\begin{center}
\subcaptionbox{
Looking at the clock environment in the premise $\Delta, \kappa$ it is clear
that this translates to the open clock ($\open$). In the conclusion we translate
$\Delta$ to the closed clock, because $\Delta$ without $\kappa$ is empty as $\kappa$ is
the only clock in a singleton clock system. The side condition $\kappa \notin fc(\Gamma)$ translates to
$nofc(\Gamma)$, because if the only clock, $\kappa$, is not allowed to
be free in $\Gamma$ then no clock is allowed free in $\Gamma$.
}{

\AxiomC{$\Delta, \kappa ; \Gamma \vdash A : Type$}
\AxiomC{$\kappa \notin fc(\Gamma)$}
\BinaryInfC{$\Delta ; \Gamma \vdash \forall \kappa . A : Type$}
\DisplayProof
\quad
    \AxiomC{$\CEopen \vdash A:Type$}
    \AxiomC{$nofc(\Gamma)$}
    \RightLabel{$I_\forall$}
    \BinaryInfC{$\CEclosed \vdash \forall \kappa . A : Type$}
\DisplayProof
}
\vspace{1em}
\subcaptionbox{
The interesting thing to note in this rule is $\kappa \in \Delta$. If $\kappa$ is in
$\Delta$ and $\kappa$ is the only clock, then $\Delta$ must translate to the
open clock for this side condition to hold. This eliminates the side condition, as $\kappa \in \open$ is always true.
}{
\AxiomC{$\Delta ; \Gamma \vdash A : Type$}
\AxiomC{$\kappa \in \Delta$}
\BinaryInfC{$\Delta ; \Gamma \vdash \laterkappa A : Type$}
\DisplayProof
\quad
    \AxiomC{$\CEopen \vdash A:Type$}
    \RightLabel{$I_{\laterkappan}$}
    \UnaryInfC{$\CEopen \vdash \laterkappan A : Type$}
\DisplayProof  
\vspace{1em}
}
\vspace{1em}
\subcaptionbox{
The only change in this rule is how we consider the environment. We translate
$\Delta$ to the open clock, as $\onk{Next}$ implies a need for
a clock.
}{
\AxiomC{$\Delta ; \Gamma \vdash t : A$}
\UnaryInfC{$\Delta ; \Gamma \vdash \onk{Next}\ t : \laterkappa A$}
\DisplayProof
\quad
    \AxiomC{$\CEopen \vdash t : A$}
    \RightLabel{$I_{\onk{Next}}$}
    \UnaryInfC{$\CEopen \vdash \onk{Next}\ t : \laterkappan A$}
\DisplayProof
\vspace{1em}
}
\vspace{1em}
\subcaptionbox{
For the same reason as in the $I_\forall$-rule, $\Delta, \kappa$ becomes the open clock, and
just $\Delta$ becomes the closed clock. As before, $\kappa \notin fc(\Gamma)$
becomes $nofc(\Gamma)$.
}{
\AxiomC{$\Delta, \kappa ; \Gamma \vdash t : A$}
\AxiomC{$\kappa \notin fc(\Gamma)$}
\BinaryInfC{$\Delta ; \Gamma \vdash \Lambda \kappa. t : \forall \kappa. A$}
\DisplayProof
\quad
\AxiomC{$\CEopen \vdash t : A$}
\AxiomC{$nofc(\Gamma)$}
    \RightLabel{$I_{\Lambda \kappa}$}
\BinaryInfC{$\CEclosed \vdash \Lambda \kappa . t : \forall \kappa . A$}
\DisplayProof
}

\AxiomC{$\kappa \notin fc(\Gamma)$}
\AxiomC{$\Delta, \kappa ; \Gamma \vdash A : Type$}
\AxiomC{$\Delta, \kappa' ; \Gamma, \Gamma' \vdash t : \forall \kappa. A$}
\TrinaryInfC{$\Delta, \kappa' ; \Gamma , \Gamma ' \vdash t[\kappa'] : A[{  \kappa'}/{\kappa  }]$}
\DisplayProof
\vspace{1em}

\subcaptionbox{
Since, originally, this rule mentions two different clocks $\kappa$ and
$\kappa'$, we have to look at the premises and the conclusion  one by one to
make sense of them in a singleton clock environment. If $\forall \kappa . A$ is
to be a type, then $A$ must be a type under the open clock. If $t$ is to be of type
$\forall \kappa . A$ it must open a clock, according to $I_{\Lambda \kappa}$,
and thusly be checkable under the closed clock. And finally the
conclusion, if we apply a clock on a term, then a clock exist, thus a clock must
be open. Furthermore, the clock substitution on $A$ makes no sense with only one
clock. As usual,  $\kappa \notin fc(\Gamma)$ becomes $nofc(\Gamma)$.
}{
\AxiomC{$nofc(\Gamma)$}
\AxiomC{$\CEopen \vdash A : Type$}
\AxiomC{$\CEclosed ,\Gamma' \vdash t : \forall \kappa . A$}
    \RightLabel{$I_{[\kappa]}$}
\TrinaryInfC{$\CEopen ,\Gamma ' \vdash t[\kappa] : A$}
\DisplayProof
}
\end{center}
\vspace{2em}
\end{figure}

\begin{figure}
\begin{center}
\ContinuedFloat
\AxiomC{$\Delta ; \Gamma \vdash t : \laterkappa (A \to B)$}
\AxiomC{$\Delta ; \Gamma \vdash u : \laterkappa A$}
\BinaryInfC{$\Delta ; \Gamma \vdash t \tensor^\kappa u : \laterkappa B$}
\DisplayProof

\vspace{1em}
\subcaptionbox{
While there is nothing that states anything about the clock environment in the
original rule, both the later types ($\later$) and later composition ($\tensor$)
requires a clock to make sense. Because of this we require the clocks to be
open.
}{

    \AxiomC{$\CEopen \vdash t : \laterkappan (A \to B) $}
    \AxiomC{$\CEopen \vdash u : \laterkappan A$}
    \RightLabel{$I_{\tensorkappan}$}    
    \BinaryInfC{$\CEopen \vdash t \tensorkappan u : \laterkappan B$}
\DisplayProof  
}
\vspace{1em}
\subcaptionbox{
This rule does not have a direct counter part in the original model, but is
rather a consequence of our fixed point elimination from
Section~\ref{sec:fixkappa-rule}. A causal recursive reference must be on the
form $\onk{Next}\ e[\kappa]$, according to the rule in
Figure~\ref{fig:fix_elim_rules}. This ensures the correct lateness compared to
the one of $\iota$, given that $e = \iota$ and $\iota : \forall \kappa . A$.
}{
\AxiomC{$\CEopencausal \vdash \iota : \forall \kappa . A$}
\RightLabel{$e = \iota \quad Rec_{\causal}$}
\UnaryInfC{$\CEopencausal \vdash \onk{Next}\ e[\kappa] : \laterkappan A$}
\DisplayProof
}
\vspace{1em}
\subcaptionbox{To recover from eliminating the indexed fixed point, this rule ensures that
  all arguments to the non-causal recursive references has a $\later^\kappa_1$
  type.}{
\AxiomC{$\begin{matrix}\CEopennoncausal \vdash (\onk{Next}\ e) \tensor^\kappa_1
    (\onk{Next}\ x_0) \tensor^\kappa_1 \\
    \cdots \tensor^\kappa_1 (\onk{Next}\ x_{m-1}) : \later^\kappa_1(A \to B)
\\ \CEopennoncausal \vdash (\onk{Next}\ x_n) : \later^\kappa_1 A\end{matrix}$}
\RightLabel{$\begin{matrix} n > 0 \\ e = \iota\end{matrix} \quad Rec_{\noncausal_n}$}
\UnaryInfC{$\CEopennoncausal \vdash (\onk{Next}\ e) \tensor^\kappa_1 (\onk{Next}\ x_0) 
  \cdots \tensor^\kappa_1 (\onk{Next}\ x_n) : \later^\kappa_1B$}
\DisplayProof
}
\vspace{1em}
\subcaptionbox{Here, the non-causal references with one argument is
  handled. This rule is necessary as a base case for the $Rec_{\noncausal_n}$ rule.}{
\AxiomC{$\begin{matrix}\CEopennoncausal \vdash (\onk{Next}\ e) : \later^\kappa_1(A \to B)\end{matrix}$}
\AxiomC{$\CEopennoncausal \vdash (\onk{Next}\ x_0) : \later^\kappa_1 A$}
\RightLabel{$e = \iota \quad Rec_{\noncausal_0}$}
\BinaryInfC{$\CEopennoncausal \vdash (\onk{Next}\ e) \tensor^\kappa_1 (\onk{Next}\ x_0) : \later^\kappa_1B$}
\DisplayProof
}
\vspace{1em}
\subcaptionbox{
This rule ensures that the recursive reference has the correct lateness compared
to $\iota$.
}{
\AxiomC{$\CEopennoncausal \vdash \iota : A$}
\RightLabel{$e = \iota \quad Rec_{\noncausal}$}
\UnaryInfC{$\CEopennoncausal \vdash \onk{Next}\ e : \laterkappan A$}
\DisplayProof
}
\vspace{1em}

\subcaptionbox{
This is an extension of the Idris $Var$ rules. They state that types with a free
clock can only occur if clock is open.
}{
\AxiomC{$(\lambda t : \onk{A}) \in \Gamma$}
\RightLabel{$Var_{1_\open}$}
\UnaryInfC{$\CEopen \vdash t : \onk{A}$}
\DisplayProof

\AxiomC{$(\forall t : \onk{A}) \in \Gamma$}
\RightLabel{$Var_{2_\open}$}
\UnaryInfC{$\CEopen \vdash t : \onk{A}$}
\DisplayProof
}

\end{center}
  \caption{Guarded Recursion rules with singleton clock.}
  \label{fig:gr_rules_sin_clock}

\end{figure}
%%% Local Variables:
%%% mode: latex
%%% TeX-master: "../copatterns-thesis"
%%% End:

\subsection{An Example}
In Figure~\ref{fig:repeat_inference_proof} an inference of the guarded recursive
version of the function \texttt{repeat} was shown. In continuation of this, we
show that $_g$\texttt{repeat} is productive according to the guarded recursion
checking rules. The derivation is given in Figure~\ref{fig:repeat_typing_example}. Additional examples of such proofs can be found in Appendix~\ref{app:check-guard-recurs}.

\begin{figure}[H]
\centering
\subcaptionbox{
The type check takes place in this checking environment. From the top the
environment is: The recursive reference is $repeat\ a$, which is a causal
(\causal) function, and due to the earlier described preprocessing, the clock starts as
open (\open). Finally the typing environment $\Gamma$, is extended as shown.
}{
$\begin{matrix*}[l]CE = & _g\text{repeat\ a};
  \\ & \causal ;\\ &

 \open;\\ &
\Gamma,\text{a} : \text{Type},\ \text{n} : \text{a},\,_g\text{repeat} : (\text{a}\ :\ \text{Type}) \to \forall \kappa.(\text{a} \to \onk{\text{Stream}}\ \text{a}),\ 
\\ & \ \ \ _{g}::\ : (\text{a}\ :\ \text{Type})
\to \text{a} \to \later^\kappa_1 \onk{\text{Stream}}\ \text{a} \to \onk{\text{Stream}}\ \text{a}
\end{matrix*}$
}
\end{figure}
\begin{figure}[H]
\centering
\ContinuedFloat
\subcaptionbox{
Starting with the inferred term in the conclusion, we first apply the $App$
rule, an Idris typing rule, as we have a regular application between $(_g::)\ a\
n$ and $(\onk{Next}\,((_grepeat\ a)[\kappa])) \tensor^\kappa_1
(\onk{Next}\,n)$. Repeating the application rule we end up having to show that
$n$ is of type $a$ which we know from the environment.
}{
\AxiomC{(1)}
\AxiomC{}
\RightLabel{$Var_1$}
\UnaryInfC{$CE \vdash n : a$}
\RightLabel{$App$}
\BinaryInfC{$CE \vdash (_g::)\ a\ n : \later^\kappa_1 \onk{Stream}\ a \to
  \onk{Stream}\ a$}
\AxiomC{(2)}
\RightLabel{$App$}
\BinaryInfC{$CE \vdash \,(_g::)\,a\,n\,((\onk{Next}\,((_grepeat\ a)[\kappa])) \tensor^\kappa_1 (\onk{Next}\,
n)) : Stream^\kappa \,a$}
\DisplayProof
}
\end{figure}
\begin{figure}[H]
\centering
\ContinuedFloat
\subcaptionbox{
This application is taken apart in the same way, leaving us with two things we
know from the environment.
}{
\AxiomC{}
\RightLabel{$Var_{1_\open}$}
\UnaryInfC{$CE \vdash (_g::) : (a : Type) \to a \to \later^\kappa_1 \onk{Stream}\ a \to
  \onk{Stream}\ a$}
\AxiomC{}
\RightLabel{$Var_1$}
\UnaryInfC{$CE \vdash a : Type$}
\LeftLabel{(1)}
\RightLabel{$App$}
\BinaryInfC{$CE \vdash (_g::)\ a : a \to \later^\kappa_1 \onk{Stream}\ a \to
  \onk{Stream}\ a$}
\DisplayProof
}
\end{figure}
\begin{figure}[H]
\centering
\ContinuedFloat
\subcaptionbox{
Here, we start by applying the $\tensorkappan$-rule, leaving us with two new
obligations, both $\laterkappan$. To the first we can apply the rule for the
recursive references rule in the causal case. Lastly, the $\onk{Next}\ a$-case
is trivially shown by the $I_{\onk{Next}}$-rule and from the environment.
}
{
\AxiomC{}
\RightLabel{$Var_1$}
\UnaryInfC{$CE \vdash \ _grepeat\ a : \forall \kappa . (a \to \onk{Stream}\ a)$}
\RightLabel{$Rec_\causal$}
\UnaryInfC{$CE \vdash (\onk{Next} (_grepeat\ a))[\kappa] : \later^\kappa_1 (a \to \onk{Stream}\ a)$}
\AxiomC{}
\RightLabel{$Var_1$}
\UnaryInfC{$CE \vdash n : a$}
\RightLabel{$I_{\onk{Next}}$}
\UnaryInfC{$CE \vdash \onk{Next}\ n : \later^\kappa_1 a$}
\LeftLabel{(2)}
\RightLabel{$I_{\tensorkappan}$}
\BinaryInfC{$CE \vdash (\onk{Next} (_grepeat\ a))[\kappa] \tensor^\kappa_1 (\onk{Next}\,
n) : \later^\kappa_1 \onk{Stream}\ a$}
\DisplayProof
}
  \caption{Proof that $_g$repeat is well-typed.}
  \label{fig:repeat_typing_example}
\end{figure}
%%% Local Variables:
%%% mode: latex
%%% TeX-master: "../copatterns-thesis"
%%% End:

%%% Local Variables:
%%% mode: latex
%%% TeX-master: "../copatterns-thesis"
%%% End:
