%!TEX root = ../copatterns-thesis.tex
\chapter{Adding Copatterns}
\label{sec:adding_copatterns}
Copatterns can be added to Idris without changing its core, the TT
calculus. Throughout this chapter, we will show how copatterns can be implemented as
syntactic sugar, i.e. a transformation from \IdrisM{} to \IdrisM{}, such that
all copattern clauses have been reduced to pattern clauses by the
time elaboration is reached.

% What, Why
\section{Intuition}
%#########
% Eliminationsregler => Introduktionsregler
% Top-level produktstruktur (een introduktionsregel)
% Pattern matching?
%#########
Following the discussion in Section~\ref{sec:prod-vs.-copr}, support for
copatterns have only been added for record types, both inductive and
coinductive. This decision is based upon the fact that Idris record types have
an obvious top-level structure, making them ideal companions for
copatterns. All examples are therefore given for record types only.

\subsection{The Duality Between Construction and Observation}
\label{sec:dual-betw-constr}
Desugaring copattern clauses, which give definitions by observation, to pattern
clauses, which give definitions by construction, relies on the duality between
introduction and elimination rules for (co)inductive types. For our first example,
we will consider the simple record type \texttt{Pair} in
Figure~\ref{fig:record_pair}. 
\begin{figure}
\begin{lstlisting}[mathescape]
record Pair (a : Type) (b : Type) where
  MkPair : (fst : a) $\to$ (snd : b) $\to$ Pair a b
\end{lstlisting}
  \centering
  $\begin{matrix} 
    \frac { \Gamma \,\vdash\, x\,:\,a \quad \Gamma \,\vdash\, y\,:\,b }{
      \Gamma\, \vdash\,MkPair\,x\,y\,:\,Pair\,a\,b} \scriptstyle I_{Pair}
  & \frac { \Gamma \,\vdash\, e\,:\,Pair\, a\,b }{ \Gamma\,
    \vdash\,fst(e)\,:\,a} \scriptstyle E_{fst}
  & \frac { \Gamma\, \vdash\, e\,:\,Pair\, a\,b }{ \Gamma\, \vdash\,
    snd(e)\,:\,b } \scriptstyle E_{snd}  \end{matrix}$
  $\vspace{12pt}$
  $\begin{matrix}
      \frac { \Gamma\, \vdash\, MkPair\,x\,y\,:\,Pair\,a\,b }{ \Gamma
      \,\vdash\, fst(MkPair\,x\,y)\,\equiv\,x  } \eta_{fst}  & \frac { \Gamma\, \vdash\,
      MkPair\,x\,y\,:\,Pair\,a\,b }{ \Gamma\, \vdash \,snd(MkPair\,x\,y)\,\equiv\,y } \eta_{snd} \end{matrix}$
  \caption{A record defining a type \texttt{Pair}, along with its introduction,
    elimination, and $\eta$-rules.}
  \label{fig:record_pair}
\end{figure}
The traditional way of creating a value of type \texttt{Pair} is by construction
using pattern clauses, as shown in Figure~\ref{fig:pair_example_pattern}. Here,
we construct a value of \texttt{Pair} using its introduction rule, giving the values
\texttt{``Idris''} and \texttt{42} as arguments. We can observe the data in the
constructed value by using the elimination rules \texttt{fst} and \texttt{snd},
which have corresponding $\eta$-rules as expected (see
Figure~\ref{fig:record_pair}). As seen from
Figure~\ref{fig:pair_example_pattern_typing_derivation}, such observations
provide unsurprising results.
\begin{figure}
\begin{lstlisting}
pattern : Pair String Nat
pattern = MkPair "Idris" 42
\end{lstlisting}
  \caption{A value of \texttt{Pair} defined by construction.}
  \label{fig:pair_example_pattern}
\end{figure}

\begin{figure}
\centering
    $
     \frac { \Gamma \,\vdash\, "Idris"\,:\,String \quad \Gamma \,\vdash\, 42\,:\,Nat }{
      \Gamma\, \vdash\,MkPair\,"Idris"\,42\,:\,Pair\,String\,Nat} \scriptstyle
    I_{Pair}$
   $\vspace{10pt}$
  $\begin{matrix} \frac { \Gamma \,\vdash\,MkPair\,"Idris"\,42\,:\,Pair\, String\,Nat }{ \Gamma\,
    \vdash\,fst(MkPair\,"Idris"\,42)\,:\,String} \scriptstyle E_{fst}
  & \frac { \Gamma\, \vdash\,MkPair\,"Idris"\,42\,:\,Pair\,String\,Nat }{ \Gamma\, \vdash\,
    snd(MkPair\,"Idris"\,42)\,:\,Nat } \scriptstyle E_{snd}  \end{matrix}$
   $\vspace{10pt}$
  $\begin{matrix}
      \frac { \Gamma\, \vdash\, MkPair\,"Idris"\,42\,:\,Pair\,String\,Nat }{ \Gamma
      \,\vdash\, fst(MkPair\,"Idris"\,42)\,\equiv\,"Idris"  } \eta_{fst}  & \frac { \Gamma\, \vdash\,
      MkPair\,"Idris"\,42\,:\,Pair\,String\,Nat }{ \Gamma\, \vdash \,snd(MkPair\,"Idris"\,42)\,\equiv\,42 } \eta_{snd} \end{matrix}$
  \caption{Using the rules for \texttt{Pair} on the example in Figure~\ref{fig:pair_example_pattern}.}
  \label{fig:pair_example_pattern_typing_derivation}
\end{figure}

The idea behind copatterns is that we can treat the (co)data constructor as a
black box (\texttt{MkPair} in above example),
defining only the values we expect to observe when making specific observations
on the constructed (co)data. As an example,
Figure~\ref{fig:pair_example_copatterns} shows how we can use copatterns to
define a program by observation which is equivalent to the definition by
construction in Figure~\ref{fig:pair_example_pattern}. The most striking
difference is that \texttt{pattern$^\prime$} itself is never explicitly
defined. Instead, it is defined implicitly by the two copattern clauses, which
provide all the data necessary to derive a corresponding pattern clause, as in Figure~\ref{fig:pair_example_pattern}. From
this observation, we arrive at the main intuition behind our desugaring
procedure: Any definition given by copattern clauses has an equivalent
definition given by pattern clauses, and this definition can be derived without
providing any further information. To make this intuition concrete, consider the
derivation in
Figure~\ref{fig:pair_elimination_on_introduction_rules_rewrite}. Rewriting with
the reduction rules defined in Figure~\ref{fig:pair_example_copatterns}, the
elimination rules for \texttt{Pair} are used to directly derive the right-hand
side of the \texttt{pattern} example from
Figure~\ref{fig:pair_example_pattern}. 

\begin{figure}
\begin{lstlisting}[mathescape]
pattern$^\prime$ : Pair String Nat
fst pattern$^\prime$ = "Idris"
snd pattern$^\prime$ = 42
\end{lstlisting}
  \caption{A value of \texttt{Pair} defined by construction.}
  \label{fig:pair_example_copatterns}
\end{figure}

\begin{figure}
  \centering
$\frac { \begin{matrix} \frac { \Gamma \, \vdash \,pattern'\, :\,
      Pair\, String\, Nat }{ \Gamma \, \vdash \, "Idris"\,:\,String
    } \scriptstyle E_{fst}   & \frac { \Gamma \, \vdash \,pattern'\, :\, Pair\, String\,
      Nat }{ \Gamma \, \vdash \, 42\,:\,Nat } \scriptstyle E_{snd}  \end{matrix} }{
  \Gamma \, \vdash \, MkPair\, "Idris"\, 42\, :\, Pair\, String\, Nat } \scriptstyle I_{ Pair }$
  \caption{Deriving the right-hand side of \texttt{pattern} from \texttt{pattern'}.}
  \label{fig:pair_elimination_on_introduction_rules_rewrite}
\end{figure}

\subsection{Desugaring Copatterns}
While the derivation in
Figure~\ref{fig:pair_elimination_on_introduction_rules_rewrite} provides a fine
display of our intention, such a direct derivation is only possible for simple
examples which have only one resulting pattern clause. In particular, desugaring
copatterns becomes more complex as the following challenges must be taken into account:

\begin{enumerate}
\item Nested copatterns
\item Pattern matching
\item Subsumed copattern clauses
\item Non-covering definitions
\end{enumerate}

To address each of these challenges, we have devised a desugaring procedure for
copatterns which operates in three steps:

\begin{enumerate}
\item \textbf{Expansion}, in which right-hand sides are expanded into partially
  defined constructor invocations.
\item \textbf{Reduction}, in which copatterns are eliminated from the reduction rules.
\item \textbf{Merging}, in which compatible clauses are merged into one clause.
  \begin{enumerate}
  \item Identify compatible clauses
  \item Merge compatible clauses
  \end{enumerate}
\end{enumerate}

Before discussing how our solution tackles each of the challenges, the inner
workings of the procedure will be illustrated with a simple example. Consider
the program \texttt{zeros} in Figure~\ref{fig:zeros}, which produces a value of
the type \texttt{Stream} defined in Figure~\ref{fig:corecord_stream}. In the
first step of desugaring \texttt{zeros}, expansion, the right-hand side of each
copattern clause is expanded into an explicit projection on a partially defined
Stream. Thus, the \texttt{head} clause of Figure~\ref{fig:desugared_zeros_step1}
may be read as: ``The \texttt{head} of \texttt{zeros} is the \texttt{head} of
the partially defined \texttt{Stream} which has \texttt{Z} as its first
constructor argument''. A corresponding reading exists for the \texttt{tail}
clause. \IdrisM{} metavariables are inserted for constructor arguments where
output has yet to be defined, e.g. \texttt{?zeroshead}.

\begin{figure}
\begin{lstlisting}[mathescape]
corecord Stream (a : Type) where
  head : a
  tail : Stream a
  constructor (::)
\end{lstlisting}
  \centering
  $\begin{matrix} 
    \frac { \Gamma \,\vdash\, x\,:\,a \quad \Gamma \,\vdash\, xs\,:\,Stream\,a }{
      \Gamma\, \vdash\,(x\,::\,xs)\,:\,Stream \,a} \scriptstyle I_{Stream}
  & \frac { \Gamma \,\vdash\, s\,:\,Stream\, a }{ \Gamma\,
    \vdash\,head(s)\,:\,a} \scriptstyle E_{head}
  & \frac { \Gamma\, \vdash\, s\,:\,Stream\, a }{ \Gamma\, \vdash\,
    tail(s)\,:\,Stream\, a } \scriptstyle E_{tail}  \end{matrix}$
  $\vspace{10pt}$
  $\begin{matrix}
      \frac { \Gamma\, \vdash\, (x\,::\,xs\,:\,Stream\,a }{ \Gamma
      \,\vdash\, head(x\,::\,xs)\,\equiv\,x  } \eta_{head}  & \frac { \Gamma\, \vdash\,
      (x\,::\,xs)\,:\,Stream\,a }{ \Gamma\, \vdash \,tail(x\,::\,xs)\,\equiv\,xs } \eta_{tail} \end{matrix}$
  \caption{A coinductive record type \texttt{Stream} defining an infinite list,
    along with introduction, elimination, and $\eta$-rules.}
  \label{fig:corecord_stream}
\end{figure}

\begin{figure}
\begin{lstlisting}[mathescape]
zeros : Stream Nat
head zeros = Z
tail zeros = zeros
\end{lstlisting}
  \caption{A function \texttt{zeros} defining a stream of 0.}
  \label{fig:zeros}
\end{figure}

In the second step, reduction, each clause from
Figure~\ref{fig:desugared_zeros_step1} is reduced by removing equivalent
projections on both side of a clause, similar to how one would reduce a
mathematical equation. The result of the reduction is shown in
Figure~\ref{fig:desugared_zeros_step2}. We are allowed to perform such
reductions at this point because all the clauses make right-hand side
projections directly on the (as yet unmerged) output of the function, while all
the left-hand side projections happen directly on the input.

\begin{figure}
\begin{lstlisting}[mathescape]
zeros : Stream Nat
head zeros = head (Z :: ?zerostail)
tail zeros = tail (?zeroshead :: zeros)
\end{lstlisting}
  \caption{Desugaring \texttt{zeros}, step 1: Expansion.}
  \label{fig:desugared_zeros_step1}
\end{figure}

\begin{figure}
\begin{lstlisting}[mathescape]
zeros : Stream Nat
-- Before: head zeros = head (Z :: ?zerostail)
-- Eliminate 'head' on each side
zeros = Z :: ?zerostail
-- Before: tail zeros = tail (?zeroshead :: zeros)
-- Eliminate 'tail' on each side
zeros = ?zeroshead :: zeros
\end{lstlisting}
  \caption{Desugaring \texttt{zeros}, step 2: Reduction.}
  \label{fig:desugared_zeros_step2}
\end{figure}

The final step, merging, is actually a two-step process, where compatible clauses are
merged with each other such that their right-hand sides are combined into
possibly complete constructor invocations. Some of the right-hand sides may remain
incomplete after the merging step if the definition we are desugaring is not
covering. In Figure~\ref{fig:desugared_zeros_step3a}, the clauses of
\texttt{zeros} which are compatible with each other are identified. Since
\texttt{zeros} is a simple example, the two clauses are compatible with each
other because their left-hand sides are identical. The clauses are
therefore merged into one clause, as shown in Figure~\ref{fig:desugared_zeros_step3b}.

\begin{figure}
\begin{lstlisting}[mathescape]
zeros : Stream Nat
-- The two clauses are compatible with each other
(1) zeros = Z :: ?zerostail
(2) zeros = ?zeroshead :: zeros
\end{lstlisting}
  \caption{Desugaring \texttt{zeros}, step 3a: Identifying compatible
    clauses. The clauses have been numbered for reference.}
  \label{fig:desugared_zeros_step3a}
\end{figure}

\begin{figure}
\begin{lstlisting}[mathescape]
zeros : Stream Nat
(1,2) zeros = Z :: zeros 
\end{lstlisting}
  \caption{Desugaring \texttt{zeros}, step 3b: Merging compatible clauses. The
    numbers indicate that both of the clauses from
    Figure~\ref{fig:desugared_zeros_step3a} have been merged into one clause.}
  \label{fig:desugared_zeros_step3b}
\end{figure}

\begin{definition}[\textit{Compatibility}]
\label{def:compatibility}
  Compatibility is a reflexive, transitive, and anti-symmetric binary relation
  $C$ on pattern clauses which describes whether the pattern in one pattern
  clause is more general or equally general to the pattern in another pattern
  clause. Let $V$ be a function which, given an argument pattern, returns the set of
  values matched by that pattern. For two pattern clauses
  $a\,p_{1}\,p_{2}\,\cdots\,p_{n}\,=\,rhs_{a}$ and
  $b\,q_{1}\,q_{2}\,\cdots\,q_{n}\,=\,rhs_{b}$ of an $n$-ary function,
  $(a,b)\in C$ if $V(p_{i})\subseteq V(q_{i})$ for $0 < i\le n$. For a nullary
  function, all pattern clauses are trivially compatible.
\end{definition}
% if the set of values
% for each argument $i$ matched by the pattern of $a$, $P_{i}(a)$, is a subset of of the set of
% values for each argument $i$ matched by the pattern of $b$, $P_{i}(b)$ i.e.
% $P_{i}(a)\subseteq P_{i}(b)$ for all $0 < i\le n$. 

\subsubsection{Addressing the Challenges}
The three-step desugaring procedure presented above has been devised as a
unified solution to four main challenges that arise when desugaring is performed
as a transformation from \IdrisM{} to \IdrisM{}. As we go through each of the
challenges, the \texttt{dupNth} program in Figure~\ref{fig:dupNth} will serve as
a running example of the desugaring of a non-trivial definition with copattern
clauses.

\paragraph{Challenge 1: Nested Copatterns}
Copatterns, i.e. left-hand side projections, can be nested arbitrarily as long
as the nesting is well-typed. As described in
Chapter~\ref{cha:coind-records-idris}, definitions with copatterns only behave
as intended when all projection names are unique. Therefore, our solution
assumes that projection names not only indicate semantics, but also argument
positions to the corresponding constructor. To illustrate, consider the
transformation of clause (2) from Figure~\ref{fig:dupNth} to the expanded clause
\texttt{(2)} in Figure~\ref{fig:desugared_dupNth_step1}. Because this clause
defines the \texttt{head} of the \texttt{tail} of the output, the expanded
clause has the right-hand side, \texttt{head~s}, inserted at this position in
the partially defined constructor on the right-hand side of (2) in
Figure~\ref{fig:desugared_dupNth_step1}. The same reasoning holds for the remaining clauses.

In general, nested copatterns are handled in the first two steps, expansion and
reduction. Expansion ensures that the original right-hand sides are inserted
into the correct positions in the expanded constructor invocations, while
reduction eliminates all left-hand side projections through equational
reasoning. A different solution to the problem of handling nested copatterns has
been proposed by Setzer et al.\,\citep{Setzer14Unnesting}, in which auxiliary
functions are created for each level of nested copatterns. This approach will be
described for an alternative solution in Section~\ref{sec:copatterns-implementation-discussion}.

\paragraph{Challenge 2: Pattern Matching}
Whenever the copattern clauses for a definition have different patterns on the
left-hand side, these patterns must be preserved in the desugared version of the
definition. After eliminating copatterns in the reduction step
(Figure~\ref{fig:desugared_dupNth_step2}), preservation of pattern matching is
ensured in the first sub-step of merging
(Figure~\ref{fig:desugared_dupNth_step3a}), where compatible clauses are
identified. In the following sub-step (Figure~\ref{fig:desugared_dupNth_step3b}), only compatible clauses are
merged. Merging only compatible clauses ensures that a clause is only merged
with other clauses which match the same or a more general set of input values. As an example, we will show exactly why clause (2) is compatible
with clause (1) as postulated in
Figure~\ref{fig:desugared_dupNth_step3a}. The proof
proceeds in Figure~\ref{fig:compatibility_proof}. Note that if, in Figure~\ref{fig:compatibility_proof}, $m_{1}$ had been
$S\,m_{1}$ instead, $(2,1)\notin C$ since $V(S\,m_{1}) =
\{\,x\,|\,x\,:\,Nat\,\setminus\{Z\}\}$, and $V(Z)\not\subseteq V(S\,m_{1})$.

\begin{figure}
From
Definition~\ref{def:compatibility}, we must show: $(2,1)\in C$, where (2) and (1) are clauses from
Figure~\ref{fig:desugared_dupNth_step3a}.
\\\\
We consider the left-hand sides of (2) and (1), respectively:\\
$lhs_{(2)} = dupNth\,s_{(2)}\,n_{(2)}\,Z$\\
$lhs_{(1)} = dupNth\,s_{(1)}\,n_{(1)}\,m_{(1)}$\\
\\
For each sub-pattern, we define the set of matching values:\\
$V(s_{2}) = \{\,x\,|\,x\,:\,Stream\,a\}$\\
$V(n_{2}) = \{\,x\,|\,x\,:\,Nat\}$\\
$V(Z) = \{Z\}$\\
$V(s_{1}) = \{\,x\,|\,x\,:\,Stream\,a\}$\\
$V(n_{1}) = \{\,x\,|\,x\,:\,Nat\}$\\
$V(m_{1}) = \{\,x\,|\,x\,:\,Nat\}$\\
\\
Having defined these, the following must hold:
\\
$V(s_{2})\subseteq V(s_{1}) = \{\,x\,|\,x\,:\,Stream\,a\}\subseteq\,\{\,x\,|\,x\,:\,Stream\,a\}\,\qed$\\
$V(n_{2})\subseteq V(n_{1}) =
\{\,x\,|\,x\,:\,Nat\}\subseteq\,\{\,x\,|\,x\,:\,Nat\}\,\qed$\\
$V(Z)\subseteq V(m_{1}) = \{Z\}\subseteq\,\{\,x\,|\,x\,:\,Nat\}\,\qed$\\\\
Consequently, $(2,1)\in C$
  \caption{From Figure~\ref{fig:desugared_dupNth_step3a}, clause (2) is
    compatible with clause (1).}
  \label{fig:compatibility_proof}
\end{figure}


\paragraph{Challenge 3: Subsumed Copattern Clauses}
In addressing Challenge 2, we stated that different left-hand side patterns must
be preserved in the output. Nevertheless, the pattern from clause (1) in
Figure~\ref{fig:desugared_dupNth_step3a} does not occur in the
desugared version of the function in
Figure~\ref{fig:desugared_dupNth_step3b}. Due to the fact that clause (1) is
more general than all the other clauses, it is subsumed by these in the final
output, such that the right-hand side of (1) is a part of the right-hand side of
both the desugared clauses. The set of subsumed copattern clauses for a
definition are exactly the clauses which are compatible with all other clauses.  

\paragraph{Challenge 4: Non-covering Defintions} As Idris allows users to
define functions which are not covering (see
Definition~\ref{def:covering_function}), partial definitions with copatterns
should be possible as well. With our desugaring procedure, the solution to this
challenge is provided ``for free'', since the metavariables which are inserted
during the expansion phase are simply not replaced with any proper definition if
none is given. A non-covering version of the \texttt{dupNth} function is shown
in Figure~\ref{fig:dupNth_partial}, along with its desugared version.

\begin{figure}
\begin{lstlisting}[mathescape]
||| Duplicate every nth element of s, 
||| starting with the mth element.
||| @s the stream
||| @n how often
||| @m where to start
dupNth : Stream a $\to$ Nat $\to$ Nat $\to$ Stream a
(1) head       (dupNth s n m)       = head s
(2) head (tail (dupNth s n Z))      = head s
(3) head (tail (dupNth s n (S m'))) = head (tail s)
(4) tail (tail (dupNth s n Z))      = tail (dupNth s n n)
(5) tail (tail (dupNth s n (S m'))) = 
                              tail (dupNth (tail s) n m')
\end{lstlisting}
  \caption{A function duplicating the nth element of a stream. The numbers have been
  added to each clause for reference, and are not a part of the actual implementation.}
  \label{fig:dupNth}
\end{figure}

\begin{figure}
\begin{lstlisting}[mathescape]
dupNth : Stream a $\to$ Nat $\to$ Nat $\to$ Stream a
(1) head       (dupNth s n m)       =
      head (head s :: ?dupNthtail)
(2) head (tail (dupNth s n Z))      = 
      head (tail (?dupNthhead :: (head s :: ?dupNthtail)))
(3) head (tail (dupNth s n (S m'))) =
      head (tail (?dupNthhead :: 
                 (head (tail s) :: ?dupNthtail)))
(4) tail (tail (dupNth s n Z))      =
      tail (tail (?dupNthhead :: 
                 (?dupNthheadtail :: tail (dupNth s n n))))
(5) tail (tail (dupNth s n (S m'))) =
      tail (tail (?dupNthhead :: 
                 (?dupNthheadtail :: 
                  tail (dupNth (tail s) n m'))))
\end{lstlisting}
  \caption{Desugaring \texttt{dupNth}, step 1: Expansion.}
  \label{fig:desugared_dupNth_step1}
\end{figure}

\begin{figure}
\begin{lstlisting}[mathescape]
dupNth : Stream a $\to$ Nat $\to$ Nat $\to$ Stream a
(1) dupNth s n m       =
      head s :: ?dupNthtail
(2) dupNth s n Z      = 
      ?dupNthhead :: (head s :: ?dupNthtail)
(3) dupNth s n (S m') =
      ?dupNthhead :: (head (tail s) :: ?dupNthtail)
(4) dupNth s n Z      =
      ?dupNthhead :: (?dupNthheadtail :: 
                      tail (dupNth s n n))
(5) dupNth s n (S m') =
      ?dupNthhead :: (?dupNthheadtail :: 
                      tail (dupNth (tail s) n m'))
\end{lstlisting}
  \caption{Desugaring \texttt{dupNth}, step 2: Reduction.}
  \label{fig:desugared_dupNth_step2}
\end{figure}

\begin{figure}
\begin{lstlisting}[mathescape]
dupNth : Stream a $\to$ Nat $\to$ Nat $\to$ Stream a
-- The 'dupNth s n m' clause is compatible with 
-- all of the other clauses.
(1) dupNth s n m = head s :: ?dupNthtail
-- compatible with (1), (2), and (4)
(2) dupNth s n Z      = 
     ?dupNthhead :: (head s :: ?dupNthtail)
-- compatible with (1), (3), and (5)
(3) dupNth s n (S m') =
     ?dupNthhead :: (head (tail s) :: ?dupNthtail)
-- compatible with (1), (2), and (4)
(4) dupNth s n Z      =
     ?dupNthhead :: (?dupNthheadtail :: 
                     tail (dupNth s n n))
-- compatible with (1), (3), and (5)
(5) dupNth s n (S m') =
     ?dupNthhead :: (?dupNthheadtail :: 
                     tail (dupNth (tail s) n m'))
\end{lstlisting}
  \caption{Desugaring \texttt{dupNth}, step 3a: Identifying compatible
    clauses. The clauses have been numbered for reference.}
  \label{fig:desugared_dupNth_step3a}
\end{figure}

\begin{figure}
\begin{lstlisting}[mathescape]
dupNth : Stream a $\to$ Nat $\to$ Nat $\to$ Stream a
(1,2,4) dupNth s n Z = 
          head s :: (head s :: tail (dupNth s n n))
(1,3,5) dupNth s n (S m') = 
          head s :: 
          (head (tail s) :: tail (dupNth (tail s) n m'))
\end{lstlisting}
  \caption{Desugaring \texttt{dupNth}, step 3b: Merging compatible
    clauses. The numbers indicate which of the clauses from
    Figure~\ref{fig:desugared_dupNth_step3a} that have been merged into each clause.}
  \label{fig:desugared_dupNth_step3b}
\end{figure}

\begin{figure}
\begin{lstlisting}[mathescape]
dupNth : Stream a $\to$ Nat $\to$ Nat $\to$ Stream a
head (tail (dupNth s n Z))      = head s
head (tail (dupNth s n (S m'))) = head (tail s)
tail (tail (dupNth s n Z))      = tail (dupNth s n n)
tail (tail (dupNth s n (S m'))) = tail (dupNth (tail s) n m')
\end{lstlisting}
\begin{lstlisting}[mathescape]
dupNth : Stream a $\to$ Nat $\to$ Nat $\to$ Stream a
dupNth s n Z = 
 ?dupNthhead :: (head s :: tail (dupNth s n n))
dupNth s n (S m') = 
 ?dupNthhead1 :: (head (tail s) :: 
                  tail (dupNth (tail s) n m'))
\end{lstlisting}

  \caption{Above: A non-covering version of \texttt{dupNth}, where the \texttt{head}
    clause has been omitted. Below: its desugared version.}
  \label{fig:dupNth_partial}
\end{figure}


% How
\section{Implementing Copatterns in Idris}

%\subsection{Implementing Copatterns}
%#############
% Hvordan foregår desugaring? 
% Hvordan implementeres intuitionen?
%% Hvilke dele af Idris (jf. figur) berører vi?
% Hvorfor desugaring, i modsætning til elaboration? (mere code sharing)
% Hvorfor udvides TT ikke?

% Identifying copatterns
% Pattern matching
% where-blocks
% Parsing?
% with-rule
% Eksempler (simpelt + avanceret)
% Discussion
%############

\subsection{Parsing Copatterns}
\label{sec:parsing-copatterns}
Identifying copatterns in the concrete Idris syntax turns out to be non-trivial,
due to the rather liberal naming scheme. The Idris parser currently identifies
the leftmost ``naked'' name (without any decorations) in a clause to be the name
of a declared function, a rule which can lead to ambiguities when copatterns
come into play. Notably, the situation presented in
Figure~\ref{fig:copatterns_parsing_hell} is allowed, since Idris allows
declarations and definitions to be freely mixed. The problem in this scenario is
that the two \texttt{head zeros} clauses are syntactically identical, but have
completely different semantics: The first occurrence is the definition of the
function \texttt{head}, and is essentially the constant function for \texttt{Z},
while the second occurrence is a copattern clause for \texttt{zeros}, defining
that the \texttt{head} of \texttt{zeros} is \texttt{Z}.
\begin{figure}
\begin{lstlisting}[mathescape]
zeros : Stream Nat
head : Nat $\to$ Nat
head zeros = Z     -- clause for function 'head'
head zeros = Z     -- copattern 'head' for function 'zeros'
tail zeros = zeros -- copattern 'tail' for function 'zeros'
\end{lstlisting}
  \caption{Copatterns can be difficult to parse, due to situations like
    this. The definition of the type \texttt{Stream} can be found in Figure~\ref{fig:corecord_stream}.}
  \label{fig:copatterns_parsing_hell}
\end{figure}

To resolve such ambiguities, our current solution is simply to prefix each
copattern with a special character, such that copatterns are easily
recognized. Using this strategy, the solution to the situation in
Figure~\ref{fig:copatterns_parsing_hell} is given in
Figure~\ref{fig:copatterns_parsing_hell_fixed_ampersand}. The choice of ``\&''
is arbitrary, aside from the fact that more obvious choices, such as ``.'', are
already used for other purposes. Prefixing with a special character (``.'') is also
the solution chosen by Abel and Pientka\,\citep{Abel13Wellfounded}, although
their language is probably not intended for practical use. Seeing as our current
solution introduces syntax which is quite unlike that of other Idris
constructs, alternative approaches to parsing copatterns will be discussed in
Section~\ref{sec:copatterns-implementation-discussion}.

\begin{figure}
\begin{lstlisting}[mathescape]
zeros : Stream Nat
head : Nat $\to$ Nat
head zeros = Z     -- clause for function 'head'
&head zeros = Z     -- copattern 'head' for function 'zeros'
&tail zeros = zeros -- copattern 'tail' for function 'zeros'
\end{lstlisting}
  \caption{Resolving the ambiguity from Figure~\ref{fig:copatterns_parsing_hell}
  by prefixing each copattern with a special character, here '\&'.}
  \label{fig:copatterns_parsing_hell_fixed_ampersand}
\end{figure}

\subsection{Implementing the Desugaring Procedure}
The desugaring procedure is implemented as a transformation from \IdrisM{} terms
to \IdrisM{} terms, in order to allow for as much as code sharing as possible in
the elaboration phase. In terms of Figure~\ref{fig:idris-overview}, copatterns
are desugared in the Desugaring phase. The idea is that if copattern clauses are
reduced to corresponding pattern clauses as early in the compilation process as
possible, the existing elaboration logic for regular pattern clauses can be
reused in its entirety. Early desugaring also has the implication that no
changes to the core of Idris, the TT calculus, are necessary.

\subsubsection{Implementing Expansion, Reduction, and Merging}
A formalization of the three steps of the desugaring procedure is provided in
Figure~\ref{fig:desugaring_formalization}. The \texttt{expand} function uses the
left-hand side projections to guide the expansion. In particular, the indices on
the projections, i.e. the $i$ in $\pi_i$, denotes the argument position at which
the right-hand side ($\rho_i$) must be given to the constructor. Expansion does
not alter left-hand sides, but leaves right-hand sides expanded as
described. The function \texttt{reduce} is quite simple, removing equivalent
projections on both sides of a clause if the projections occur as the leftmost
expressions on both the left-hand and the right-hand side, respectively. Note that since
\texttt{expand} always returns a clause where the projection pattern on both
sides is identical, \texttt{expand} and \texttt{reduce} can be implemented as
one step in practice, if we simply refrain from reapplying the projections to the
output of \texttt{expand}.

The \texttt{merge} function merges the right-hand side of its second argument
into the right-hand side of its first argument, but only if the two input
clauses are compatible. Merging two right-hand sides is a matter of replacing
the metavariables inserted during expansion with proper terms, an operation
implemented in \texttt{merge$^\prime$}.

\subsubsection{Implementing the Compatibility Relation}
The definition of compatibility (see Definition~\ref{def:compatibility})
requires us to make a pairwise comparison of the sets of matched values for the
argument patterns of two pattern clauses. Taking advantage of the fact that the two
argument patterns compared must not only be well-typed, but also have the same
type (since they are in the same argument position), we can implement the compatibility
relation as a term-level comparison on \IdrisM{} terms. Since two argument patterns
can only have a finite number of different forms, it is possible to encode the
compatibility relation in a table, as shown in
Figure~\ref{fig:compatibility_table}. Whenever two clauses are compatible, we may have to perform one or more
substitutions during the merge step to ensure that the right-hand side of a
merged pattern clause does not refer to names which are not in
scope. Concretely, substitution happens with the function \texttt{subst}, called
by \texttt{merge} in Figure~\ref{fig:desugaring_formalization}, where
\texttt{subs} is a collection of substitutions. Necessary substitutions are indicated in
Figure~\ref{fig:compatibility_table} with ``Yes, if subst.'', which means that
the column pattern must be substituted with the row pattern in the right-hand
side of the merged clause. The substitution function \texttt{subst} performs a simple traversal of the
syntax tree for the right-hand side, replacing terms where necessary.

\subsubsection{Handling \texttt{where}-blocks}
When two pattern clauses are merged, their \texttt{where}-blocks must also be
merged into one. As the last step of merging (not shown in
Figure~\ref{fig:desugaring_formalization}), \texttt{where}-blocks are
concatenated, inventing fresh names for definitions where name collisions
occur. As a consequence, the invented name must be substited for the old name on the
merged right-hand side and in the remainder of the where-block involved. 

\begin{figure}
\begin{lstlisting}[mathescape]
--| Expands the right-hand side to make 
--| projections on a constructor definition
expand $\ulcorner$$\pi_i$ (f x) = $\rho_{i}$$\urcorner$ = 
  $\ulcorner$$\pi_i$ (f x) = $\pi_i$ (MkA ?arg$_1$ ?arg$_2$ $\cdots$ $\rho_i$ $\cdots$ ?arg$_n$)$\urcorner$
expand $\ulcorner$$\pi_j$ lhs = $\rho_{j}$$\urcorner$ = 
  $\underline{let}$ $\ulcorner$lhs$^\prime$ = $\rho_{j}^\prime$$\urcorner$ = expand $\ulcorner$lhs = MkA ?arg$_1$ ?arg$_2$ $\cdots$ $\rho_j$ $\cdots$ ?arg$_n$$\urcorner$
  $\underline{in}$ $\ulcorner$$\pi_j$ lhs$^\prime$ = $\pi_j$ $\rho_{j}^\prime$$\urcorner$
expand $\ulcorner$lhs = $\rho$$\urcorner$ = $\ulcorner$lhs = $\rho$$\urcorner$

--| Reduce an expression by removing equal projections 
--| on the left and right-hand sides of a clause
reduce $\ulcorner$$\pi_i$ lhs = $\pi_i$ $\rho$$\urcorner$  = reduce $\ulcorner$lhs = $\rho$$\urcorner$
reduce $\ulcorner$f x = $\rho$$\urcorner$ = $\ulcorner$f x = $\rho$$\urcorner$
reduce _ _ = error -- Trying to reduce non-compatible projections

--| Merges the first clause with the second clause
--| if they are compatible
merge $\ulcorner$lhs$_1$ = $\rho_1$$\urcorner$ $\ulcorner$lhs$_2$ = $\rho_2$$\urcorner$ =
  $\underline{let}$ (isCompatible, subs) = compatible $\ulcorner$lhs$_1$ = $\rho_1$$\urcorner$ $\ulcorner$lhs$_2$ = $\rho_2$$\urcorner$
  $\underline{in}$ $\underline{if}$ isCompatible
     $\underline{then}$ $\underline{let}$ $\ulcorner$$\rho^\prime$$\urcorner$ = merge$^\prime$ $\ulcorner$$\rho_1$$\urcorner$ $\ulcorner$subst $\rho_2$ subs$\urcorner$
         $\underline{in}$ $\ulcorner$lhs$_1$ = $\rho^\prime$$\urcorner$
     $\underline{else}$ $\ulcorner$lhs$_1$ = $\rho_1$$\urcorner$

--| Merges the first constructor with the second
merge$^\prime$ $\ulcorner$MkA e$_{1_1}$ e$_{2_1}$ $\cdots$ ?arg$_{i_1}$ $\cdots$ e$_{n_1}$$\urcorner$
      $\;$$\ulcorner$MkA e$_{1_2}$ e$_{2_2}$ $\cdots$ $\rho_{i_2}$  $\cdots$ e$_{n_2}$$\urcorner$
     =$\;$$\ulcorner$MkA e$_{1_1}$ e$_{2_1}$ $\cdots$ $\rho_{i_2}$ $\cdots$ e$_{n_1}$$\urcorner$
merge$^\prime$ $\ulcorner$MkA e$_{1_1}$ e$_{2_1}$ $\cdots$ $\rho_{i_1}$ $\cdots$ e$_{n_1}$$\urcorner$
      $\;$$\ulcorner$MkA e$_{1_2}$ e$_{2_2}$ $\cdots$ ?arg$_{i_2}$  $\cdots$ e$_{n_2}$$\urcorner$
     =$\;$$\ulcorner$MkA e$_{1_1}$ e$_{2_1}$ $\cdots$ $\rho_{i_1}$ $\cdots$ e$_{n_1}$$\urcorner$
merge$^\prime$ $\ulcorner$MkA e$_{1_1}$ e$_{2_1}$ $\cdots$ $\rho_{i_1}$ $\cdots$ e$_{n_1}$$\urcorner$
      $\;$$\ulcorner$MkA e$_{1_2}$ e$_{2_2}$ $\cdots$ $\rho_{i_2}$  $\cdots$ e$_{n_2}$$\urcorner$
     =$\;$$\ulcorner$MkA e$_{1_1}$ e$_{2_1}$ $\cdots$ $\rho_{i_1}$ $\cdots$ e$_{n_1}$$\urcorner$
merge$^\prime$ $\ulcorner$MkA e$_{1_1}$ e$_{2_1}$ $\cdots$ MkA($\cdots$)$_{i_1}$ $\cdots$ e$_{n_1}$$\urcorner$
      $\;$$\ulcorner$MkA e$_{1_2}$ e$_{2_2}$ $\cdots$ MkA($\cdots$)$_{i_2}$ $\cdots$ e$_{n_2}$$\urcorner$
     =$\;$$\underline{let}$ $\ulcorner$$\rho^\prime$$\urcorner$ = merge$^\prime$$\;$$\ulcorner$MkA($\cdots$)$_{i_1}$ $\urcorner$ $\ulcorner$MkA($\cdots$)$_{i_2}$$\urcorner$
      $\;$$\underline{in}$ $\ulcorner$MkA e$_{1_1}$ e$_{2_1}$ $\cdots$ $\rho^\prime$ $\cdots$ e$_{n_1}$$\urcorner$
merge$^\prime$ $\ulcorner$$\rho_1$$\urcorner$ $\ulcorner$$\rho_2$$\urcorner$ = $\ulcorner$$\rho_1$$\urcorner$
\end{lstlisting}
\caption{High-level formalization of the desugaring procedure for copatterns,
  given in Haskell-like syntax. \IdrisM{} expressions are given in Quine
  brackets, e.g. $\ulcorner e\urcorner$. Expressions with a preceding ``?'' are
  \IdrisM{} metavariables. The constructor name \texttt{MkA} signifies a
  constructor for an arbitrary type A with a top-level product structure, and $\pi_i$ denotes projections,
  i.e. copatterns on left-hand sides and regular record projections on right-hand sides.}
\label{fig:desugaring_formalization}
\end{figure}

\begin{figure}
  
\begin{center}
  \begin{tabular}{ p{3cm} | c || c | c | c | c | }
    \cline{2-6}
    & $C$ & \quine{$y$} & \quine{\_} & \quine{$c^\prime$} & \quine{MkA($y_1\cdots y_n$)} \\ \cline{2-6}\cline{2-6}
    Pattern variable & \quine{$x$} & \begin{tabular}[c]{@{}c@{}}Yes,\\ if subst.\end{tabular} & Yes & No & No\\ \cline{2-6}
    Placeholder & \quine{\_} & Yes & Yes & No & No\\ \cline{2-6}
    Constant& \quine{$c$} & \begin{tabular}[c]{@{}c@{}}Yes,\\ if subst.\end{tabular} & Yes & if $c$ = $c^\prime$ & No \\ \cline{2-6}
    Constructor pattern & \quine{MkA($x_1\cdots x_n$)}
          & \begin{tabular}[c]{@{}c@{}}Yes,\\ if subst.\end{tabular} & Yes & No & if $C$($x_1$,$y_1$)$\cdots$$C$($x_n$,$y_n$)\\
    \cline{2-6}
  \end{tabular}
\end{center}
  \caption{Encoding the compatibility relation, $C$, in a table, describing
    whether two clause arguments given as \IdrisM{} terms can lead to
    compatibility between two clauses. Consider two clauses, $a$ and $b$ for an
    $n$-ary function, where both clauses have $n$ argument patterns on their
    left-hand side. The argument patterns for $a$ are found in the rows, and
    for $b$ in the columns. Further, let $i$ indicate a position in the sequence of
    argument patterns, such that $0 < i \le n$. For two argument patterns, $p_{i_a}$ and $q_{i_b}$,
    where $p_{i_a}$ is the argument pattern in position $i$ for a clause $a$ and $q_{i_b}$
    is the argument pattern in position $i$ for a clause $b$, each cell answers
    the question: ``Does $C$($p_{i_a}$,$q_{i_b}$) hold?'', i.e. ``Is pattern $q_{i_b}$ more general
    or equally general to pattern $p_{i_a}$?''. Possible left-hand side
    constructs not presented in this table are currently not supported for
    definitions with copatterns (see Section~\ref{sec:merg-comp}).}
  \label{fig:compatibility_table}
\end{figure}
\todo{Is the caption for Figure~\ref{fig:compatibility_table} too long?}

% \begin{figure}
% \begin{lstlisting}[mathescape]
% compatible $\ulcorner$lhs$_1$ = $\rho_1$$\urcorner$ $\ulcorner$lhs$_2$ = $\rho_2$$\urcorner$ = compatible$^\prime$ $\ulcorner$lhs$_1$$\urcorner$ $\ulcorner$lhs$_2$$\urcorner$ $[\;]$

% compatible$^\prime$ $\ulcorner$Ref n$\urcorner$ $\ulcorner$Ref n$^\prime$$\urcorner$ subs
%  | isConstructorName n && isConstructorName n$^\prime$ && n == n$^\prime$
%    = subs
%  | isConstructorName n && isConstructorName n$^\prime$ && n != n$^\prime$
%    = $Not\;compatible$
%  | not (isConstructorName n) && isConstructorName n$^\prime$ && n == n$^\prime$
%    = subs




% compatible' :: Context -> PTerm -> PTerm -> [Substitution] -> Maybe [Substitution]
% compatible' ctxt t@(PRef _ n) (PRef _ n') subs =
%   let sub = (n', t)
%   in case (isConName n ctxt, isConName n' ctxt) of
%        (True,  True)  -> if n == n' then return subs else Nothing
%        (True,  False) -> return (sub : subs)
%        (False, True)  -> Nothing
%        (False, False) -> return (if n == n' then subs else sub : subs)
% compatible' _ (PRef _ _) Placeholder subs = return subs
% compatible' _ (PRef _ _) (PApp _ _ _) _ = Nothing
% compatible' ctxt Placeholder (PRef _ n) subs
%  | isConName n ctxt = Nothing
%  | otherwise        = return subs
% compatible' _ Placeholder Placeholder subs = return subs
% compatible' _ Placeholder (PApp _ _ _) _ = Nothing
% compatible' ctxt t@(PApp _ _ _) (PRef _ n) subs
%  | isConName n ctxt = Nothing
%  | otherwise        = return ((n, t) : subs)
% compatible' _ (PApp _ _ _) Placeholder subs = return subs
% compatible' ctxt (PApp _ (PRef _ n) args) (PApp _ (PRef _ n') args') subs
%  | n == n'   = compatibleArgLists ctxt args args' subs
%  | otherwise = Nothing
% compatible' _ _ _ _ = Nothing
% \end{lstlisting}
%   \caption{A formalization of the compatibility relation.}
%   \label{fig:compatibility_formalization}
% \end{figure}

\section{Discussion}
\label{sec:copatterns-implementation-discussion}
% Is this the best approach to desugaring?
% Would a TT-based approach have been better?
% Discuss 'Unnesting of copatterns' approach
% fejlmeddelelser

While our current approach to desugaring copatterns leads to maximum code
sharing in the elaborator, it comes with its own set of challenges. Concretely
limitations exist for the following:

\begin{itemize}
\item Error reporting
\item Disambiguating names
\item Merging and compatibility
\item The \texttt{with}-rule
\end{itemize}

\subsection{Error Reporting}
After a definition with copatterns has been desugared, errors from the remaining
compilation process no longer occur on the definition provided by the user, but on the desugared version. Obviously,
error messages about terms not written by the user are not very helpful. One way
to alleviate this situation would be store information about the original
copattern clauses in the desugared pattern clauses, such that it would be possible to
retrieve which part of a right-hand side that originated from which copattern
clause. Additionally, extending the delaborator \todo{Do we need to describe the
  delaborator somewhere?} to create copattern clauses from
pattern clauses would be necessary to enable the compiler to create error
messages where the user is shown exactly where his or her implementation is faulty.

\subsection{Disambiguating Names}
Because desugaring of copatterns happens prior to elaboration, we cannot
disambiguate names based on type information. The consequence is that if more
than one projection is defined with the same name, the user must fully qualify
the name for each of the copatterns affected at every application-site. Without
additional information, a more advanced disambiguation scheme is not possible.

\subsection{Merging and Compatibility}
\label{sec:merg-comp}
Merging pattern clauses as directed by the compatibility relation (see
Figure~\ref{fig:compatibility_table}) becomes brittle when viewed from a larger
perspective. Each time a new construct which can appear on a left-hand side is
added to \IdrisM{}, the compatibility relation will have to be
updated. Furthermore, some possible left-hand side constructs, such as functions
and $\Sigma$-types, have no obvious reflexive compatibility relation. For
example, the compatibility relation between two function patterns is difficult to
define. Copattern clauses involving these are therefore currently not supported.

\subsection{The \texttt{with}-rule}
The current implementation does not allow users to use the \texttt{with}-rule on
copattern clauses. Due to the merging step, allowing the \texttt{with}-rule on
copattern clauses would require a procedure for merging two
\texttt{with}-rules into one. At the time of writing, we are not aware of any
such procedure.

\subsection{An Alternative Solution}
As a direct translation to pattern clauses exists for any definition with
copattern clauses, elaborating copattern clauses by reusing the existing
elaboration logic for pattern clauses seems like a logical solution at
first. Working at the level of \IdrisM{} gives rise to many limitations, though,
since most of the type information on terms, case trees, and coverage checking
is not available until later in the compilation process, namely during and after
elaboration. We believe that a more robust solution could be achieved by
elaborating copattern clauses separately, i.e. as a direct transformation from
\IdrisM{} to TT, discarding the solution presented
above.

\subsubsection{Parsing and Disambiguating Names}
One benefit of elaborating copattern clauses separately is that it
allows for name disambiguation from type information. The situation shown in
Figure~\ref{fig:copatterns_type_disambiguation} cannot be solved prior to
elaboration, because the type of \texttt{switcher} requires evaluation. The type
of the pattern for \texttt{switcher} changes according to the boolean argument,
and the corresponding copatterns applied to the pattern must therefore be
derived from an evaluated type. In our current implementation, the user is required to fully
qualify the name for each copattern as soon as more than one projection with the
same name is defined, regardless of namespaces.

Another benefit is that copatterns can be identified more
elegantly. As discussed in Section~\ref{sec:parsing-copatterns}, prefixing all
copatterns with a special character is not a satisfactory solution, neither
aesthetically nor in terms of the concrete syntax of the remaining Idris
constructs. If identification of copatterns is delayed until elaboration, the
situation from Figure~\ref{fig:copatterns_parsing_hell} can be resolved by
introducing a per-definition \texttt{copatterns} block, as exemplified in Figure~\ref{fig:copatterns_parsing_hell_fixed_copatterns_block}.
\begin{figure}
\begin{lstlisting}[mathescape]
namespace A
  corecord Stream a where
    head : a
    tail : Stream a
namespace B
  corecord Stream a where
    head : a
    tail : Stream a

switcher : (c : Bool) $\to$ a $\to$ b $\to$ if c
                                    then A.Stream a 
                                    else B.Stream b
copatterns
  head (switcher True  a _) = a
  head (switcher False _ b) = b
  tail (switcher True  a b) = switcher True a b
  tail (switcher False a b) = switcher False a b   
\end{lstlisting}
  \caption{Elaborating copattern clauses separately allows us to solve this
    situation. From top to bottom, the copatterns for \texttt{switcher} are
    identified as: \texttt{A.Stream.head}, \texttt{B.Stream.head},
    \texttt{A.Stream.tail}, and \texttt{B.Stream.tail}.}
  \label{fig:copatterns_type_disambiguation}
\end{figure}

\begin{figure}
\begin{lstlisting}[mathescape]
zeros : Stream Nat
head : Nat $\to$ Nat
head zeros = Z     -- clause for function 'head'
copatterns
  head zeros = Z     -- copattern 'head' for function 'zeros'
  tail zeros = zeros -- copattern 'tail' for function 'zeros'
\end{lstlisting}
  \caption{Resolving the ambiguity from Figure~\ref{fig:copatterns_parsing_hell}
  by placing copattern clauses inside a per-definition \texttt{copatterns} block.}
  \label{fig:copatterns_parsing_hell_fixed_copatterns_block}
\end{figure}


\subsubsection{Elaboration}
Following Setzer et al.\,\citep{Setzer14Unnesting}, the
complex left-hand side structure of the \texttt{dupNth} function from
Figure~\ref{fig:dupNth} can be simplified as in
Figure~\ref{fig:dupNth_unnesting}. This unnested version leaves us with only
simple copatterns, which can trivially be converted to pattern clauses, as
described in Section~\ref{sec:dual-betw-constr}. Figure~\ref{fig:dupNth_unnesting_pattern_clauses} shows the converted
program. Compared to our current implementation, this strategy for simplifying
copatterns has several advantages. First, it eliminates the need for merging
right-hand sides by creating auxiliary functions. At the \IdrisM{} level,
creating auxiliary functions is non-trivial, because computing the types of such
functions may require evaluation. Secondly, the complex compatibility relation
becomes unnecessary, because pattern matching and copatterns are separated, and
all the resulting copatterns are simple. Thirdly, it enables support for the
\texttt{with}-rule on copattern clauses, since clauses using the
\texttt{with}-rule can be separated into distinct functions, as shown in
Figure~\ref{fig:copatterns_with_rule}.

As a final benefit, a separate elaboration eases meaningful error reporting
considerably. Since user-provided constructs are handled directly, pointing the
user to errors in the concrete syntax is possible without having the delaborator
transform pattern clauses to copattern clauses.\todo{Should we have some sort of
conclusion on this chapter?}

\begin{figure}
\begin{lstlisting}[mathescape]
dupNth : Stream a $\to$ Nat $\to$ Nat $\to$ Stream a
copatterns
 head (dupNth s n m) = head s
 tail (dupNth s n m) = dupNth$^\prime$ s n m

dupNth$^\prime$ : Stream a $\to$ Nat $\to$ Nat $\to$ Stream a
dupNth$^\prime$ s n Z = dupNthZ s n
dupNth$^\prime$ s n (S m$^\prime$) = dupNthS s n m$^\prime$

dupNthZ : Stream a $\to$ Nat $\to$ Stream a
copatterns
 head (dupNthZ s n) = head s
 tail (dupNthZ s n) = tail (dupNth s n n)

dupNthS : Stream a $\to$ Nat $\to$ Nat $\to$ Stream a
copatterns
 head (dupNthS s n m$^\prime$) = head (tail s)
 tail (dupNthS s n m$^\prime$) = tail (dupNth (tail s) n m$^\prime$)
\end{lstlisting}
  \caption{Unnesting the \texttt{dupNth} function from Figure~\ref{fig:dupNth}.}
  \label{fig:dupNth_unnesting}
\end{figure}

\begin{figure}
\begin{lstlisting}[mathescape]
dupNth : Stream a $\to$ Nat $\to$ Nat $\to$ Stream a
dupNth s n m = head s :: dupNth$^\prime$ s n m

dupNth$^\prime$ : Stream a $\to$ Nat $\to$ Nat $\to$ Stream a
dupNth$^\prime$ s n Z = dupNthZ s n
dupNth$^\prime$ s n (S m$^\prime$) = dupNthS s n m$^\prime$

dupNthZ : Stream a $\to$ Nat $\to$ Stream a
dupNthZ s n = head s :: tail (dupNth s n n)

dupNthS : Stream a $\to$ Nat $\to$ Nat $\to$ Stream a
dupNthS s n m$^\prime$ = head (tail s) :: tail (dupNth (tail s) n m$^\prime$)
\end{lstlisting}
  
  \caption{Converting the unnested version of \texttt{dupNth} into pattern clauses.}
  \label{fig:dupNth_unnesting_pattern_clauses}
\end{figure}


\begin{figure}
\begin{lstlisting}[mathescape]
filter : (a $\to$ Bool) $\to$ Stream a $\to$ Stream a
head (filter p s) with (filter p (tail s))
  | s$^\prime$ = if p (head s)) then head s else head s$^\prime$
tail (filter p s) = filter p (tail s)
\end{lstlisting}

\begin{lstlisting}[mathescape]
filter : (a $\to$ Bool) $\to$ Stream a $\to$ Stream a
filter p s = headFilter p s :: filter p (tail s)

headFilter : (a $\to$ Bool) $\to$ Stream a $\to$ a
headFilter p s with (filter p (tail s))
  | s$^\prime$ = if p (head s)) then head s else head s$^\prime$
\end{lstlisting}
  \caption{Above: A \texttt{filter} function on \texttt{Stream}s, showcasing the
    combination of copatterns and the \texttt{with}-rule. Below: The reduced
    version.}
\label{fig:copatterns_with_rule}
\end{figure}
%%% Local Variables:
%%% mode: latex
%%% TeX-master: "../copatterns-thesis"
%%% End:
