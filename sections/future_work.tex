
%!TEX root = ../copatterns-thesis.tex
\chapter{Future Work}
\label{cha:future-work}
In this chapter we will discuss how we see our project evolve in the
future. While we have implemented both copatterns and a new productivity checker
in the Idris compiler, the following sections are characterized by what we have learned
during the development of these. Both of the solutions could be improved
and extended in several different interesting ways. 

% Late in the forming of our solutions, we have discovered better
% solutions than the ones we have implemented. Sadly, our discoveries of better
% approaches have surfaced too late to be implemented for this report. What we see
% as future work is therefore pervaded and inspired by our plans to improve our implementation.
\section{Copatterns}
While our implementation of copatterns does work, there are improvements to be
made. In Section~\ref{sec:copatterns-implementation-discussion} we discussed
several topics, and in the following we will go over those which inspire future work.
\subsection{An Alternative Solution}
While we have implemented copatterns as a desugaring from \IdrisM{} definitions
to \IdrisM{} definitions,
we proposed an alternative implementation in
Section~\ref{sec:an-altern-solut}. This solution moves our implementation from
from a desugaring of \IdrisM{} to an elaboration from \IdrisM{} to
\texttt{TT}. This seems like a more robust solution which simplifies the disambiguation
of names, a problem discussed in Section~\ref{sec:pars-disamb-names}.
\subsection{Improved Syntax}
As a solution based on elaborated copatterns rather than desugaring
simplifies name disambiguation, it also opens up new syntactical options. In
Section~\ref{sec:parsing-copatterns}, we discussed that we are not happy with
prefixing all copatterns with a special character, and therefore suggested a
per-definition \texttt{copatterns} block as an alternative in
Section~\ref{sec:pars-disamb-names}. A simple implementation of such a
\texttt{copatterns} block could already now be implemented by making the parser
more clever, but this is not a desirable solution in general.
\subsection{The \texttt{with}-rule for Copattern Clauses}
In the presence of inductive families, the \texttt{with}-rule is a very powerful
tool, allowing users to match on intermediate calculations. In
Section~\ref{sec:textttwith-rule} we explained how the current copatterns implementation
does not work with the \texttt{with}-rule. As such, an interesting future
improvement is to have copattern clauses compatible with the \texttt{with}-rule in Idris.

%### Copatterns ###
% Copatterns in with-rule
% Pattern matching på codata i Idris?
% Måske bedre syntaks
% 
%##################
\section{Guarded Recursion}
\label{sec:guarded-recursion-1}
The most important continuation of our project is to reintroduce the fixed point
operator into our implementation. In Section~\ref{sec:depend-funct-types} we
discussed the implications of such a reintroduction for the inference and
checking system, respectively. In practice, readjusting our implementation to
work with the fixed point operator would be a simple task. Having the fixed
point operator means that all rules for recursive references could be replaced
by one, simple rule. Additionally, this would simplify a working implementation
with support for non-causal functions.

Our intention is to implement a system without fixed point elimination within a reasonable
time frame. Therefore, we will base our suggestions for future work on a new, improved
implementation.

% \subsection{General Improvements}
% We have previously discussed totality dependencies, in
% Section~\ref{sec:total-depend}, and general bugs in our implementation, in
% Section~\ref{sec:our-implementation}. We intend to implement and fix these
% problems alongside a implementation of the fixed point operator.

\subsection{Type Class Instances}
Our implementation does not currently work for type class instances in
Idris. This is not an
inherent flaw of our system, but rather because our implementation does not
treat type classes in any special way.

In Section~\ref{sec:type-class-instances} we briefly touched upon a possible
solution to this problem. For any coinductive type, we not only infer a
guarded type, but we also infer its guarded instances. As an example, we
consider \texttt{Stream} as an instance of the type class \texttt{Functor}:

\begin{lstlisting}[mathescape, title=\idrisBlock]
instance Functor Stream where
  map f (x :: xs) = (f x) :: (map f xs)
\end{lstlisting}

To use the guarded stream, \texttt{Stream}$^\kappa$, as a \texttt{Functor} as well, we
must infer an instance of \texttt{Functor} for \texttt{Stream}$^\kappa$. We do
this by creating a new instance \texttt{Functor Stream}$^\kappa$, and use our
inference system to infer a guarded term for the implementation:

\begin{lstlisting}[mathescape, title=\idrisBlock]
instance Functor Stream$^\kappa$ where
  map f (x $_g$:: xs) = (f x) $_g$:: (Next$^\kappa$ (map f) $\tensorkappan$ xs)
\end{lstlisting}

% Implementing this, while seemingly simple, could be complicated as it requires a
% thorough understanding of type classes in Idris. Both how they are elaborated to
% TT, and how they behave in TT. The main task of adding type
% classes to our implementation is therefore understanding type classes in Idris.

\subsection{Mutual Recursion}
We discussed how we are unable to ensure productivity for mutually recursive
functions, as our analysis prevents infinite recursion by securing that the
recursive reference is only available later. Because mutual recursion can lead
to infinite recursion through other constructs, we can not prove any mutually
defined function definitions productive.

Clouston et al.\,\citep{BirkedalL:guarded-lambda-conf} discuss that one solution
to mutual recursion is to define the fixed point for a product of functions. A
similar approach could be adopted in our system to support mutual
recursion. This would require changes to the inference system, because since it
would have to take more than one recursive reference into account. No changes
would be needed in the checking system, as the fixed point rule works for elements any
(non-dependent) type, and thereby also for a product of functions.

\subsection{User-written Guarded Recursion}
Another interesting continuation of our project is to allow user-written
guarded recursion. In Section~\ref{sec:user-written-guarded}, we discussed the
challenge of allowing the user to write guarded recursion, namely that arbitrary
axioms could be added to the checking system by allowing pattern matching on
values of type $\laterkappa$A. 

% In order to allow user-written guarded recursion
% to exist alongside our inference system, further research Are there other such pitfalls to avoid, or can we simply
% just disallow pattern matching on \texttt{Later} types?

A first step towards user-written guarded recursion in Idris could be to
allow the user to specify guarded recursive types, but still let the system infer the
guarded recursive term. We already let the user specify part of the type, namely
whether it is causal or non-causal. By allowing the user to fully specify the guarded type, we could get
closer to checking productivity of such functions as those described in Section~\ref{sec:function-types}.

\subsection{Multiple Clocks}
We restricted our system to a singleton clock environment to simplify
inference. However, in Section~\ref{sec:multiple-clocks} we argued that
extending the type checker to multiple clocks could work well alongside
user-written guarded recursion. This would not affect the inference system, as
it would still not infer definitions with multiple clocks. 

\subsection{Dependent Types and Guarded Recursion}
In Section~\ref{sec:handling-parameters} we discussed the problems of type
substitution when using later application, and how it has affected our
system. Should future research show that type substitution is indeed sound for
later application, adjusting our system would be straightforward, yet
important. Should the rule from Figure~\ref{fig:tensor_with_subst} prove to be
sound, it would change the $\infer_{\tensorkappan}$-rule and the
$I_{\tensorkappan}$-rule as shown in
Figure~\ref{fig:tensor-rules_with_type_subst}. As such, it is possible to adjust
our system to at least some possible further research in the area of guarded recursion and
dependent types.

\begin{figure}[h]
\centering

\AXD{\begin{matrix} \IEopen \, \vdash \, f\, :\, (a' : A)\, \to \, B\, \infer \,
    g\,:\, \laterkappan ((a : A')\, \to \, B') \\ 
                    \IEopen \, \vdash \, x\, :\, A\, \infer \, y\,:\,\laterkappan A'
     \end{matrix}} 
\RLabel{n \geq 1, f \not = \iota \quad (\infer_{\tensor^{\kappa}_{n}})} 
\UID{\IEopen \, \vdash \, f\, x\, :\, B[{x}/{a}]\, \infer \, g\, \tensor^{\kappa}_{n} \, y\, :\, \laterkappan B'[{y}/{a'}]}
\DisplayProof
  
\vspace{1em}

    \AxiomC{$\CEopen \vdash t : \laterkappan ((a : A) \to B) $}
    \AxiomC{$\CEopen \vdash u : \laterkappan A$}
    \RightLabel{$I_{\tensorkappan}$}    
    \BinaryInfC{$\CEopen \vdash t \tensorkappan u : \laterkappan B[{u}/{a}]$}
\DisplayProof  
  \caption{$\infer_{\tensorkappan}$ and $I_{\tensorkappan}$ rules with type
    substitution.}
  \label{fig:tensor-rules_with_type_subst}
\end{figure}

%%% Local Variables:
%%% mode: latex
%%% TeX-master: "../copatterns-thesis"
%%% End:
