%!TEX root = ../copatterns-thesis.tex
\chapter{Introduction}
%###########
% Contributions
%% Vi vælger at se på guarded recursion, frem for eksempelvis sized types

%###########
Total functional programming has gained an increasing amount of attention in the
past decade. Especially with the advent of proof assistants such as
Coq\,\citep{Coq:manual} and Agda\,\citep{Norell:thesis}, automating
the construction of totality proofs has become more and more important. Combined
with dependent types, total functional programming offers the compelling notion
of programs being \emph{correct by construction}. In order to arrive at this
point, however, we must first ensure that our programs are in fact
\emph{total}. For programs resulting in inductive data, this can be done by constructing a proof of
\emph{termination}. Dually, a proof of \emph{productivity}
must be constructed for programs resulting in coinductive data. Among several
different techniques for automating proofs of productivity, the principle of
syntactic guardedness\,\citep{Coquand94} is arguably the most widely used. Although this principle
is often criticised for being very conservative, it has the benefit of being easy
to implement while requiring no user interaction. More expressive techniques,
such as sized types\,\citep{Abel99terminationchecking} and guarded
recursion\,\citep{Nakano:2000} have emerged as a consequence, but these often
require the user to manually provide proofs of productivity. To bridge at least
part of the gap between the user and the world of total functional programming,
this work presents a productivity analysis in which guarded recursive definitions
can be inferred from user-written definitions with only a minimum amount of user
involvement.

However, better automation of productivity proofs does not necessarily lead us
all the way in terms of having total coinductive definitions. Some dependently
typed languages with explicit coinductive types choose to provide the user with
coinductive definitions by construction, which has the side effect of allowing
dependent pattern matching on coinductive data as well. Dependent pattern
matching on coinductive data can lead to an
unfortunate loss of subject reduction, even when our program is both well-typed
and supposedly total. To avoid this situation, we therefore need another method
for providing coinductive definitions. Hagino\,\citep{Hagino89} pioneered of
giving coinductive definitions by \emph{observation}. In SymML, he introduced
the \emph{merge} construct, an example of which is shown in Figure~\ref{fig:merge_hagino}.

%We choose guarded recursion

\begin{figure}[H]
  \centering
  \begin{lstlisting}[mathescape]
    codatatype 'a inflist = head is 'a & tail is 'a inflist;

    fun iseq1() = merge head <= 1 & tail <= iseq1();
  \end{lstlisting}
  \caption{The \texttt{merge} construct introduced by Hagino for SymML.}
  \label{fig:merge_hagino}
\end{figure}

In the wake of work done by Abel and Pientka\,\citep{Abel13Wellfounded} and
Abel, Pientka, Thibodeau, and Setzer\,\citep{Abel13Copatterns}, a construct
similar to Hagino's \texttt{merge} has recently been implemented in Agda. This
construct is called a \emph{copattern}. By allowing projections on the left-hand
side of function definitions, copatterns enable users to define corecursive
functions in terms of observations. In effect, coinductive data no longer has to
be analyzed with pattern matching, but can more naturally be expressed by the
observations that can be made on it. Our goal is to bring copatterns to Idris.


\subsection{Goals and Contributions}
The goals of this work are as follows:

\begin{itemize}
\item To implement copatterns in Idris.
\item To implement a productivity checker for Idris which widens the range of
  programs that can be automatically proven productive by the Idris compiler.
\end{itemize}

As a result of pursuing these goals, our contributions are:

\begin{itemize}
\item An implementation of copatterns in Idris for (co)inductive record types,
  implemented as a high-level desugaring.
\item An implementation of coindutive record types, where type definitions can be
  given by observations.
\item An implementation of a productivity checker based on guarded recursion. 
  \begin{itemize}
    \item An inference system for inferring guarded recursive definitions
      from user-written definitions, with a minimum amount of user involvement.
    \item A guarded recursion type checking system for ensuring that the
      inferred guarded recursive terms lead to productive definitions.
  \end{itemize}
\end{itemize}
%%% Local Variables:
%%% mode: latex
%%% TeX-master: "../copatterns-thesis"
%%% End:
