% Sune
%#############
% Hvad er copatterns?
% Hvorfor giver copatterns mening, isoleret set?
% Hvor kommer copatterns fra? (Hagino, Abel og venner)
% Definition ved observation (eliminationsregler)
% Eksempler

% Ny viden: Copatterns
%#############

\section{Copatterns}
\label{sec:copatterns}
In functional programming, inductive data is commonly defined by \emph{constructors} in an
elegant and simple fashion. The data can be analyzed and manipulated using
pattern matching, which follows nicely from the finite nature of inductive
data. The growing consensus\todo{Among who?} is that coinductive data should be defined by
\emph{observations} due to its possible infinity. This\todo{Perhaps say this at
  the beginning?} means that one should distingush
between finite and infinite data. Hagino\todo{Hagino introduced a construct for
  synthesizing coinductive data. The distinction between finite and infinite
  data was already known.} first introduced this idea in his SymML
language \,\citep{Hagino89}, where the programmer could define coinductive types
by their \emph{destructors} or \emph{observations}. In other words\todo{We have
  to be precise here. }, inductive
types should be defined by their \emph{introduction rules}, and coinductive types by
their \emph{elimination rules}.

\begin{figure}[h]
\begin{lstlisting}[mathescape]
corecord Stream : Type -> Type where
  head : Stream a -> a
  tail : Stream a -> Stream a 
  constructor MkStream
\end{lstlisting}
\caption{An infinite list defined by observations.}
\label{fig:stream}
\end{figure}

The syntax for \texttt{codata} definitions presented in
Figure~\ref{fig:stream} defines a coinductive type by observations, as oppose to
by constructors. On a \texttt{Stream}, two observations can be made:
\texttt{head} and \texttt{tail}. The former provides us with the first element
of the stream, while the latter gives us with the rest of the infinite stream,
upon which another element can be observed with \texttt{head}. 

\emph{Destructor copatterns}, or simply
\emph{copatterns}\,\citep{Abel13Copatterns}, provide a way of defining functions
on coinductive data in terms of observations. Like pattern matching allows us to
define functions on inductive data by analyzing the structure of the input,
copatterns enable us to make experiments on functions with a result of
coinductive type.

The \texttt{Stream} defined by observations can be used to define a function
\texttt{nats}, an list of all natural numbers, using copatterns, as shown in
Figure~\ref{fig:nats_copatterns}.


\begin{figure}[h]
\begin{lstlisting}[mathescape]
nats : Stream Nat
head nats = Z
tail nats = map S nats
\end{lstlisting}
\caption{A definition of \texttt{nats} using copatterns.}
\label{fig:nats_copatterns}
\end{figure}

Because the result type of \texttt{nats} is defined by observations, we can use
copatterns to define the outcomes of our observations. The intuition is that the
first element of \texttt{nats} is zero (\texttt{Z}), and the rest of the natural
numbers are all the natural numbers incremented by one (\texttt{map S
  nats}). Initially, the \texttt{head} observation will therefore return
\texttt{Z}. Making a \texttt{tail} observation results in a new stream where all
the elements of \texttt{nats} are incremented by one (using the \texttt{S}
constructor for natural numbers). Consequently, the outcome of making a
subsequent \texttt{head} observation is \texttt{S Z}. As we can increment a
natural number infinitely many times, we can also make infinitely many
\texttt{tail} observations, where the result of a \texttt{head} observation will
be incremented for each \texttt{tail} observation. 


Syntactically, projection happens on the outside of definitions when we use
copatterns, as opposed to pattern matching, where projection on parameters
happens inside of definitions. As an example, consider the definition of
\texttt{map} in Figure~\ref{fig:map_copatterns}.


\begin{figure}[h]
\begin{lstlisting}[mathescape]
map : (a -> b) -> Stream a -> Stream b
head (map f s) = f (head s)
tail (map f s) = map f (tail s)
\end{lstlisting}
\caption{The \texttt{map} function defined with copatterns.}
\label{fig:map_copatterns}
\end{figure}

For \texttt{map}, it is clear that the observations are applied on the entire
definition \texttt{map f s}. Projections on the entire definition make sense
because \texttt{map f s} has the coinductive type \texttt{Stream b}, which can
be the subject of observations. In this sense, copatterns can be said to be dual
to pattern matching in the same way that coinductive data is dual to inductive
data. With pattern matching, we can analyze how data has been constructed, and
with copatterns we can define the outcome of observations. Where pattern
matching is a way of processing input, copatterns provide the means for
describing output. 

\subsection{The Anatomy of Copatterns}
\todo{Establish names for different parts of a definition with copatterns here:
  In particular, we need to define the following: clause, pattern clause, pattern,
  argument pattern, copattern clause, copattern, left-hand side projection, observation, and probably more}

\subsection{Existing Implementations of Copatterns}
Copatterns already exist in programming languages, for example in
Agda\,\cite{Norell:thesis}. Here, definitions for coinductive types are quite
interesting. A coinductive type is defined as a record type with a
\texttt{coinductive} flag, an example of which can be seen in
Figure~\ref{fig:agda_stream}. Observations are defined as fields of the record
type.

\begin{figure}[h]
\begin{lstlisting}[mathescape]
record Stream (A : Set) : Set where
  coinductive
  field
    head : A
    tail : Stream A
open Stream
\end{lstlisting}
\caption{\texttt{Stream} definition in Agda.}
\label{fig:agda_stream}
\end{figure}

Definitions with copatterns are almost identical to what has already been
discussed. A simple example \texttt{repeat} in Figure~\ref{fig:agda_repeat}. 

\begin{figure}[h]
\begin{lstlisting}[mathescape]
repeat : {A : Set} -> A -> Stream A
head (repeat a) = a
tail (repeat a) = repeat a 
\end{lstlisting}
\caption{A corecursive \texttt{repeat} function in Agda.}
\label{fig:agda_repeat}
\end{figure}

In Section\,\ref{sec:motivation_copatterns} we discuss the motivation behind the
use of copatterns. Later, in Section\,\ref{sec:adding_copatterns}, we describe
how we have added a syntax for copatterns and for defining coinductive types by
their observations (\texttt{corecord}s) to the programming language Idris.
%%% Local Variables: 
%%% mode: latex
%%% TeX-master: "../../copatterns-thesis"
%%% End: 
