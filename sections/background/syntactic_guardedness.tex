\section{Syntactic Guardedness}
\label{sec:synt-guard-1}
% Sune
%#######
% Hvad er syntactic guardedness?
% Hvordan virker det?
% Hvorfor syntactic guardedness?
% Hvor kommer syntactic guardedness fra? Hvilken problemstilling løser det?
% Hvor kommer syntactic guardedness til kort?

% Ny viden: Hvordan syntactic guardedness virker, begrænsninger ved syntactic guardedness
%#######

A productive function produces a finite number of unfoldings of its output in
finite time (see Definition~\ref{def:productive_function}). In a very intuitive
way, this can be ensured by using the principle of syntactic guardedness for
productivity checking. Where values become structurally smaller by pattern
matching, they become structurally larger by constructor application. Therefore,
the syntactic guardedness principle states that a corecursive definition is
productive if all corecursive calls appear directly under a coinductive
constructor. This has the implication that the productivity of corecursive
functions can be detected by a purely syntactic check.

The idea is as follows: Each time we unfold a corecursive definition, all
corecursive calls must be part of a later unfolding. We enforce this by
demanding that the following unfoldings are built by explicit constructor
invocations. Consider the example of \texttt{zip}, implemented first by
construction, and then by observation:
\begin{lstlisting}[mathescape,title=\idrisBlock]
zip : Stream a $\to$ Stream a $\to$ Stream a
zip s t = MkStream ((head s), (head t)) (zip (tail s) (tail t))
\end{lstlisting}
\begin{lstlisting}[mathescape,title=\idrisBlock]
zip : Stream a $\to$ Stream a $\to$ Stream a
&head zip s t = (head s, head t)
&tail zip s t = zip (tail s) (tail t)
\end{lstlisting}
Here, we enforce that all calls to \texttt{zip} appear directly under an
invocation of \texttt{MkStream}. Thus, we can unfold \texttt{zip s
  t} a finite number of times, and for each unfolding receive \texttt{((head s,
  head t))} and a new \texttt{Stream} built by the corecursive call. The
principle holds for all coinductive types, and easily generalizes to examples
using more interesting types the a simple \texttt{Stream}.

Although syntactic guardedness is a very simple principle, it is also quite
restrictive. If the next unfolding is not provided directly by a
constructor invocation, but indirectly through a call to another productive
function, then the program will not be productive according to the syntactic
guardedness principle. For example, we might define all the odd natural numbers
as follows (where \texttt{(.)} is function composition):
\begin{lstlisting}[mathescape,title=\idrisBlock]
oddNumbers : Stream a
oddNumbers = S Z :: map (S . S) oddNumbers
\end{lstlisting}
\begin{lstlisting}[mathescape,title=\idrisBlock]
oddNumbers : Stream a
&head oddNumbers = S Z
&tail oddNumbers = map (S . S) oddNumbers
\end{lstlisting}
This definition will not be deemed productive by a syntactic guarededness
analysis, since the corecursive call to \texttt{oddNumbers} is not directly
under a call to \texttt{MkStream}, but is instead given as argument to the
productive function \texttt{map}. However, \texttt{oddNumbers} is clearly
productive, since \texttt{map} always provides an additional unfolding. 

The syntactic guardedness principle is widely used, e.g. in Idris, Agda, and
Coq, probably due to its simplicity, but its restrictions does constrain the range
of corecursive definitions we can show to be productive. To widen this range,
other techniques must be used, for instance guarded recursion, which will be
presented in Section~\ref{sec:guarded-recursion}.

% Due to the duality between inductive and coinductive types, it may be compelling to imagine a productivity analysis which, dual to the size-change principle, works by identifying structurally larger values. Such a dual notion of ``size-change productivity'' is exactly the idea behind the syntactic guardedness checkers found in Idris, Agda, and Coq\,\citep{Coq:manual}. Where values become structurally smaller by pattern matching, they become structurally larger by constructor application. Therefore, the \emph{guardedness principle} states that a coinductive definition is guarded if all corecursive calls appear directly under a coinductive constructor. This has the implication that the productivity of corecursive functions can be detected by a purely syntactic check.


%%% Local Variables: 
%%% mode: latex
%%% TeX-master: "../../copatterns-thesis"
%%% End: 
