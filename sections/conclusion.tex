%!TEX root = ../copatterns-thesis.tex
\chapter{Conclusion}
% Copatterns virker i princippet, men en desugaring var nok ikke den bedste
% løsning
% Subject reduction
% Hvornår virker copatterns

% Dependent types og guarded recursion
% Fixed point elimination was a bad idea
% Singleton clock var en succes
% Modality var en succes
% Opdeling mellem inferens og checking
% Flere funktioner kan bevises at være totale i Idris

% Et godt udgangspunkt for videre arbejde
We set out to implement both copatterns and a better productivity checker for
Idris than the existing implementation of the syntactic guardedness
principle. To a certain degree, both of these goals were achieved.

The addition of copatterns was motivated by examining
dependent pattern matching on coinductive data in Idris. As it has been the case
for Coq and Agda, Idris also loses subject reduction when faced with dependent
pattern matching. Our implementation of copatterns was realized as a desugaring from \IdrisM{}
definitions to \IdrisM{} definitions. While this decision seemed like the best
solution at first, it does come with some limitations. Notably, we found that
our transformation had to rely on a somewhat awkward notion of compatibility
between pattern clauses. Also, since the desugaring happens prior to
elaboration, the present implementation does not allow us to disambiguate names
from types. Desugaring copatterns as a pre-elaboration step additionally means
it may become difficult to make the \texttt{with}-rule work. Consequently, we
conclude that an implementation of copatterns as a separate elaboration to TT is
probably a better solution. However, our implementation of copatterns was, for
the most part, a success, seeing as it is now possible to write definitions
using copatterns in Idris.

We implemented a productivity checker based on guarded recursion, which can
automatically construct productivity proofs for programs which are not
productive according to the syntactic guardedness principle. The productivity
checker was implemented as two disjoint systems, an inference system an a
checking system. The inference system is in charge of inferring guarded
recursive definitions from user-written definitions. The checking system ensures
that the inferred definition is well-typed according to the guarded recursive typing rules. This separation of concerns was a success, since it made
the inference system simpler and increased our confidence in the correctness of our
approach. 

One of our goals was to provide a more expressive productivity
analysis than the syntactic guardedness principle, while at the same time
requiring as little user involvement as possible. This was quite successfully
achieved, since users only have to specify the modality (i.e. causal or
non-causal) of each function definition returning coinductive data. Also, we
found that the use of a singleton clock greatly simplified the analysis, without
any perceived loss of expressiveness. 

The choice of eliminating the guarded
recursive fixed point in order to generalize the productivity analysis turned
out to be unfortunate, since this elimination may not be sound. However, for
function definitions without dependent types, fixed point elimination does not
seem to have an impact on the result of the analysis.

In continuation of the above, we find that a dependently typed system does not
fully benefit from the approach, since the theory behind guarded recursion
is not expressive enough. In particular, the rule for later
application does not (yet) hold for dependent function spaces.

Although our productivity analysis has its flaws, our implementation has widened the range of
programs that can be proven productive by the Idris compiler. In conclusion, we
find that while our approach is not yet fully refined, it is a good starting
point for further investigation into the area of automating productivity proofs
based on guarded recursion.


%%% Local Variables:
%%% mode: latex
%%% TeX-master: "../copatterns-thesis"
%%% End:
